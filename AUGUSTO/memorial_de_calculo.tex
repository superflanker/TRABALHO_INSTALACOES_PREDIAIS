\chapter[MEMORIAL DE CÁLCULO]{MEMORIAL DE CÁLCULO}

\section{CÁLCULOS DAS PREVISÕES DE CARGAS}

\subsection{CRITÉRIOS DA NBR-5410}
\label{criterios_carga_nbr}
\subsubsection{CÁLCULO DE CARGA DAS TOMADAS DE USO GERAL}

Para os banheiros, a NBR-5410 prevê pelo menos um ponto de tomada próximo ao lavatório, com cuidados para evitar riscos de choques elétricos.

Para varandas com área inferior a $2$ $m^2$ ou profundidade inferior a $0,80$ $m$, é necessário pelo menos um ponto de tomada, podendo ser instalado próximo ao acesso.

Em salas e dormitórios, deve ser previsto pelo menos um ponto de tomada para cada $5$ $m$, ou fração de perímetro, sendo recomendável equipar pontos com mais de uma tomada.

Para cozinhas, copas, copas-cozinhas, áreas de serviço, lavanderias e locais similares, é necessário pelo menos um ponto de tomada para cada $3,5$ $m$, ou fração de perímetro, sendo que acima da bancada da pia devem ser previstas no mínimo duas tomadas, no mesmo ponto ou em pontos distintos.

Para demais cômodos, o número mínimo de pontos de tomada é definido de acordo com a área, variando de um ponto para áreas de até $2,25$ $m^2$ até um ponto para cada $5$ $m$, ou fração de perímetro, para áreas acima de $6$ $m^2$.

A recomendação é incluir no projeto elétrico tomadas adicionais em locais estratégicos, devido à crescente utilização de aparelhos eletroeletrônicos no dia a dia, para evitar a necessidade de adaptadores, extensões e outros multiplicadores de tomadas.

Caso os locais, como banheiros, cozinhas, copas, copas-cozinhas, áreas de serviço, lavanderias ou locais análogos, tenham 3 ou mais tomadas, 3 delas devem ser instaladas com capacidade de pelo menos 600 VA, e as demais com capacidade de pelo menos 100 VA.

Já nos demais cômodos ou dependências, a norma exige que cada tomada tenha capacidade de pelo menos 100 VA.

\subsubsection{CÁLCULO DE CARGA DAS TOMADAS DE USO ESPECÍFICO}

A NBR-5410 não estabelece uma previsão de carga padrão para tomadas de uso específico, pois essas tomadas são destinadas a equipamentos elétricos específicos, cada um com sua demanda de carga elétrica particular. Dessa forma, a previsão de carga para tomadas de uso específico deve ser feita de acordo com as informações do fabricante do equipamento elétrico a ser conectado a ela. As informações sobre a carga elétrica requerida pelo equipamento podem ser encontradas no manual do usuário ou na placa de identificação do equipamento. A instalação elétrica deve ser dimensionada de forma a atender a demanda de carga de todos os equipamentos elétricos conectados, incluindo as tomadas de uso específico, garantindo assim a segurança e o bom funcionamento do sistema elétrico.

Desta forma, a previsão de carga das tomadas de uso específico é simplesmente a soma da potência instalada de cada equipamento previsto no projeto.

\subsubsection{CÁLCULO DE CARGA DE ILUMINAÇÃO}

A NBR-5410 estabelece critérios para a previsão de carga de iluminação em instalações elétricas residenciais. As diretrizes para a previsão de carga de iluminação são as seguintes:

\begin{itemize}
	\item Em cada cômodo ou dependência, deve ser previsto pelo menos um ponto de luz com potência mínima de $100$ $VA$;
	\item Em cômodos com área igual ou inferior a $6$ $m^2$, deve ser prevista pelo menos uma carga de $100$ $VA$;
	\item Em cômodos com área superior a $6$ $m^2$, a previsão de carga deve considerar $100$ $VA$ para os primeiros $6$ $m^2$, acrescidos de $60$ $VA$ para cada aumento de $4$ $m^2$ inteiros.
	
\end{itemize}

É importante destacar que esses valores são mínimos e que a previsão de carga de iluminação deve considerar a utilização prevista para cada ambiente, a fim de garantir que a capacidade do circuito seja suficiente para atender à demanda de carga elétrica requerida pelos pontos de iluminação, sem sobrecarregar o sistema elétrico e garantindo a segurança do usuário.

\subsection{CARGA INSTALADA}

A TABELA~\ref{tab:previsao_de_carga} mostra o levantamento dos pontos de iluminação, tomadas de uso geral e tomadas de uso específico, bem como suas cargas instaladas, de acordo com os critérios descritos na Seção~\ref{criterios_carga_nbr}:

\begin{table}[H]
	\centering
	\caption{CÁLCULO DE PONTOS E DE PREVISÃO DE CARGA}
	\label{tab:previsao_de_carga}
	\resizebox{\columnwidth}{!}{%
	\begin{tabular}{|l|l|ll|llll|lll|ll|}
		\hline
		\multirow{2}{*}{PAVIMENTO}          & \multirow{2}{*}{DEPENDÊNCIA} & \multicolumn{2}{l|}{DIMENSÕES}                                       & \multicolumn{4}{l|}{ILUMINAÇÃO}                                                                                                                        & \multicolumn{3}{l|}{TUG}                                                                                    & \multicolumn{2}{l|}{TUE}                             \\ \cline{3-13} 
		&                              & \multicolumn{1}{l|}{AREA ($m^2$)}              & PERÍMETRO (m)          & \multicolumn{1}{l|}{N. PONTOS}          & \multicolumn{1}{l|}{POT.UNIT}             & \multicolumn{1}{l|}{POT. UNIT. CALC}      & POT. TOTAL           & \multicolumn{1}{l|}{N. PONTOS}          & \multicolumn{1}{l|}{POT. UNIT}            & POT. TOTAL            & \multicolumn{1}{l|}{APARELHO}             & POTENCIA \\ \hline
		\multirow{17}{*}{TERREO}            & HALL ENT + EXTERNO           & \multicolumn{1}{l|}{1,001}                  & 4,1399                 & \multicolumn{1}{l|}{1}                  & \multicolumn{1}{l|}{100}                  & \multicolumn{1}{l|}{40}                   & 400                  & \multicolumn{1}{l|}{1}                  & \multicolumn{1}{l|}{100}                  & 100                   & \multicolumn{1}{l|}{}                     &          \\ \cline{2-13} 
		& SALA DE ESTAR                & \multicolumn{1}{l|}{25,39}                  & 20,71                  & \multicolumn{1}{l|}{4}                  & \multicolumn{1}{l|}{100}                  & \multicolumn{1}{l|}{340}                  & 400                  & \multicolumn{1}{l|}{5}                  & \multicolumn{1}{l|}{100}                  & 500                   & \multicolumn{1}{l|}{AR CONDICIONADO}      & 3600     \\ \cline{2-13} 
		& QUARTO 2                     & \multicolumn{1}{l|}{19,08}                  & 20                     & \multicolumn{1}{l|}{3}                  & \multicolumn{1}{l|}{100}                  & \multicolumn{1}{l|}{280}                  & 300                  & \multicolumn{1}{l|}{5}                  & \multicolumn{1}{l|}{100}                  & 500                   & \multicolumn{1}{l|}{AR CONDICIONADO}      & 3600     \\ \cline{2-13} 
		& \multirow{2}{*}{GARAGEM}     & \multicolumn{1}{l|}{\multirow{2}{*}{36,93}} & \multirow{2}{*}{29,6}  & \multicolumn{1}{l|}{\multirow{2}{*}{6}} & \multicolumn{1}{l|}{\multirow{2}{*}{100}} & \multicolumn{1}{l|}{\multirow{2}{*}{520}} & \multirow{2}{*}{600} & \multicolumn{1}{l|}{\multirow{2}{*}{6}} & \multicolumn{1}{l|}{\multirow{2}{*}{100}} & \multirow{2}{*}{600}  & \multicolumn{1}{l|}{MOTOR PORTÃO}         & 650      \\ \cline{12-13} 
		&                              & \multicolumn{1}{l|}{}                       &                        & \multicolumn{1}{l|}{}                   & \multicolumn{1}{l|}{}                     & \multicolumn{1}{l|}{}                     &                      & \multicolumn{1}{l|}{}                   & \multicolumn{1}{l|}{}                     &                       & \multicolumn{1}{l|}{TOMADA POTENCIA 220V} & 1000     \\ \cline{2-13} 
		& SALA DE JANTAR               & \multicolumn{1}{l|}{26,34}                  & 21,43                  & \multicolumn{1}{l|}{4}                  & \multicolumn{1}{l|}{100}                  & \multicolumn{1}{l|}{400}                  & 400                  & \multicolumn{1}{l|}{5}                  & \multicolumn{1}{l|}{100}                  & 500                   & \multicolumn{1}{l|}{}                     &          \\ \cline{2-13} 
		& \multirow{5}{*}{COZINHA}     & \multicolumn{1}{l|}{\multirow{5}{*}{9,49}}  & \multirow{5}{*}{12,65} & \multicolumn{1}{l|}{\multirow{5}{*}{1}} & \multicolumn{1}{l|}{\multirow{5}{*}{100}} & \multicolumn{1}{l|}{\multirow{5}{*}{100}} & \multirow{5}{*}{100} & \multicolumn{1}{l|}{\multirow{2}{*}{1}} & \multicolumn{1}{l|}{\multirow{2}{*}{}}    & \multirow{2}{*}{600}  & \multicolumn{1}{l|}{GRILL}                & 1200     \\ \cline{12-13} 
		&                              & \multicolumn{1}{l|}{}                       &                        & \multicolumn{1}{l|}{}                   & \multicolumn{1}{l|}{}                     & \multicolumn{1}{l|}{}                     &                      & \multicolumn{1}{l|}{}                   & \multicolumn{1}{l|}{}                     &                       & \multicolumn{1}{l|}{LAVA LOUÇA}           & 1500     \\ \cline{9-13} 
		&                              & \multicolumn{1}{l|}{}                       &                        & \multicolumn{1}{l|}{}                   & \multicolumn{1}{l|}{}                     & \multicolumn{1}{l|}{}                     &                      & \multicolumn{1}{l|}{\multirow{3}{*}{3}} & \multicolumn{1}{l|}{\multirow{3}{*}{}}    & \multirow{3}{*}{1350} & \multicolumn{1}{l|}{TORNEIRA ELETRICA}    & 4500     \\ \cline{12-13} 
		&                              & \multicolumn{1}{l|}{}                       &                        & \multicolumn{1}{l|}{}                   & \multicolumn{1}{l|}{}                     & \multicolumn{1}{l|}{}                     &                      & \multicolumn{1}{l|}{}                   & \multicolumn{1}{l|}{}                     &                       & \multicolumn{1}{l|}{FOGÃO ELÉTRICO}       & 6500     \\ \cline{12-13} 
		&                              & \multicolumn{1}{l|}{}                       &                        & \multicolumn{1}{l|}{}                   & \multicolumn{1}{l|}{}                     & \multicolumn{1}{l|}{}                     &                      & \multicolumn{1}{l|}{}                   & \multicolumn{1}{l|}{}                     &                       & \multicolumn{1}{l|}{FORNO ELÉTRICO}       & 6500     \\ \cline{2-13} 
		& BWC SOCIAL                   & \multicolumn{1}{l|}{2,23}                   & 6                      & \multicolumn{1}{l|}{1}                  & \multicolumn{1}{l|}{100}                  & \multicolumn{1}{l|}{100}                  & 100                  & \multicolumn{1}{l|}{1}                  & \multicolumn{1}{l|}{100}                  & 600                   & \multicolumn{1}{l|}{}                     &          \\ \cline{2-13} 
		& BWC Q2                       & \multicolumn{1}{l|}{4,12}                   & 8,8                    & \multicolumn{1}{l|}{1}                  & \multicolumn{1}{l|}{100}                  & \multicolumn{1}{l|}{100}                  & 100                  & \multicolumn{1}{l|}{1}                  & \multicolumn{1}{l|}{100}                  & 600                   & \multicolumn{1}{l|}{CHUVEIRO}             & 6500     \\ \cline{2-13} 
		& BWC Q1                       & \multicolumn{1}{l|}{4,02}                   & 8,1                    & \multicolumn{1}{l|}{1}                  & \multicolumn{1}{l|}{100}                  & \multicolumn{1}{l|}{100}                  & 100                  & \multicolumn{1}{l|}{1}                  & \multicolumn{1}{l|}{100}                  & 600                   & \multicolumn{1}{l|}{CHUVEIRO}             & 6500     \\ \cline{2-13} 
		& QUARTO 1                     & \multicolumn{1}{l|}{18,91}                  & 19,79                  & \multicolumn{1}{l|}{3}                  & \multicolumn{1}{l|}{100}                  & \multicolumn{1}{l|}{280}                  & 300                  & \multicolumn{1}{l|}{4}                  & \multicolumn{1}{l|}{100}                  & 400                   & \multicolumn{1}{l|}{AR CONDICIONADO}      & 3600     \\ \cline{2-13} 
		& JARDIM EXTERNO               & \multicolumn{1}{l|}{8,09}                   & 11,5                   & \multicolumn{1}{l|}{1}                  & \multicolumn{1}{l|}{100}                  & \multicolumn{1}{l|}{100}                  & 100                  & \multicolumn{1}{l|}{3}                  & \multicolumn{1}{l|}{100}                  & 300                   & \multicolumn{1}{l|}{}                     &          \\ \cline{2-13} 
		& JARDIM INTERNO               & \multicolumn{1}{l|}{12,64}                  & 17,49                  & \multicolumn{1}{l|}{2}                  & \multicolumn{1}{l|}{100}                  & \multicolumn{1}{l|}{160}                  & 200                  & \multicolumn{1}{l|}{4}                  & \multicolumn{1}{l|}{100}                  & 400                   & \multicolumn{1}{l|}{}                     &          \\ \hline
		\multirow{13}{*}{SEGUNDO PAVIMENTO} & SUITE                        & \multicolumn{1}{l|}{25,59}                  & 23,41                  & \multicolumn{1}{l|}{4}                  & \multicolumn{1}{l|}{100}                  & \multicolumn{1}{l|}{340}                  & 400                  & \multicolumn{1}{l|}{5}                  & \multicolumn{1}{l|}{100}                  & 500                   & \multicolumn{1}{l|}{AR CONDICIONADO}      & 3600     \\ \cline{2-13} 
		& BWC SUITE                    & \multicolumn{1}{l|}{3,12}                   & 7,4                    & \multicolumn{1}{l|}{1}                  & \multicolumn{1}{l|}{100}                  & \multicolumn{1}{l|}{100}                  & 100                  & \multicolumn{1}{l|}{1}                  & \multicolumn{1}{l|}{100}                  & 600                   & \multicolumn{1}{l|}{CHUVEIRO}             & 6500     \\ \cline{2-13} 
		& \multirow{2}{*}{LAVANDERIA}  & \multicolumn{1}{l|}{\multirow{2}{*}{9,8}}   & \multirow{2}{*}{13}    & \multicolumn{1}{l|}{\multirow{2}{*}{1}} & \multicolumn{1}{l|}{\multirow{2}{*}{100}} & \multicolumn{1}{l|}{\multirow{2}{*}{100}} & \multirow{2}{*}{100} & \multicolumn{1}{l|}{\multirow{2}{*}{4}} & \multicolumn{1}{l|}{}                     & 700                   & \multicolumn{1}{l|}{MÁQUINA LAVA E SECA}  & 3500     \\ \cline{10-13} 
		&                              & \multicolumn{1}{l|}{}                       &                        & \multicolumn{1}{l|}{}                   & \multicolumn{1}{l|}{}                     & \multicolumn{1}{l|}{}                     &                      & \multicolumn{1}{l|}{}                   & \multicolumn{1}{l|}{}                     & 1200                  & \multicolumn{1}{l|}{}                     &          \\ \cline{2-13} 
		& BWC SOCIAL                   & \multicolumn{1}{l|}{2,04}                   & 6,06                   & \multicolumn{1}{l|}{1}                  & \multicolumn{1}{l|}{100}                  & \multicolumn{1}{l|}{100}                  & 100                  & \multicolumn{1}{l|}{1}                  & \multicolumn{1}{l|}{100}                  & 600                   & \multicolumn{1}{l|}{}                     &          \\ \cline{2-13} 
		& CLOSET Q3                    & \multicolumn{1}{l|}{4,7}                    & 8,63                   & \multicolumn{1}{l|}{1}                  & \multicolumn{1}{l|}{100}                  & \multicolumn{1}{l|}{100}                  & 100                  & \multicolumn{1}{l|}{2}                  & \multicolumn{1}{l|}{100}                  & 200                   & \multicolumn{1}{l|}{}                     &          \\ \cline{2-13} 
		& QUARTO 3                     & \multicolumn{1}{l|}{19,48}                  & 20,8                   & \multicolumn{1}{l|}{3}                  & \multicolumn{1}{l|}{100}                  & \multicolumn{1}{l|}{280}                  & 300                  & \multicolumn{1}{l|}{5}                  & \multicolumn{1}{l|}{100}                  & 500                   & \multicolumn{1}{l|}{AR CONDICIONADO}      & 3600     \\ \cline{2-13} 
		& BWC Q3                       & \multicolumn{1}{l|}{3,97}                   & 8,7                    & \multicolumn{1}{l|}{1}                  & \multicolumn{1}{l|}{100}                  & \multicolumn{1}{l|}{100}                  & 100                  & \multicolumn{1}{l|}{1}                  & \multicolumn{1}{l|}{100}                  & 600                   & \multicolumn{1}{l|}{CHUVEIRO}             & 6500     \\ \cline{2-13} 
		& TERRAÇO                      & \multicolumn{1}{l|}{16,18}                  & 16,28                  & \multicolumn{1}{l|}{2}                  & \multicolumn{1}{l|}{100}                  & \multicolumn{1}{l|}{220}                  & 200                  & \multicolumn{1}{l|}{2}                  & \multicolumn{1}{l|}{100}                  & 200                   & \multicolumn{1}{l|}{}                     &          \\ \cline{2-13} 
		& SALA INTIMA                  & \multicolumn{1}{l|}{19,58}                  & 18,02                  & \multicolumn{1}{l|}{3}                  & \multicolumn{1}{l|}{100}                  & \multicolumn{1}{l|}{280}                  & 300                  & \multicolumn{1}{l|}{4}                  & \multicolumn{1}{l|}{100}                  & 400                   & \multicolumn{1}{l|}{AR CONDICIONADO}      & 3600     \\ \cline{2-13} 
		& CORREDOR SUITE               & \multicolumn{1}{l|}{1,71}                   & 5,5                    & \multicolumn{1}{l|}{1}                  & \multicolumn{1}{l|}{100}                  & \multicolumn{1}{l|}{40}                   & 100                  & \multicolumn{1}{l|}{1}                  & \multicolumn{1}{l|}{100}                  & 100                   & \multicolumn{1}{l|}{}                     &          \\ \cline{2-13} 
		& HALL                         & \multicolumn{1}{l|}{1,85}                   & 5,65                   & \multicolumn{1}{l|}{1}                  & \multicolumn{1}{l|}{100}                  & \multicolumn{1}{l|}{40}                   & 100                  & \multicolumn{1}{l|}{1}                  & \multicolumn{1}{l|}{100}                  & 100                   & \multicolumn{1}{l|}{}                     &          \\ \cline{2-13} 
		& ESCADA                       & \multicolumn{1}{l|}{4,9}                    & 8,9                    & \multicolumn{1}{l|}{2}                  & \multicolumn{1}{l|}{100}                  & \multicolumn{1}{l|}{100}                  & 200                  & \multicolumn{3}{l|}{NORMA NÃO PREVÊ}                                           & \multicolumn{1}{l|}{}                     &          \\ \hline
		& TOTAL                        & \multicolumn{1}{l|}{}                       &                        & \multicolumn{1}{l|}{48}                 & \multicolumn{1}{l|}{}                     & \multicolumn{1}{l|}{}                     & 4800                 & \multicolumn{1}{l|}{66}                 & \multicolumn{1}{l|}{}                     & 12650                 & \multicolumn{1}{l|}{}                     & 72950    \\ \hline
	\end{tabular}}
\end{table}

A previsão de carga é de $92.400$ $VA$, segundo a soma total das cargas instaladas.

\section{DETERMINAÇÃO DA PROVÁVEL DEMANDA}

\subsection{MÉTODOS DE CÁLCULO}

\subsubsection{MÉTODO COBEI}
O Método COBEI é uma forma de calcular a demanda de potência elétrica para um determinado estabelecimento ou edifício. A sigla COBEI significa "Comitê Brasileiro de Eletricidade, Eletrônica, Iluminação e Telecomunicações", e esse método é baseado em normas técnicas brasileiras.

A demanda de potência elétrica é a quantidade de energia que é requerida pelo estabelecimento em um determinado momento. Essa demanda é influenciada pelo número de aparelhos eletrodomésticos e eletrônicos em uso, pelo tipo de atividade realizada no local, entre outros fatores.

O Método COBEI leva em consideração a potência nominal dos equipamentos elétricos que serão utilizados no local, além de fatores como o fator de simultaneidade (ou seja, a probabilidade de que todos os equipamentos elétricos estejam em uso ao mesmo tempo), o fator de demanda (que é a fração da carga máxima esperada) e a previsão de crescimento da demanda futura.

Com base nesses dados, o Método COBEI permite calcular a demanda de potência elétrica para o estabelecimento, o que é importante para a escolha adequada do tamanho do transformador e do quadro de distribuição de energia elétrica.

\subsubsection{MÉTODO CEMIG}

O método CEMIG é um método utilizado para o cálculo da demanda de potência elétrica em edificações. Ele foi desenvolvido pela Companhia Energética de Minas Gerais (CEMIG) e leva em consideração diversos fatores, como o número e a potência dos equipamentos instalados, o tipo de atividade desenvolvida na edificação, as características da iluminação e o tipo de climatização utilizado, entre outros. A partir dessas informações, é possível estimar a demanda de potência elétrica máxima que será requerida pela edificação em determinado momento, o que é importante para a adequação do sistema elétrico e a definição da tarifa de energia elétrica a ser aplicada.

\subsection{PROVÁVEL DEMANDA}

Para cálculo da demanda, utiliza-se dois métodos, um para iluminação e tomadas de uso geral e um diferente para cálculo da demanda das tomadas de uso específico. O método COBEI é bastante prático para cálculos de demanda de tomadas de uso geral e iluminação, pois considera somente a carga instalada para a determinação da demanda. Porém, para as tomadas de uso especifico, o método COBEI considera demanda de 100\%. Portanto, o método CEMIG é utilizado apenas para as tomadas de uso específico, pois considera .

Segundo o método COBEI, o fator de demanda de tomadas de uso geral e iluminação é obtido da tabela mostrada na \autoref{fig:demanda_cobei}:

\figura
{FATORES DE DEMANDA SEGUNDO A POTÊNCIA INSTALADA - CT-64/COBEI} % Legenda
{.8} % % da largura da área de texto
{imagens/demanda_cobei} % localização da figura
{CT-64/COBEI} % fonte da figura
{demanda_cobei} % etiqueta
{}
{}

Segundo o método CEMIG, o fator de demanda para tomadas de uso específico é obtida da tabela mostrada na \autoref{fig:demanda_cemig}:

\figura
{FATORES DE DEMANDA PARA TOMADAS DE USO ESPECIFICO - MÉTODO CEMIG} % Legenda
{.8} % % da largura da área de texto
{imagens/demanda_cemig} % localização da figura
{ND 5.2 CEMIG} % fonte da figura
{demanda_cemig} % etiqueta
{}
{}

A potência instalada para iluminação e tomadas de uso geral é de $17.450$ $VA$ ($P_{COB}$), portanto, segundo a \autoref{fig:demanda_cobei}, o fator de demanda ($F_{COB}$) é $0,24$;

A potência instalada para as tomads de uso geral é de $72.950$ $VA$ ($P_{CEMIG}$), sendo o número de tomadas igual a 18. Portanto, a \autoref{fig:demanda_cemig} indica um fator de demanda ($F_{CEMIG}$) de $0,41$.

A demanda total da edificação é a soma das demandas individuais:

\begin{equation}
	D = F_{COB} \times P_{COB} + F_{CEMIG} \times P_{CEMIG}
\end{equation}

Portanto, a demanda provável será igual a:

\begin{equation}
	\begin{split}
		D = 17.450 \times 0,24 + 72.950 \times 0,41\\
		D = 34.097,5 VA
	\end{split}
\end{equation}

Logo, a demanda provável da edificação é de $34.097,5$ $VA$

A demanda provável é importante para a escolha da entrada de serviço. Segundo a tabela 2 da norma NTC 901100, o padrão da entrada de serviço segue as especificações listadas abaixo:

\begin{itemize}
	\item \textbf{\textit{Categoria}}: 41
	\item \textbf{\textit{Demanda Máxima:}} $38$ $kVA$
	\item \textbf{\textit{Tipo de fornecimento:}} Trifásico a 4 fios.
	\item \textbf{\textit{Bitola do fio (Ramal de entrada):}} para as fases, $25$ $mm^2$ EPR/XLPE 90°C,  condutor de aterramento nu ou encapado de $16$ $mm^2$;
	\item \textbf{\textit{Poste: }}DAN 200
\end{itemize}

A entrada de serviço fica assim definida:

\begin{itemize}
	\item \textit{Tipo de entrada:} Categoria 41 COPEL, segundo a tabela 2 e a demanda requerida ($34.097,5$ $VA$);
	\item \textit{Ramal de Entrada:} tipo trifásica, categoria 41, com disjuntor trifásico de $100$ $A$ e Medidor Trifásico a 4 fios tipo T;
	\item \textit{Fios de fase/neutro:} cabos de cobre com seção nominal de $25$ $mm^2$ com isolamento EPR/XLPE 90°C;
	\item \textit{Fio de Proteção (PE):} cabo de cobre com seção nominal de $16$ $mm^2$ com isolamento EPR/XLPE 90°C;
	\item \textit{Modo de Instalação:} Maneira ``D'' (Subterrânea);
	\item \textit{Eletrodutos: } eletroduto corrugado com diâmetro de 40mm para o circuito 3F+N e um eletroduto corrugado com 20mm de diâmetro para o condutor de proteção.
\end{itemize}

\section{DIVISÃO DE CIRCUITOS}

A NBR-5410 define critérios para a divisão de circuitos em instalações elétricas de baixa tensão. Entre os critérios definidos, podemos citar:

\begin{itemize}
	\item A divisão dos circuitos deve ser feita de forma a evitar a sobrecarga dos condutores e dos dispositivos de proteção.
	
	\item A corrente de projeto dos circuitos deve ser escolhida com base na carga elétrica a ser alimentada.
	
	\item Os circuitos devem ser agrupados de acordo com a sua finalidade, podendo ser divididos em circuitos de iluminação, circuitos de tomadas, circuitos de equipamentos fixos, entre outros.
	
	\item Os circuitos de iluminação devem ser separados dos circuitos de tomadas, exceto em casos específicos, como em áreas de trabalho.
	
	\item Em áreas externas ou úmidas, os circuitos devem ser protegidos por dispositivos diferenciais-residuais (DR).
	
	\item Para circuitos de equipamentos fixos, devem ser consideradas as especificações do fabricante em relação à potência e corrente de partida.
	
	\item Em áreas com risco de explosão ou incêndio, devem ser utilizados dispositivos de proteção específicos.
	
\end{itemize}


A aplicação destes critérios gerou os circuitos de distribuição, dispostos na TABELA~\ref{tab:distribuicao_circuitos}, de acordo com os equipamentos elétricos instalados e determinação dos quadros de distribuição:

\begin{longtable}{|l|l|l|l|}
	\caption{DISTRIBUIÇÃO DE CIRCUITOS DE ACORDO COM O EQUIPAMENTO E QUADRO DE DISTRIBUIÇÃO}
	\label{tab:distribuicao_circuitos}\\
	\hline
	CIRCUITO & TIPO & QUADRO & TENSÃO \\ \hline
	\endfirsthead
	%
	\endhead
	%
	C12, C13 & AR CONDICIONADO & QDC1 & 220 \\ \hline
	C20, C21 & TORNEIRA ELETRICA & QDC1 & 220 \\ \hline
	C26, C27 & CHUVEIRO & QDC1 & 220 \\ \hline
	C28, C29 & CHUVEIRO & QDC1 & 220 \\ \hline
	C14, C15 & AR CONDICIONADO & QDC1 & 220 \\ \hline
	C16, C17 & AR CONDICIONADO & QDC1 & 220 \\ \hline
	C12, C13 & AR CONDICIONADO & QDC2 & 220 \\ \hline
	C18, C19 & CHUVEIRO & QDC2 & 220 \\ \hline
	C8, C9 & AR CONDICIONADO & QDC2 & 220 \\ \hline
	C16, C17 & CHUVEIRO & QDC2 & 220 \\ \hline
	C10, C11 & AR CONDICIONADO & QDC2 & 220 \\ \hline
	C18, C19 & \begin{tabular}[c]{@{}l@{}}MOTOR PORTÃO\\ TOMADA POTENCIA 220V\end{tabular} & QDC1 & 220 \\ \hline
	C22, C23 & FOGÃO ELÉTRICO & QDC1 & 220 \\ \hline
	C24, C25 & FORNO ELÉTRICO & QDC1 & 220 \\ \hline
	C14, C15 & MÁQUINA LAVA E SECA & QDC2 & 220 \\ \hline
	C4 & \begin{tabular}[c]{@{}l@{}}TUG \\ HALL ENTRADA\\ SALA DE ESTAR\\ BWC SOCIAL\end{tabular} & QDC1 & 127 \\ \hline
	C5 & \begin{tabular}[c]{@{}l@{}}TUG \\ QUARTO 2\\ BWC Q2\end{tabular} & QDC1 & 127 \\ \hline
	C6 & \begin{tabular}[c]{@{}l@{}}TUG\\ GARAGEM\\ JARDIM EXTERNO\end{tabular} & QDC1 & 127 \\ \hline
	C7 & \begin{tabular}[c]{@{}l@{}}TUG\\ SALA DE JANTAR\\ COZINHA\end{tabular} & QDC1 & 127 \\ \hline
	C8 & \begin{tabular}[c]{@{}l@{}}TUG\\ COZINHA GRILL\end{tabular} & QDC1 & 127 \\ \hline
	C9 & \begin{tabular}[c]{@{}l@{}}TUG\\ COZINHA LAVA LOUÇA\end{tabular} & QDC1 & 127 \\ \hline
	C10 & \begin{tabular}[c]{@{}l@{}}TUG\\ COZINHA\end{tabular} & QDC1 & 127 \\ \hline
	C11 & \begin{tabular}[c]{@{}l@{}}TUG\\ BWC Q1\\ QUARTO 1\\ JARDIM INTERNO\end{tabular} & QDC1 & 127 \\ \hline
	C3 & \begin{tabular}[c]{@{}l@{}}TUG\\ CLOSET Q3\\ QUARTO 3\\ BWC Q3\end{tabular} & QDC2 & 127 \\ \hline
	C4 & \begin{tabular}[c]{@{}l@{}}TUG\\ TERRAÇO\\ SALA INTIMA\\ BWC SOCIAL\end{tabular} & QDC2 & 127 \\ \hline
	C5 & \begin{tabular}[c]{@{}l@{}}TUG\\ LAVANDERIA\\ CORREDOR SUITE\\ HALL\end{tabular} & QDC2 & 127 \\ \hline
	C6 & \begin{tabular}[c]{@{}l@{}}TUG\\ LAVANDERIA\end{tabular} & QDC2 & 127 \\ \hline
	C7 & \begin{tabular}[c]{@{}l@{}}TUG\\ SUITE\\ BWC SUITE\end{tabular} & QDC2 & 127 \\ \hline
	C1 & \begin{tabular}[c]{@{}l@{}}ILUMINAÇÃO\\ HALL ENTRADA\\ SALA DE ESTAR\\ QUARTO 2\end{tabular} & QDC1 & 127 \\ \hline
	C2 & \begin{tabular}[c]{@{}l@{}}ILUMINAÇÃO\\ GARAGEM\\ SALA DE JANTAR\end{tabular} & QDC1 & 127 \\ \hline
	C3 & \begin{tabular}[c]{@{}l@{}}ILUMINAÇÃO\\ COZINHA\\ BWC SOCIAL\\ BWC Q2\\ BWC Q1\\ QUARTO 1\\ JARDIM EXTERNO\\ JARDIM INTERNO\end{tabular} & QDC1 & 127 \\ \hline
	C1 & \begin{tabular}[c]{@{}l@{}}ILUMINAÇÃO\\ CLOSET Q3\\ QUARTO 3\\ BWC Q3\\ TERRAÇO\\ SALA INTIMA\\ ESCADA\end{tabular} & QDC2 & 127 \\ \hline
	C2 & \begin{tabular}[c]{@{}l@{}}ILUMINAÇÃO\\ SUITE\\ LAVANDERIA\\ BWC SUITE\\ BWC SOCIAL\\ CORREDOR SUITE\\ HALL\end{tabular} & QDC2 & 127 \\ \hline
\end{longtable}


\section{BALANCEAMENTO DE CARGA}

O balanceamento de carga é feita utilizando o seguinte critério:

\begin{itemize}
	
	\item Distribuir os circuitos nos quadros de distribuição, de forma equilibrada, para evitar sobrecarga em uma ou mais fases da rede trifásica.
	
\end{itemize}

A aplicação deste critério gerou a distribuição de carga disposta na TABELA~\ref{tab:balanceamento_de_carga}:

\begin{longtable}{|l|l|l|l|l|l|}
	\caption{BALANCEAMENTO DE CARGA DE ACORDO COM O EQUIPAMENTO E POTÊNCIA INSTALADA}
	\label{tab:balanceamento_de_carga}\\
	\hline
	CIRCUITO & QUADRO & TENSÃO & FASE F1 & FASE F2 & FASE F3 \\ \hline
	\endfirsthead
	%
	\endhead
	%
	C12, C13 & QDC1   & 220    & 1800    & 1800    &         \\ \hline
	C20, C21 & QDC1   & 220    & 2250    & 2250    &         \\ \hline
	C26, C27 & QDC1   & 220    & 3250    &         & 3250    \\ \hline
	C28, C29 & QDC1   & 220    & 3250    & 3250    &         \\ \hline
	C14, C15 & QDC1   & 220    & 1800    &         & 1800    \\ \hline
	C16, C17 & QDC1   & 220    &         & 1800    & 1800    \\ \hline
	C12, C13 & QDC2   & 220    &         & 1800    & 1800    \\ \hline
	C18, C19 & QDC2   & 220    &         & 3250    & 3250    \\ \hline
	C8, C9   & QDC2   & 220    & 1800    & 1800    &         \\ \hline
	C16, C17 & QDC2   & 220    & 3250    &         & 3250    \\ \hline
	C10, C11 & QDC2   & 220    & 1800    &         & 1800    \\ \hline
	C18, C19 & QDC1   & 220    &         & 825     & 825     \\ \hline
	C22, C23 & QDC1   & 220    & 3250    &         & 3250    \\ \hline
	C24, C25 & QDC1   & 220    &         & 3250    & 3250    \\ \hline
	C14, C15 & QDC2   & 220    & 1750    & 1750    &         \\ \hline
	C4       & QDC1   & 127    &         &         & 1200    \\ \hline
	C5       & QDC1   & 127    &         &         & 1100    \\ \hline
	C6       & QDC1   & 127    & 900     &         &         \\ \hline
	C7       & QDC1   & 127    &         & 1100    &         \\ \hline
	C8       & QDC1   & 127    &         &         & 1200    \\ \hline
	C9       & QDC1   & 127    &         & 1500    &         \\ \hline
	C10      & QDC1   & 127    &         & 1350    &         \\ \hline
	C11      & QDC1   & 127    &         & 1400    &         \\ \hline
	C3       & QDC2   & 127    &         & 1300    &         \\ \hline
	C4       & QDC2   & 127    &         & 1200    &         \\ \hline
	C5       & QDC2   & 127    &         & 900     &         \\ \hline
	C6       & QDC2   & 127    &         &         & 1200    \\ \hline
	C7       & QDC2   & 127    & 1100    &         &         \\ \hline
	C1       & QDC1   & 127    & 1100    &         &         \\ \hline
	C2       & QDC1   & 127    & 1000    &         &         \\ \hline
	C3       & QDC1   & 127    &         &         & 1000    \\ \hline
	C1       & QDC2   & 127    & 1200    &         &         \\ \hline
	C2       & QDC2   & 127    & 900     &         &         \\ \hline
	&        &        &         &         &         \\ \hline
	TOTAL    &        &        & 30400   & 30525   & 29975   \\ \hline
\end{longtable}

As fases ficam assim com cargas de:

\begin{itemize}
	\item Fase F1: $30.400$ $VA$
	\item Fase F2: $30.525$ $VA$
	\item Fase F3: $29.975$ $VA$
\end{itemize}

Não existe diferença significativa entre as cargas das fases, sendo portanto considerada equilibrada como carga trifásica.

\section{FATOR DE CORREÇÃO POR AGRUPAMENTO DE CIRCUITOS}

A escolha de agrupar circuitos elétricos em uma instalação pode ter vários motivos, incluindo a necessidade de facilitar a identificação e manutenção dos circuitos, a redução do risco de sobrecarga elétrica e a adequação às normas de segurança elétrica aplicáveis. Também pode ser uma forma de otimizar o uso da energia elétrica e garantir um melhor funcionamento dos equipamentos elétricos conectados.

Determinar se haverá agrupamento de circuitos é importante para determinar fatores de cálculo, como o fator de agrupamento, um dos fatores de cálculo da corrente de projeto, que de fato determina o dimensionamento do fio dos circuitos em questão. Os fatores de agrupamento, por quantidade de condutores carregados e quantidade de circuitos são determinados pela tabela 42 da NBR-5410, que é mostrada na \autoref{fig:fator_de_agrupamento_nbr}:

\figura
{FATORES DE CORREÇÃO POR AGRUPAMENTO DE CIRCUITOS - NBR-5410} % Legenda
{.8} % % da largura da área de texto
{imagens/fator_de_agrupamento_nbr} % localização da figura
{NBR-5410:2004} % fonte da figura
{fator_de_agrupamento_nbr} % etiqueta
{}
{}
\pagebreak
O agrupamento de circuitos da edificação, junto com o fator de agrupamento ($FCA$) fica definido como mostrado na TABELA~\ref{tab:agrupamento_circuitos}

\begin{longtable}{|lllll|}
	\caption{AGRUPAMENTO MÁXIMO DE CIRCUITOS}
	\label{tab:agrupamento_circuitos}\\
	\hline
	\endfirsthead
	%
	\endhead
	%
	\multicolumn{3}{|l|}{CIRCUITOS}                                                                & \multicolumn{1}{l|}{TOTAL DE CIRCUITOS} & FCA \\ \hline
	\multicolumn{1}{|l|}{C1}       & \multicolumn{1}{l|}{C4}       & \multicolumn{1}{l|}{C12, C13} & \multicolumn{1}{l|}{3}                  & 0,7 \\ \hline
	\multicolumn{1}{|l|}{C4}       & \multicolumn{1}{l|}{C5}       & \multicolumn{1}{l|}{C14, C15} & \multicolumn{1}{l|}{3}                  & 0,7 \\ \hline
	\multicolumn{1}{|l|}{C3}       & \multicolumn{1}{l|}{C26, C27} & \multicolumn{1}{l|}{}         & \multicolumn{1}{l|}{2}                  & 0,8 \\ \hline
	\multicolumn{1}{|l|}{C2}       & \multicolumn{1}{l|}{C6}       & \multicolumn{1}{l|}{C18, C19} & \multicolumn{1}{l|}{3}                  & 0,7 \\ \hline
	\multicolumn{1}{|l|}{C8}       & \multicolumn{1}{l|}{C9}       & \multicolumn{1}{l|}{C10}      & \multicolumn{1}{l|}{3}                  & 0,7 \\ \hline
	\multicolumn{1}{|l|}{C2}       & \multicolumn{1}{l|}{C7}       & \multicolumn{1}{l|}{C20, C21} & \multicolumn{1}{l|}{3}                  & 0,7 \\ \hline
	\multicolumn{1}{|l|}{C3}      & \multicolumn{1}{l|}{C11}      & \multicolumn{1}{l|}{C16, C17} & \multicolumn{1}{l|}{3} & 0,7 \\ \hline
	\multicolumn{1}{|l|}{C22,C23} & \multicolumn{1}{l|}{C24, C25} & \multicolumn{1}{l|}{}         & \multicolumn{1}{l|}{2} & 0,8 \\ \hline
	\multicolumn{1}{|l|}{C28, C29} & \multicolumn{1}{l|}{}         & \multicolumn{1}{l|}{}         & \multicolumn{1}{l|}{1}                  & 1   \\ \hline
	\multicolumn{1}{|l|}{C1}       & \multicolumn{1}{l|}{C3}       & \multicolumn{1}{l|}{C8, C9}   & \multicolumn{1}{l|}{3}                  & 0,7 \\ \hline
	\multicolumn{1}{|l|}{C4}       & \multicolumn{1}{l|}{C16, C17} & \multicolumn{1}{l|}{}         & \multicolumn{1}{l|}{2}                  & 0,8 \\ \hline
	\multicolumn{1}{|l|}{C6}       & \multicolumn{1}{l|}{C18, C19} & \multicolumn{1}{l|}{}         & \multicolumn{1}{l|}{2}                  & 0,8 \\ \hline
	\multicolumn{1}{|l|}{C1}       & \multicolumn{1}{l|}{C5}       & \multicolumn{1}{l|}{C14, C15} & \multicolumn{1}{l|}{3}                  & 0,7 \\ \hline
	\multicolumn{1}{|l|}{C2}       & \multicolumn{1}{l|}{C7}       & \multicolumn{1}{l|}{C12, C13} & \multicolumn{1}{l|}{3}                  & 0,7 \\ \hline
	\multicolumn{1}{|l|}{C4}       & \multicolumn{1}{l|}{C10, C11} & \multicolumn{1}{l|}{}         & \multicolumn{1}{l|}{2}                  & 0,8 \\ \hline
\end{longtable}

\section{FATOR DE CORREÇÃO DE TEMPERATURA}

A temperatura média anual de Curitiba é de cerca de 16°C. No verão, a temperatura média fica em torno de 21°C e no inverno a temperatura média fica em torno de 12°C. No entanto, é importante lembrar que esses valores são apenas médias e as temperaturas podem variar bastante ao longo do ano e até mesmo dentro do mesmo dia.

O fator de correção de temperatura, para a temperatura ambiente e de solo, é dado pela tabela 40 da NMBR-5410, mostrada na \autoref{fig:fator_temperatura_nbr}:

\figura
{FATORES DE CORREÇÃO DE TEMPERATURA - NBR-5410} % Legenda
{.8} % % da largura da área de texto
{imagens/fator_temperatura_nbr} % localização da figura
{NBR-5410:2004} % fonte da figura
{fator_temperatura_nbr} % etiqueta
{}
{}

A temperatura considerada no projeto é 30°C, para linhas não subterrâneas e embutidas em alvenaria e lajes, e de 20°C para linhas subterrâneas, para ser conservador. Portanto, O fator de correção de temperatura ($FCT$) é igual a 1.

\section{DIMENSIONAMENTO DE CONDUTORES}

\subsection{CRITÉRIOS DE CÁLCULO}

O dimensionamento dos condutores é feito seguindo o roteiro abaixo:

\begin{itemize}
	\item Determina-se as seções dos condutores conforme a Capacidade de Corrente;
	\item Determina-se as seções dos condutores pelo Limite de Queda de Tensão;
	\item Determina-se as seções dos condutores pela seção mínima;
\end{itemize}

\subsubsection{CAPACIDADE DE CORRENTE}

A corrente corrigida é calculada da seguinte maneira:

\begin{equation}
	I_p = \frac{I_n}{FCT \times FCA} 
	\label{eqn:corrente_corrigida}
\end{equation}

Onde:

\begin{itemize}
	\item $I_p$: corrente corrigida;
	\item $I_n$: corrente de projeto;
	\begin{itemize}
		\item Para circuitos terminais será a corrente de projeto da maior derivação do disjuntor;
		\item Para circuitos de distribuição será a maior corrente de fase;
	\end{itemize}
	\item $FCT$: fator de correção de temperatura;
	\item $FCA$: fator de correção por agrupamento de circuitos;
\end{itemize}

A corrente de projeto é calculada da seguinte forma:

\begin{equation}
	I_n = \frac{P_n}{\eta_f \times V \times cos(\phi) \times \eta}
	\label{eqn:corrente_projeto}
\end{equation}

Onde:

\begin{itemize}
	\item $I_n$: corrente de projeto;
	\item $P_n$:  Potência nominal, em $W$;
	\item $\eta_f$: número de fases do circuito;
	\begin{itemize}
		\item Para circuitos com esquema F+N, F+F o valor é 1;
		\item Para circuitos com esquema 3F+N o valor é 3;
		\item Para circuitos com esquema 3F o valor é $\sqrt{3}$;
	\end{itemize}
	\item $V$: Tensão nominal;
	\begin{itemize}
		\item Para circuitos com esquema F+N e 3F+N é utilizado o valor de tensão de fase;
		\item Para circuitos com esquema F+F e 3F é utilizado o valor de tensão de linha;
	\end{itemize}
	\item $cos(\phi)$: fator de potência do equipamento;
	\item $\eta$: rendimento do equipamento.
\end{itemize}

Após calculada a corrente projeto de cada equipamento instalado em um determinado circuito, a corrente de projeto do circuito será a soma de todas as correntes de projeto dos equipamentos ligados ao circuito em questão.

\subsubsection{QUEDA DE TENSÃO}

O método utilizado para cálculo da queda de tensão é o método da tensão unitária.

O método da tensão unitária é um dos métodos utilizados para calcular a queda de tensão em uma instalação elétrica. Esse método é baseado no cálculo da queda de tensão por unidade de comprimento do condutor, considerando a corrente elétrica que passa pelo mesmo.

O cálculo da queda de tensão pelo método da tensão unitária pode ser realizado a partir da seguinte equação:

\begin{equation}
	\Delta V = K \times I \times L
\end{equation}

Onde:

\begin{itemize}
	\item $\Delta V$ é a queda de tensão no condutor, em volts (V);
	\item $K$ é a constante de proporcionalidade que relaciona a queda de tensão com a corrente e o comprimento do condutor, em ohms por quilômetro ($\frac{\Omega}{km}$);
	\item $I$ é a corrente elétrica que passa pelo condutor, em amperes ($A$);
	\item $L$ é o comprimento do condutor, em quilômetros ($km$).	
\end{itemize}


Para calcular o valor da constante $K$, é necessário conhecer a resistividade do material do condutor e sua seção transversal. A NBR-5410 define valores padrão para a resistividade dos condutores de cobre e alumínio, que podem ser utilizados para esse cálculo.

É importante ressaltar que a queda de tensão deve ser limitada a valores aceitáveis para garantir o correto funcionamento dos equipamentos elétricos. A NBR-5410 estabelece limites de queda de tensão para diferentes tipos de circuitos, que devem ser respeitados durante o projeto e a instalação da rede elétrica.

a constante $K$ é calculada pela seguinte equação:

\begin{equation}
	K = \frac{e_{\%} \times V}{I_n \times L}
\end{equation}

Onde:

\begin{itemize}
	\item $e_{\%}$: queda de tensão em porcentagem;
	\item $V$: Tensão nominal;
	\item $I_n$: Corrente de projeto;
	\item $L$: comprimento em quilômetros.
\end{itemize}

\subsubsection{SEÇÃO MÍNIMA}

A NBR-5410 define valores mínimos de seção transversal para os condutores utilizados em instalações elétricas de baixa tensão. Esses valores são definidos com base na corrente elétrica e na finalidade do circuito.

De forma geral, para circuitos de iluminação, a seção mínima dos condutores deve ser de $1,5$ $mm^2$ para correntes até $10$ $A$, e $2,5$ $mm^2$ para correntes entre $10$ $A$ e $16$ $A$. Para circuitos de tomadas, a seção mínima deve ser de $2,5$ $mm^2$ para correntes até $16$ $A$, e $4$ $mm^2$ para correntes entre $16$ $A$ e $20$ $A$.

Para circuitos de equipamentos específicos, como chuveiros elétricos, ar-condicionado, motores elétricos, entre outros, a seção mínima dos condutores deve ser definida com base nas especificações do fabricante do equipamento e nos critérios definidos pela NBR 5410.

Vale lembrar que a seção transversal dos condutores deve ser capaz de suportar a corrente elétrica prevista no circuito, evitando sobreaquecimentos e riscos de incêndio.
\pagebreak
\subsection{CÁLCULO DOS CONDUTORES}

Aplicando os critérios de cálculo aos valores da TABELA~\ref{tab:previsao_de_carga} resulta nos valores da TABELA~\ref{tab:calculo_de_condutores}:

\begin{table}[H]
	\caption{DIMENSIONAMENTO DE CONDUTORES - TODOS OS CRITÉRIOS}
	\label{tab:calculo_de_condutores}
	\resizebox{\columnwidth}{!}{%
	\begin{tabular}{|c|c|c|c|c|c|c|c|c|c|c|c|}
		\hline
		CIRCUITO &
		CORRENTE DE PROJETO IP &
		FCT &
		FCA &
		CORRENTE DE CORRIGIDA IP' &
		\begin{tabular}[c]{@{}c@{}}COMPRIMENTO \\  L (m)\end{tabular} &
		QUEDA DE TENSÃO &
		I CONDUTOR &
		DIA FASE CORRENTE &
		DIA FASE QUEDA &
		DIA NEUTRO &
		DIA PROTEÇÃO \\ \hline
		C12, C13 & 16,364 & 1 & 0,70 & 23,38 & 12,23 & 43,97 & 24,00 & 2,5 &  & 2,5 & 2,5 \\ \hline
		C20, C21 & 20,455 & 1 & 0,70 & 29,22 & 14,86 & 28,95 & 32,00 & 4   &  & 4   & 4   \\ \hline
		C26, C27 & 29,545 & 1 & 0,80 & 36,93 & 5,88  & 50,65 & 41,00 & 6   &  & 6   & 6   \\ \hline
		C28, C29 & 29,545 & 1 & 1,00 & 29,55 & 9,36  & 31,82 & 32,00 & 4   &  & 4   & 4   \\ \hline
		C14, C15 & 16,364 & 1 & 0,70 & 23,38 & 10,85 & 49,56 & 24,00 & 2,5 &  & 2,5 & 2,5 \\ \hline
		C16, C17 & 16,364 & 1 & 0,70 & 23,38 & 13,79 & 39,00 & 24,00 & 2,5 &  & 2,5 & 2,5 \\ \hline
		C12, C13 & 16,364 & 1 & 0,70 & 23,38 & 13,35 & 40,28 & 24,00 & 2,5 &  & 2,5 & 2,5 \\ \hline
		C18, C19 & 29,545 & 1 & 0,80 & 36,93 & 11,36 & 26,22 & 41,00 & 6   &  & 6   & 6   \\ \hline
		C8, C9   & 16,364 & 1 & 0,70 & 23,38 & 6,86  & 78,39 & 24,00 & 2,5 &  & 2,5 & 2,5 \\ \hline
		C16, C17 & 29,545 & 1 & 0,80 & 36,93 & 9,22  & 32,30 & 41,00 & 6   &  & 6   & 6   \\ \hline
		C10, C11 & 16,364 & 1 & 0,80 & 20,45 & 9,04  & 59,49 & 24,00 & 2,5 &  & 2,5 & 2,5 \\ \hline
		C18, C19 & 7,500  & 1 & 0,70 & 10,71 & 19,92 & 58,90 & 24,00 & 2,5 &  & 2,5 & 2,5 \\ \hline
		C22, C23 & 29,545 & 1 & 0,80 & 36,93 & 14,70 & 20,26 & 41,00 & 6   &  & 6   & 6   \\ \hline
		C24, C25 & 29,545 & 1 & 0,80 & 36,93 & 14,84 & 20,07 & 41,00 & 6   &  & 6   & 6   \\ \hline
		C14, C15 & 15,909 & 1 & 0,70 & 22,73 & 16,83 & 32,87 & 24,00 & 2,5 &  & 2,5 & 2,5 \\ \hline
		C4       & 9,449  & 1 & 0,70 & 13,50 & 15,14 & 35,51 & 24,00 & 2,5 &  & 2,5 & 2,5 \\ \hline
		C5       & 8,661  & 1 & 0,70 & 12,37 & 14,05 & 41,74 & 24,00 & 2,5 &  & 2,5 & 2,5 \\ \hline
		C6       & 7,087  & 1 & 0,70 & 10,12 & 20,72 & 34,60 & 24,00 & 2,5 &  & 2,5 & 2,5 \\ \hline
		C7       & 8,661  & 1 & 0,70 & 12,37 & 12,06 & 48,63 & 24,00 & 2,5 &  & 2,5 & 2,5 \\ \hline
		C8       & 9,449  & 1 & 0,70 & 13,50 & 14,12 & 38,08 & 24,00 & 2,5 &  & 2,5 & 2,5 \\ \hline
		C9       & 11,811 & 1 & 0,70 & 16,87 & 15,70 & 27,40 & 24,00 & 2,5 &  & 2,5 & 2,5 \\ \hline
		C10      & 10,630 & 1 & 0,70 & 15,19 & 16,48 & 29,00 & 24,00 & 2,5 &  & 2,5 & 2,5 \\ \hline
		C11      & 11,024 & 1 & 0,70 & 15,75 & 17,06 & 27,01 & 24,00 & 2,5 &  & 2,5 & 2,5 \\ \hline
		C3       & 10,236 & 1 & 0,70 & 14,62 & 14,02 & 35,40 & 24,00 & 2,5 &  & 2,5 & 2,5 \\ \hline
		C4       & 9,449  & 1 & 0,80 & 11,81 & 10,86 & 49,51 & 24,00 & 2,5 &  & 2,5 & 2,5 \\ \hline
		C5       & 7,087  & 1 & 0,70 & 10,12 & 14,64 & 48,96 & 24,00 & 2,5 &  & 2,5 & 2,5 \\ \hline
		C6       & 9,449  & 1 & 0,80 & 11,81 & 17,03 & 31,57 & 24,00 & 2,5 &  & 2,5 & 2,5 \\ \hline
		C7       & 8,661  & 1 & 0,70 & 12,37 & 16,88 & 34,75 & 24,00 & 2,5 &  & 2,5 & 2,5 \\ \hline
		C1       & 8,661  & 1 & 0,70 & 12,37 & 16,64 & 35,25 & 17,50 & 1,5 &  & 1,5 & 1,5 \\ \hline
		C2       & 7,874  & 1 & 0,70 & 11,25 & 15,50 & 41,62 & 17,50 & 1,5 &  & 1,5 & 1,5 \\ \hline
		C3       & 7,874  & 1 & 0,70 & 11,25 & 17,87 & 36,10 & 17,50 & 1,5 &  & 1,5 & 1,5 \\ \hline
		C1       & 9,449  & 1 & 0,70 & 13,50 & 7,98  & 67,37 & 17,50 & 1,5 &  & 1,5 & 1,5 \\ \hline
		C2       & 7,087  & 1 & 0,70 & 10,12 & 12,63 & 56,76 & 17,50 & 1,5 &  & 1,5 & 1,5 \\ \hline
	\end{tabular}}
\end{table}

A tensão unitária não é suficiente para considerar o critério da queda de tensão, portanto a decisão fica entre o critério da capacidade de corrente e a seção mínima. Para os circuitos de iluminação e tomadas de uso geral, o critério da seção mínima mostrou-se mais do que suficiente.

As tomadas de uso específico foram calculadas pelo critério da capacidade de corrente. Para os circuitos com seção de $4$ $mm^2$, decidiu-se optar por cabos de $6$ $mm^2$ por padronização da instalação.

Os cabos de proteção são, segundo a norma NBR-5410, de seção igual a fase e neutro.

\section{DIMENSIONAMENTO DE ELETRODUTOS}

O dimensionamento de eletrodutos é uma etapa importante no projeto de uma instalação elétrica, pois os eletrodutos são responsáveis por proteger e conduzir os condutores elétricos de forma segura e eficiente. O dimensionamento adequado dos eletrodutos é importante para evitar o superaquecimento dos condutores e garantir a dissipação de calor, bem como para permitir a manutenção e a passagem de novos condutores no futuro.

O dimensionamento de eletrodutos deve levar em consideração a quantidade de condutores que serão instalados, sua seção transversal, o tipo de instalação (aparente ou embutida), o tipo de eletroduto (flexível ou rígido), a temperatura ambiente, entre outros fatores.

A NBR 5410 estabelece critérios para o dimensionamento de eletrodutos, levando em conta a quantidade e o diâmetro dos condutores, bem como a disposição dos eletrodutos no ambiente. Para isso, a norma apresenta tabelas e fórmulas que permitem calcular o diâmetro mínimo do eletroduto, a partir do número de condutores, sua seção transversal e o tipo de instalação.

Além dos critérios estabelecidos pela norma, o dimensionamento de eletrodutos deve levar em conta a facilidade de manutenção e inspeção dos condutores, bem como a necessidade de espaço para passagem de novos cabos no futuro. Também é importante considerar a possibilidade de expansão da instalação elétrica, prevendo o uso de eletrodutos com capacidade para suportar novos circuitos e equipamentos.

A taxa máxima de ocupação em relação à área da seção transversal dos
eletrodutos não deve ser superior a:


\begin{itemize}
	
	\item 53\% no caso de um condutor ou cabo;
	\item 31\% no caso de dois condutores ou cabos;
	\item 40\% no caso de três ou mais condutores ou cabos
	
\end{itemize}

A área útil de um eletroduto é definido como:

\begin{equation}
	A_e = \pi \times \frac{D_i^2}{4}
\end{equation}

E considerando que a soma das áreas externas dos condutores a serem instalados é dado por:

\begin{equation}
	\sum A_{cond}
\end{equation}

Então, o diâmetro interno do eletroduto é dado por:

\begin{equation}
	D_d = \sqrt{\frac{4 \times \sum A_{cond}}{f \times \pi}}
\end{equation}

Onde $f$ é a taxa de ocupação do eletroduto.

Segue-se um roteiro básico:

\begin{itemize}
	\item Determina-se a seção total ocupada pelos	condutores usando a tabela de fabricantes de condutores.
	\item Determina-se o diâmetro externo nominal do eletroduto
	(mm) consultando as	tabelas de eletrodutos
	\item Caso os condutores instalados em um mesmo	eletroduto sejam do mesmo tipo e tenham seções nominais iguais, pode-se eliminar os passos anteriores e encontrar o	diâmetro externo nominal	do eletroduto em função	da quantidade e seção dos condutores diretamente	por tabelas específicas
\end{itemize}

No caso, o terceiro passo do roteiro não é utilizado pois os circuitos agrupados são de seções nominais diferentes. Portanto, somente os 2 primeiros passos são utilizados.

Aplicados os 2 passos citados, chega-se à conclusão de que os eletrodutos indicados são de $\frac{3}{4}\inches$ e $1\inches$ de diâmetro interno. Por padronização, escolhe-se o eletroduto corrugado de $32$ $mm$ ou $1$ $\frac{1}{4}\inches$ de diâmetro externo.

\section{DIMENSIONAMENTO DOS DISPOSITIVOS DE PROTEÇÃO}

O dimensionamento dos dispositivos de proteção elétrica é fundamental para garantir a segurança da instalação elétrica e dos usuários, evitando riscos de curto-circuito, sobrecarga e outros tipos de falhas elétricas.

Os dispositivos de proteção elétrica mais comuns em uma instalação elétrica de baixa tensão incluem os disjuntores, os fusíveis, os interruptores diferenciais residuais (IDRs) e os dispositivos de proteção contra surtos (DPS).

O dimensionamento desses dispositivos deve ser realizado levando em consideração a corrente elétrica prevista no circuito, bem como as características dos equipamentos elétricos a serem protegidos. Os disjuntores, por exemplo, devem ser dimensionados para suportar a corrente elétrica máxima prevista no circuito, evitando o risco de sobrecarga. Os fusíveis também devem ser dimensionados para suportar a corrente máxima, além de serem escolhidos de acordo com a curva de atuação necessária para a proteção do circuito.

Já os interruptores diferenciais residuais (IDRs) são responsáveis por proteger as pessoas contra choques elétricos, detectando correntes de fuga no circuito e interrompendo o fornecimento de energia elétrica. O dimensionamento dos IDRs deve levar em conta a corrente elétrica máxima do circuito e o tempo de atuação necessário para garantir a proteção adequada.

Por fim, os dispositivos de proteção contra surtos (DPS) são responsáveis por proteger os equipamentos eletrônicos contra danos causados por surtos de tensão. O dimensionamento dos DPS deve levar em conta a tensão nominal da instalação elétrica, bem como a capacidade de absorção dos surtos de tensão.

\subsection{DISJUNTORES DE PROTEÇÃO (DP)}

Os disjuntores devem proteger contra sobrecarga e curto-circuito. O circuito mais longo da residência tem cerca de 18 metros de comprimento, portanto, desconsidera-se o calculo de curto-circuito para efeitos de dimensionamento, considerando-se somente a corrente máxima de sobrecarga.

Utilizando a corrente máxima de sobrecarga como critério de dimensionamento, chega-se aos resultados mostrados na TABELA~\ref{tab:dimensionamento_disjuntores} para cada circuito:


\begin{longtable}{|c|c|c|c|c|}
	\caption{DIMENSIONAMENTO DOS DISJUNTORES POR QUADRO E CIRCUITO}
	\label{tab:dimensionamento_disjuntores}\\
	\hline
	CIRCUITO                       & QUADRO & I CONDUTOR & DISJUNTOR CORRENTE & N POLOS \\ \hline
	\endfirsthead
	%
	\endhead
	%
	C12, C13                       & QDC1   & 24,00      & 20                 & 2       \\ \hline
	C20, C21                       & QDC1   & 32,00      & 30                 & 2       \\ \hline
	\multicolumn{1}{|l|}{C26, C27} & QDC1   & 41,00      & 40                 & 2       \\ \hline
	\multicolumn{1}{|l|}{C28, C29} & QDC1   & 32,00      & 30                 & 2       \\ \hline
	\multicolumn{1}{|l|}{C14, C15} & QDC1   & 24,00      & 20                 & 2       \\ \hline
	\multicolumn{1}{|l|}{C16, C17} & QDC1   & 24,00      & 20                 & 2       \\ \hline
	C12, C13                       & QDC2   & 24,00      & 20                 & 2       \\ \hline
	C18, C19                       & QDC2   & 41,00      & 40                 & 2       \\ \hline
	C8, C9                         & QDC2   & 24,00      & 20                 & 2       \\ \hline
	C16, C17                       & QDC2   & 41,00      & 40                 & 2       \\ \hline
	C10, C11                       & QDC2   & 24,00      & 20                 & 2       \\ \hline
	\multicolumn{1}{|l|}{C18, C19} & QDC1   & 24,00      & 20                 & 2       \\ \hline
	C22, C23                       & QDC1   & 41,00      & 40                 & 2       \\ \hline
	C24, C25                       & QDC1   & 41,00      & 40                 & 2       \\ \hline
	C14, C15                       & QDC2   & 24,00      & 20                 & 2       \\ \hline
	C4                             & QDC1   & 24,00      & 20                 & 1       \\ \hline
	C5                             & QDC1   & 24,00      & 20                 & 1       \\ \hline
	C6                             & QDC1   & 24,00      & 20                 & 1       \\ \hline
	C7                             & QDC1   & 24,00      & 20                 & 1       \\ \hline
	C8                             & QDC1   & 24,00      & 20                 & 1       \\ \hline
	C9                             & QDC1   & 24,00      & 20                 & 1       \\ \hline
	C10                            & QDC1   & 24,00      & 20                 & 1       \\ \hline
	C11                            & QDC1   & 24,00      & 20                 & 1       \\ \hline
	C3                             & QDC2   & 24,00      & 20                 & 1       \\ \hline
	C4                             & QDC2   & 24,00      & 20                 & 1       \\ \hline
	C5                             & QDC2   & 24,00      & 20                 & 1       \\ \hline
	C6                             & QDC2   & 24,00      & 20                 & 1       \\ \hline
	C7                             & QDC2   & 24,00      & 20                 & 1       \\ \hline
	C1                             & QDC1   & 17,50      & 15                 & 1       \\ \hline
	C2                             & QDC1   & 17,50      & 15                 & 1       \\ \hline
	C3                             & QDC1   & 17,50      & 15                 & 1       \\ \hline
	C1                             & QDC2   & 17,50      & 15                 & 1       \\ \hline
	C2                             & QDC2   & 17,50      & 15                 & 1       \\ \hline
\end{longtable}

Como disjuntor geral dos quadros de distribuição, escolhe-se:

\begin{itemize}
	\item QDC1: $80$ $A$
	\item QDC2: $70$ $A$
\end{itemize}

\subsection{DISPOSITIVOS DE PROTEÇÃO CONTRA SURTOS (DPS)}

O cálculo de DPS (Dispositivos de Proteção contra Surtos) é feito levando em consideração a tensão nominal da instalação elétrica e a capacidade de absorção de surtos de tensão dos equipamentos eletrônicos que serão protegidos. O objetivo é garantir que os equipamentos estejam adequadamente protegidos contra surtos de tensão que possam causar danos ou falhas.

O cálculo do DPS envolve a análise da curva de atuação do dispositivo, que determina a capacidade de absorção de surtos de tensão em função do nível de tensão aplicado. Essa curva de atuação é definida pelo fabricante do DPS e deve ser levada em conta no momento do cálculo.

O cálculo é feito a partir da seguinte fórmula:
\begin{equation}
	P = \frac{U_c \times I_{sc} \times K \times S}{U_p}
\end{equation}

Onde:

\begin{itemize}
	\item $P$: é a capacidade de corrente de impulso do DPS, em kA;
	\item $U_c$: é a tensão nominal da instalação elétrica, em volts;
	\item $I_{sc}$: é a corrente de curto-circuito máxima prevista na instalação elétrica, em kA;
	\item $K$: é o fator de correção, que leva em conta as características da instalação elétrica (fator de potência, comprimento dos cabos, etc.);
	\item $S$: é o fator de proteção, que leva em conta o tipo de equipamento a ser protegido (por exemplo, computadores, equipamentos de telecomunicações, etc.);
	\item $U_p$: é a tensão residual máxima permitida após o surto de tensão, em volts.
	
\end{itemize}

Após o cálculo, é importante selecionar um DPS com capacidade de corrente de impulso igual ou superior ao valor obtido na fórmula, levando em conta também as normas e regulamentos locais.

Para um fator de proteção $S=1$, $K=0,8$, $U_p=600V$, $I_{sc}=10000A$ (Assumida) e $U_c=127V$, obtém-se:

\begin{equation}
	\begin{split}
		P = \frac{127 \times 0,8 \times 1 \times 10000}{600} \\
		P = 1693,33 A
	\end{split}
\end{equation}

Portanto, DPS com correntes de pulso maiores que $1,7$ $kA$ servem ao propósito.

É bastante comum no mercado DPS com correntes de pulso de $5kA$ e tensão nominal de $275$ $V$, sendo este o valor escolhido como calculado para os DPS da instalação elétrica.

Para o QDC1, utiliza-se 4 DPS de $5kA$ e tensão nominal de $275$ $V$, sendo o mesmo válido para o QDC2.

\subsection{DISPOSITIVOS DIFERENCIAIS RESIDUAIS (DR)}

O cálculo do DR (Dispositivo Diferencial Residual) é feito para garantir a proteção das pessoas contra choques elétricos, desligando o circuito elétrico em caso de fuga de corrente elétrica.

O IDR é calculado a partir da corrente nominal do circuito elétrico que ele vai proteger, que é geralmente igual ou inferior à corrente nominal do dispositivo de proteção contra sobrecorrente (disjuntor ou fusível) correspondente.

A norma NBR-5410 exige que circuitos em áreas úmidas e tomadas de uso específico sejam protegidos por DRs. Para a proteção da vida humana, é necessário também que a corrente de sensibilidade seja igual ou inferior a $30$ $mA$.


Dos circuitos de tomadas de uso específico, tem-se a previsão de circuitos com DR mostrada na TABELA~\ref{tab:dr}:

% Please add the following required packages to your document preamble:
% \usepackage{longtable}
% Note: It may be necessary to compile the document several times to get a multi-page table to line up properly
\begin{longtable}{|c|c|c|c|c|}
	\caption{DIMENSIONAMENTO DOS DISPOSITIVOS DIFERENCIAIS RESIDUAIS}
	\label{tab:dr}\\
	\hline
	CIRCUITO                       & QUADRO & I CONDUTOR & DR (I=30mA) & N POLOS \\ \hline
	\endfirsthead
	%
	\endhead
	%
	C12, C13                       & QDC1   & 24,00      & 20          & 2       \\ \hline
	C20, C21                       & QDC1   & 32,00      & 30          & 2       \\ \hline
	\multicolumn{1}{|l|}{C26, C27} & QDC1   & 41,00      & 40          & 2       \\ \hline
	\multicolumn{1}{|l|}{C28, C29} & QDC1   & 32,00      & 30          & 2       \\ \hline
	\multicolumn{1}{|l|}{C14, C15} & QDC1   & 24,00      & 20          & 2       \\ \hline
	\multicolumn{1}{|l|}{C16, C17} & QDC1   & 24,00      & 20          & 2       \\ \hline
	C12, C13                       & QDC2   & 24,00      & 20          & 2       \\ \hline
	C18, C19                       & QDC2   & 41,00      & 40          & 2       \\ \hline
	C8, C9                         & QDC2   & 24,00      & 20          & 2       \\ \hline
	C16, C17                       & QDC2   & 41,00      & 40          & 2       \\ \hline
	C10, C11                       & QDC2   & 24,00      & 20          & 2       \\ \hline
	\multicolumn{1}{|l|}{C18, C19} & QDC1   & 24,00      & 20          & 2       \\ \hline
	C22, C23                       & QDC1   & 41,00      & 40          & 2       \\ \hline
	C24, C25                       & QDC1   & 41,00      & 40          & 2       \\ \hline
	C14, C15                       & QDC2   & 24,00      & 20          & 2       \\ \hline
\end{longtable}

Em virtude de haver muitos circuitos com DR, optou-se por 2 IDRs, um para cada quadro:

\begin{itemize}
	\item \textit{QDC1:} IDR $I_n = 80 A$ ($\Delta I_n = 30 mA$)
	\item \textit{QDC2:} IDR $I_n = 70 A$ ($\Delta I_n = 30 mA$)
\end{itemize}

Portanto, todos os circuitos serão protegidos contra fuga de corrente nos quadros \textit{QDC1} e \textit{QDC2}.
