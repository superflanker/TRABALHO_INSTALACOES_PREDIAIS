\chapter[MEMORIAL DESCRITIVO]{MEMORIAL DESCRITIVO}

\section{APRESENTAÇÃO}

Este memorial descritivo é um documento técnico que descreve detalhadamente os aspectos da instalação elétrica da residência TE344, localizada na Rua dos Alfeneiros, número 04, bairro Boa Vista na cidade de Curitiba - Paraná - Brasil. O objetivo principal é fornecer informações precisas e completas sobre a instalação elétrica, incluindo os materiais utilizados, as normas técnicas aplicáveis e o dimensionamento dos componentes elétricos. A partir deste documento é possível ter uma visão geral do funcionamento e características da instalação elétrica, além de garantir sua segurança e eficiência. 

Essa documentação é essencial para garantir a segurança das pessoas que frequentam o local, bem como para assegurar o bom funcionamento dos equipamentos elétricos. 

\section{NORMAS APLICÁVEIS}

\subsection{NORMA NBR-5410}

A NBR-5410 é uma norma técnica brasileira que estabelece as condições necessárias para a instalação elétrica de baixa tensão em edificações. Ela define os princípios e requisitos mínimos de segurança para o projeto, construção e manutenção de instalações elétricas, visando garantir a proteção das pessoas e a integridade das edificações.

Entre os aspectos abordados pela norma estão a seleção dos materiais e equipamentos, a instalação de condutores e dispositivos de proteção, a execução de aterramentos, a proteção contra sobrecargas e curto-circuitos, entre outros.

A NBR-5410 é de suma importância para a segurança elétrica das edificações no Brasil e é uma referência obrigatória para os profissionais do setor, como engenheiros elétricos, eletricistas e projetistas. É importante ressaltar que o não cumprimento das normas pode acarretar riscos de acidentes elétricos, além de problemas na regularização de projetos e obras.

Portanto, a NBR-5410 é uma norma técnica fundamental para garantir a segurança nas instalações elétricas de baixa tensão em edificações, e sua observância é fundamental para evitar riscos à vida e ao patrimônio.

\subsection{NORMA NBR-5413}

A NBR-5413 é uma norma técnica brasileira que estabelece os critérios e parâmetros para o projeto de iluminação artificial em ambientes internos e externos. A norma é de extrema importância para garantir a segurança, conforto e bem-estar das pessoas que frequentam os espaços iluminados.

A NBR-5413 estabelece os requisitos mínimos de iluminância (quantidade de luz) que devem ser atendidos em diferentes tipos de ambientes, como escritórios, hospitais, escolas, lojas, vias públicas, entre outros. A norma também especifica as características das lâmpadas, luminárias e equipamentos auxiliares que devem ser utilizados, bem como os procedimentos de medição e avaliação da iluminância.

A norma aborda diversos aspectos importantes para o projeto de iluminação, como a distribuição da luz no ambiente, o controle de ofuscamento (brilho excessivo), a uniformidade da iluminação e o índice de reprodução de cor. Esses fatores são fundamentais para garantir uma iluminação adequada e confortável para as atividades realizadas no ambiente, além de contribuir para a saúde visual dos usuários.

Além disso, a NBR-5413 também estabelece diretrizes para o uso racional da energia elétrica, como a escolha de lâmpadas eficientes e o uso de sistemas de controle de iluminação, como sensores de presença e timers, para evitar o desperdício de energia.

Em resumo, a NBR-5413 é uma norma técnica importante para garantir a qualidade da iluminação artificial em ambientes internos e externos, levando em consideração não apenas a quantidade de luz, mas também a distribuição, uniformidade e eficiência energética. É fundamental que os projetos de iluminação sigam as diretrizes estabelecidas pela norma para garantir a segurança, conforto e bem-estar dos usuários dos espaços iluminados.

\subsection{NORMA NBR-5419}

A NBR-5419 é uma norma técnica brasileira que estabelece as diretrizes para o projeto, instalação e manutenção de sistemas de proteção contra descargas atmosféricas (SPDA). Seu objetivo é proteger as edificações, pessoas e equipamentos contra os efeitos das descargas elétricas atmosféricas, popularmente conhecidas como raios.

A norma aborda diversos aspectos relacionados ao projeto e instalação do SPDA, como a definição da área de proteção, a seleção dos materiais e equipamentos, a disposição dos condutores, as conexões e aterramentos, entre outros. Além disso, a NBR5419 estabelece as inspeções e manutenções necessárias para garantir o funcionamento correto do sistema.

É importante destacar que a instalação de um SPDA é essencial em locais com alta incidência de raios, e sua observância é obrigatória para edificações públicas e privadas com altura superior a 15 metros, segundo a legislação brasileira. A norma NBR-5419 é uma referência obrigatória para os profissionais responsáveis pelo projeto e instalação de sistemas de proteção contra descargas atmosféricas, como engenheiros, eletricistas e projetistas.

Portanto, a NBR-5419 é uma norma técnica fundamental para garantir a segurança das edificações, pessoas e equipamentos contra os efeitos das descargas elétricas atmosféricas. A observância da norma é fundamental para evitar riscos à vida e ao patrimônio, bem como para cumprir a legislação vigente.

\subsection{NORMA NBR-5444}

A NBR-5444 é uma norma técnica que define os símbolos gráficos utilizados em projetos de instalações elétricas prediais. Esses símbolos são padronizados e devem ser utilizados de acordo com as especificações da norma, para garantir uniformidade e clareza na representação das instalações elétricas.

A norma estabelece uma lista de símbolos gráficos que devem ser utilizados para representar os diversos elementos presentes em uma instalação elétrica predial, como tomadas, interruptores, disjuntores, luminárias, entre outros. Cada símbolo tem uma representação gráfica específica, com dimensões e formas padronizadas, permitindo que os projetistas e profissionais envolvidos na instalação elétrica tenham uma visão clara e precisa do projeto.

Além dos símbolos gráficos, a NBR-5444 também estabelece outras regras para a representação gráfica em desenho técnico de instalações elétricas prediais, como as linhas e os tipos de traços que devem ser utilizados, as dimensões das letras e dos números, as escalas, entre outros aspectos.

A padronização dos símbolos gráficos em projetos de instalações elétricas prediais é fundamental para garantir a segurança e a eficiência do sistema elétrico. Com a utilização de símbolos padronizados, é possível minimizar erros de interpretação e falhas na execução da instalação elétrica, evitando acidentes e problemas na utilização do sistema.

Em resumo, a NBR-5444 é uma norma técnica importante para garantir a padronização e a clareza na representação gráfica de projetos de instalações elétricas prediais, permitindo uma comunicação mais eficiente entre os profissionais envolvidos na construção e manutenção das edificações.

\subsection{NORMA NBR-14136}

A ABNT NBR-14136 é uma norma técnica brasileira que estabelece os requisitos de segurança para plugues e tomadas de uso doméstico e análogo, com tensões nominais de até 250 V e correntes nominais de até 20 A. Essa norma tem como objetivo garantir a segurança do usuário em relação a possíveis riscos elétricos, tais como choques elétricos e incêndios.

A norma NBR-14136 estabelece as dimensões, características construtivas, requisitos de segurança, marcações e ensaios que as tomadas e plugues devem atender. Ela determina, por exemplo, que as tomadas devem ser fabricadas com materiais resistentes a altas temperaturas e que os plugues devem ter proteção para evitar o contato acidental com as partes energizadas.

Essa norma é de grande importância para a indústria de equipamentos elétricos, fabricantes de plugues e tomadas, empresas de construção civil e para os consumidores finais, pois ela estabelece um padrão de segurança que deve ser observado na fabricação, instalação e uso desses equipamentos elétricos.

\subsection{NORMA NBR-7488}

A norma NBR-7288 estabelece os requisitos para cabos de potência isolados, com tensão nominal de 1 kV a 6 kV. A norma tem como objetivo garantir a segurança e a qualidade dos cabos de potência, estabelecendo critérios para a seleção dos materiais utilizados na fabricação, o dimensionamento dos condutores, o isolamento elétrico e a montagem dos cabos.

A norma define os critérios para a escolha do tipo de cabo de acordo com a sua aplicação, como por exemplo, para transmissão, distribuição ou uso em instalações fixas. Também são definidos os requisitos para a isolação dos cabos, que deve ser capaz de suportar as condições de operação, como temperatura, pressão e umidade.

A NBR-7288 também define as características dos cabos de potência, como as dimensões dos condutores, a resistência elétrica, a espessura do isolamento, a resistência mecânica e a flexibilidade dos cabos. Além disso, a norma estabelece os ensaios que devem ser realizados para avaliar a qualidade dos cabos, como ensaios elétricos, mecânicos e de envelhecimento.

É importante destacar que o cumprimento da NBR-7288 é obrigatório para a fabricação e instalação de cabos de potência isolados no Brasil. A norma é fundamental para garantir a segurança das instalações elétricas, reduzir o risco de falhas e assegurar a qualidade dos cabos utilizados em sistemas de transmissão e distribuição de energia elétrica.

\subsection{NORMA NBR-5111}

A NBR-5111 é uma norma técnica brasileira que estabelece as especificações para cabos de cobre nus, isolados e cobertos utilizados em instalações elétricas. A norma define os requisitos para as características físicas, mecânicas e elétricas dos cabos, bem como os ensaios que devem ser realizados para avaliar sua qualidade.

A NBR-5111 se aplica a cabos de cobre utilizados em baixa, média e alta tensão, em instalações fixas ou móveis. A norma estabelece as características dos cabos, como a composição química do cobre, as dimensões dos condutores, a resistência elétrica, a espessura do isolamento, a resistência mecânica e a flexibilidade dos cabos.

A norma também estabelece os ensaios que devem ser realizados para avaliar a qualidade dos cabos, como ensaios elétricos, mecânicos e de envelhecimento, que devem ser realizados em laboratórios acreditados pelo INMETRO.

O cumprimento da NBR-5111 é obrigatório para a fabricação e instalação de cabos de cobre no Brasil. A norma é fundamental para garantir a segurança das instalações elétricas, reduzir o risco de falhas e assegurar a qualidade dos cabos utilizados em sistemas de transmissão e distribuição de energia elétrica.

\subsection{NORMA TÉCNICA COPEL - NTC 900100}

A norma NTC 900100 é uma especificação técnica da Companhia Paranaense de Energia (Copel) que define as condições exigidas para a entrada de serviço de energia elétrica em empreendimentos. Ela estabelece os requisitos técnicos para o projeto, construção e montagem das instalações elétricas, incluindo a entrada de energia, o quadro geral de baixa tensão, a proteção contra surtos e curtos-circuitos, dentre outros aspectos relevantes.

A NTC 900100 é aplicável a todos os clientes da Copel que necessitam de uma nova entrada de serviço, além de servir como referência para projetistas, instaladores, construtores e outros profissionais envolvidos na construção de empreendimentos elétricos. É importante ressaltar que essa norma é específica da Copel e não necessariamente se aplica a outras concessionárias de energia elétrica.

\subsection{NORMA TÉCNICA COPEL - NTC 901100}

A norma técnica NTC 901100 estabelece as condições técnicas mínimas para o fornecimento de energia elétrica em tensão secundária em redes de distribuição de energia elétrica. Ela é aplicável a todos os clientes da Companhia Paranaense de Energia (Copel) que utilizam essa tensão de fornecimento.

Entre os aspectos regulados pela NTC 901100 estão: as características da tensão fornecida, os limites de variação da tensão, a proteção contra sobrecarga e curto-circuito, as condições para o fornecimento de energia em emergências, a instalação e a manutenção dos equipamentos elétricos dos clientes, dentre outros.

A NTC 901100 tem como objetivo garantir a qualidade e a continuidade do fornecimento de energia elétrica em tensão secundária, buscando sempre a satisfação do cliente e a segurança das instalações elétricas. É importante ressaltar que essa norma é específica da Copel e não necessariamente se aplica a outras concessionárias de energia elétrica.


Além dessas normas, existem diversas outras normas e regulamentações específicas para cada tipo de instalação elétrica, como as normas da ANEEL (Agência Nacional de Energia Elétrica) e as normas regulamentadoras do Ministério do Trabalho e Emprego (MTE).

Portanto, é fundamental que os profissionais responsáveis pelo projeto de instalações elétricas conheçam e observem as normas técnicas aplicáveis, a fim de garantir a segurança, a qualidade e a conformidade do sistema com as exigências legais.

\subsection{NORMA REGULAMENTADORA 10 - NR-10}

A Norma Regulamentadora NR-10 é uma norma técnica que estabelece os requisitos e as medidas de controle necessárias para garantir a segurança e a saúde dos trabalhadores que atuam em instalações elétricas. A NR-10 é de cumprimento obrigatório para todas as empresas que possuam empregados que trabalhem diretamente com eletricidade, seja na geração, transmissão, distribuição ou consumo.

A NR-10 tem como objetivo principal garantir a segurança dos trabalhadores que atuam com eletricidade, prevenindo acidentes e doenças ocupacionais decorrentes da exposição a riscos elétricos. A norma estabelece uma série de requisitos e medidas de controle, que incluem desde a elaboração de procedimentos de segurança até a utilização de equipamentos de proteção individual e coletiva.

Algumas das principais exigências da NR-10 incluem a realização de treinamentos para os trabalhadores que atuam em instalações elétricas, a elaboração de um prontuário de instalações elétricas, a realização de inspeções periódicas nas instalações elétricas e a adoção de medidas de segurança para a prevenção de acidentes, como o uso de equipamentos de proteção individual e coletiva.

Além disso, a NR-10 estabelece os limites de tensão e corrente elétrica para trabalhos em instalações elétricas, bem como as zonas de risco elétrico e as medidas de segurança necessárias para cada zona.

Em resumo, a NR-10 é uma norma técnica fundamental para garantir a segurança dos trabalhadores que atuam com eletricidade, estabelecendo requisitos e medidas de controle necessárias para prevenir acidentes e doenças ocupacionais decorrentes da exposição a riscos elétricos. O cumprimento da NR-10 é de suma importância para as empresas que atuam com eletricidade, garantindo a segurança e a saúde dos trabalhadores e a conformidade com as normas regulamentadoras do Ministério do Trabalho.

\section{CARACTERÍSTICAS DA EDIFICAÇÃO}
\begin{itemize}
	\item \textbf{\textit{Finalidade:}} Residência;
	\item \textbf{\textit{Paredes:}} Alvenaria;
	\item \textbf{\textit{Tipo de instalação:}} Baixa tensão;
	\item \textbf{\textit{Área total:}} $282$ $m^2$;
	\item \textbf{\textit{Número de pavimentos:}} $2$;
	\item \textbf{\textit{Número de unidades consumidoras:}} 1;
	\item \textbf{\textit{Tensão nominal:}} $127/220$ $V$;
\end{itemize}

\section{INSTALAÇÕES ELÉTRICAS}

\subsection{ENTRADA DE SERVIÇO}

O fornecimento de energia elétrica pela concessionária COPEL será realizada por uma entrada Categoria 41 (NTC 901100), do tipo trifásico, com 4 condutores, sendo 3 fases e 1 neutro de $25$ $mm^2$ (fase e neutro) EPR/XLPE 90°C e fio de proteção (PE) de $16$ $mm^2$ (aterramento) EPR/XLPE 90°C que serão conectados ao quadro de medição localizado próximo ao muro da residência, de modo a permitir fácil acesso ao leitor contratado pela concessionária e evitando problemas futuros para o cliente.

A especificação da entrada de serviço pode ser consultada TABELA 2 da NTC901100, item 9,2, referência ao ítem 4.3, na página 42, sendo a referida tabela mostrada na \autoref{fig:tabela2_ntc901100}:

\pagebreak

\figura
{ENTRADAS DE SERVIÇO - NTC901100} % Legenda
{.8} % % da largura da área de texto
{imagens/tabela2_ntc901100} % localização da figura
{Página 34 da NTC901100} % fonte da figura
{tabela2_ntc901100} % etiqueta
{}
{}

A entrada de serviço fica assim definida:

\begin{itemize}
	\item \textit{Tipo de entrada:} Categoria 41 COPEL, segundo a tabela 2 e a demanda calculada ($34.097,5$ $VA$);
	\item \textit{Ramal de Entrada:} tipo trifásica, categoria 41, com disjuntor trifásico de $100$ $A$ e Medidor Trifásico a 4 fios tipo T;
	\item \textit{Fios de fase/neutro:} cabos de cobre com seção nominal de $25$ $mm^2$ com isolamento EPR/XLPE 90°C;
	\item \textit{Fio de Proteção (PE):} cabo de cobre com seção nominal de $16$ $mm^2$ com isolamento EPR/XLPE 90°C;
	\item \textit{Modo de Instalação:} Maneira ``D'' (Subterrânea);
	\item \textit{Eletrodutos: } eletroduto corrugado com diâmetro de 40mm para o circuito 3F+N e um eletroduto corrugado com 20mm de diâmetro para o condutor de proteção.
\end{itemize}

Notas (NTC 901100): 

\begin{itemize}

	\item Maneira de instalar

	\subitem  \textit{\textbf{Maneira de instalar D:}} cabos unipolares ou cabo multipolar em eletroduto enterrado no solo (com proteção mecânica e/ou química adicional – popularmente cabo 1 kV).
	
	\item As dimensões estabelecidas na tabela para condutores e eletrodutos são mínimas.

	\item Para o ramal de entrada, a seção nominal do condutor neutro deve ser igual ao do(s) condutor (es) da(s) fase(s).

	\item Cada eletroduto deverá possuir um circuito completo [fase(s) e 1 neutro].

	\item Medidor:

	\subitem Medidor Trifásico 4 fios 127/220 V = \textit{Medidor T}
	
\end{itemize}


\subsubsection{ATERRAMENTO}

O aterramento junto ao poste da concessionária será feito através de uma haste de aterramento de $\frac{1}{2} \inches$ ($12,8$ $mm$) e $2,40$ $m$ de comprimento.

A instalação da haste de aterramento deverá ser feita segundo a norma NTC 901100 da COPEL, conforme mostra a \autoref{fig:aterramento_copel}:
\pagebreak 
\figura
{ALTERNATIVA DE ELETRODO - FIGURA 15 DA NTC901100} % Legenda
{.8} % % da largura da área de texto
{imagens/aterramento_copel} % localização da figura
{Página 53 da NTC901100} % fonte da figura
{aterramento_copel} % etiqueta
{}
{}


Notas (NTC 901100):

\begin{itemize}
	\item A instalação do ramal multiplexado (entrada da casa) será subterrânea, ou a maneira ``D'' segundo as notas da tabela 2 da NTC901100:
	
	\item Como alternativa a utilização aos conectores ``GAR'' ou Tipo Parafuso, poderá ser utilizada conexão com solda exotérmica ou conector cunha.
	
	\item A utilização de caixa para a haste de aterramento não é obrigatória.
	
	\item Não será permitida a utilização de conector tipo ``Terminal de Bateria'', conforme prescreve a NTC 927105.
	
\end{itemize}

O aterramento dos quadros de distribuição e seus respectivos barramentos de terra serão feitos utilizxanbdo esquema TN-S, conforme mostra a figura \autoref{fig:esquema_tns}:


\figura
{ESQUEMA TN-S} % Legenda
{.8} % % da largura da área de texto
{imagens/esquema_tns} % localização da figura
{Mundo da Elétrica} % fonte da figura
{esquema_tns} % etiqueta
{}
{}

Neste esquema, os condutores de neutro e proteção (N e PE) são separados, sendo unidos somente na haste de aterramento. O condutor de proteção que sai da haste de aterramento e vai até os quadros de distribuição tem seção nominal de $16$ $mm^2$.

\subsection{QUADROS DE DISTRIBUIÇÃO}

Os Quadro Geral e Auxiliar de Distribuição (QDC1 e QDC2) seguirão os padrões DIN/IEC e NEMA/UL, com disjuntores gerais para cada quadro, de acordo com o projeto. Além disso, os disjuntores para alimentação dos circuitos específicos e os interruptores diferenciais residuais (DR) serão instalados nesses quadros, conforme o diagrama unifilar.

Os disjuntores utilizados nos quadros de distribuição seguem o padrão DIN/IEC e são fornecidos por marcas como STECK, ABB, WEG ou similares, dispostos de acordo com o Diagrama Unifilar e respeitando o balanceamento de fases. A capacidade de condução de corrente dos barramentos também é indicada nos Quadros de Carga em planta. O Quadro de Distribuição será identificado de forma definitiva e duradoura em plaqueta acrílica individual e resinada, com a relação dos circuitos e seus equipamentos correspondentes, e todos os circuitos serão identificados nos quadros com etiquetas e anilhas plásticas. A entrada de energia nos quadros será comandada e protegida por disjuntores, com módulos de reserva para futuras ampliações. Todos os materiais utilizados serão de qualidade e procedência confiável.

De acordo com o item 6.5.4.10 da NBR-5410 \textit{``Os quadros de distribuição destinados a instalações residenciais e análogas devem ser entregues com a seguinte a advertência"} - mostrada na \autoref{fig:advertencia_nbr_5410_item_65410}:

\figura
{Advertência dos Quadros de Distribuição - NBR-5410} % Legenda
{.8} % % da largura da área de texto
{imagens/advertencia_nbr} % localização da figura
{Atom Elétrica} % fonte da figura
{advertencia_nbr_5410_item_65410} % etiqueta
{}
{}

Ainda, o item 6.5.4.11 da NBR-5410 diz que \textit{``A advertência de que trata 6.5.4.10 pode vir de fábrica ou ser provida no local, antes de a instalação ser entregue ao usuário, e não deve ser facilmente removível.''}

No caso de algum disjuntor não poder ser desligado sem aviso prévio aos usuários de equipamentos específicos, é necessário que o disjuntor seja equipado com um acessório apropriado ou algum tipo de sinalização que permita seu funcionamento normal. O uso de fitas adesivas não deve ser utilizado sob nenhuma circunstância. É importante ressaltar que apenas eletricistas qualificados devem ter acesso aos painéis.

\subsection{ILUMINAÇÃO}

Os circuitos de iluminação seguirão as especificações do projeto elétrico e serão derivados dos quadros de distribuição com fiação mínima de $1,5$ $mm^2$. As luminárias internas para área de convivência serão do tipo tubular de LED de $20W$ em chapa de aço galvanizada e pintada na cor branca, com refletor parabólico em alumínio anodizado de alta pureza e refletância. Já para as áreas de guarita, copa, banheiros e recepção serão utilizados plafons de plástico de sobrepor com lâmpadas LED de $9W$ e $12W$.

As caixas embutidas para interruptores seguirão dimensões padronizadas ($4\inches x 2 \inches$, $3 \inches x 3 \inches$ ou $4 \inches x 4 \inches$) e as luminárias serão instaladas em caixas embutidas tipo arandelas nas paredes a $2,20m$ do piso acabado, em caixas embutidas no forro para iluminação interna, e em caixas de ligação à prova de tempo para iluminação externa. As caixas de embutir em ambiente externo deverão ter apenas o olhal superior aberto para conexão com o eletroduto, a fim de evitar entrada de água e corpos estranhos na caixa. E nas caixas internas, apenas os olhais das caixas onde forem introduzidos eletrodutos serão abertos, devendo estar alinhadas e aprumadas.

\subsection{TOMADAS DE USO GERAL}


Tomadas de uso geral são aquelas que podem ser utilizadas para conectar equipamentos elétricos de diferentes tipos, como eletrodomésticos, ferramentas elétricas, aparelhos de som, computadores, entre outros. São pontos de energia elétrica previstos para atender às necessidades de uso comum em ambientes residenciais, comerciais ou industriais. A norma NBR-5410 estabelece as diretrizes para previsão de carga e número mínimo de tomadas de uso geral em diferentes tipos de ambientes. O objetivo é garantir a segurança e a eficiência da instalação elétrica, evitando sobrecarga e riscos de acidentes elétricos.

As tomadas de uso geral deste projeto serão alimentadas a partir dos quadros de distribuição correspondentes, de acordo com as normas e especificações aplicáveis. É exigido que todas as tomadas sejam aterradas, com pino de ligação a terra no padrão Brasileiro de conectores, seguindo as normas de segurança vigentes. As tomadas de uso geral serão projetadas em cada ambiente, próximas à porta de entrada e sob o interruptor da iluminação, em conformidade com as normas e diretrizes do projeto elétrico. As caixas para tomadas deverão ter dimensões padronizadas ($4 \inches x 2 \inches$ ou $4 \inches x 4 \inches$), de forma a permitir a instalação dos módulos previstos e garantir a funcionalidade e segurança do sistema. Todas as tomadas de uso geral deverão ser dotadas de conector de aterramento (PE), conforme ABNT \textit{NBR-14136}, com indicação diferenciada de tensão de trabalho. As tomadas de energia elétrica poderão ser de instalação embutida caixa $4 \inches x 2 \inches$, quando destinadas a uma única tomada, e em caixa $4 \inches x4 \inches$ quando se destinarem a mais de uma tomada. Ademais, é exigido que todas as tomadas sejam providas de fio-terra, conforme normas e especificações técnicas aplicáveis.

As tomadas de energia elétrica a serem utilizadas serão do tipo 2P + T, com capacidade para 10A/250V e serão embutidas na alvenaria de acordo com a altura indicada no projeto. A instalação das tomadas deverá seguir a polarização especificada, garantindo a correta conexão dos polos fase, neutro e terra, confirme a \autoref{fig:fig_tomada}:

\figura
{Polaridade das Tomadas - NBR-14136} % Legenda
{.8} % % da largura da área de texto
{imagens/polaridade_tomadas} % localização da figura
{Sala da Elétrica} % fonte da figura
{fig_tomada} % etiqueta
{}
{}

\subsection{TOMADAS DE USO ESPECÍFICO}
De acordo com a NBR5410, tomada de uso específico é aquela destinada a um único equipamento elétrico, com características especiais de plugue e tensão. São exemplos de equipamentos que necessitam de tomadas de uso específico: ar-condicionado, forno elétrico, máquina de lavar, secadora de roupas, entre outros. A norma estabelece requisitos específicos para as tomadas de uso específico, como a necessidade de serem identificadas com a informação do equipamento que deve ser conectado, além de exigir a instalação de dispositivos de proteção contra choques elétricos, como o DR. É importante ressaltar que as tomadas de uso específico devem atender aos requisitos de segurança estabelecidos na NBR-14136, que regulamenta as tomadas de uso geral e de uso específico.

As tomadas de uso específico para chuveiros e equipamentos com potência superior a $3000W$ serão constituídas por fios de 2 fases, neutro e terra saindo do Quadro de Distribuição corespondente, sendo a seção do aterramento o mesmo dos condutores carregados desse circuito, segundo as definições da NBR-5410. 

As tomadas de uso específico destinadas a equipamentos da cozinha e lavanderia serão alimentadas por fios de fase, neutro e terra saindo dos quadros de distribuição correspondente, sendo a seção do aterramento o mesmo dos condutores carregados desse circuito, segundo as definições da NBR-5410. As tomadas de energia elétrica a serem utilizadas serão do tipo 2P + T, com capacidade para 20A/250V e serão embutidas na alvenaria de acordo com a altura indicada no projeto.

Todas as tomadas de tensão nominal de $127V$, exceto as tomadas para chuveiros e equipamentos com potência superior a $3000W$, cuja tensão nominal é de $220V$.

\subsection{ELETRODUTOS}

Todos os circuitos serão instalados em eletrodutos de PVC corrugados, de cor amarela, com propriedades antichamas e antitóxicos, embutidos em paredes ou em lajes. Serão adotados eletrodutos de $1$ $\frac{1}{4} \inches$ quando não houver indicação de diâmetro externo. Os eletrodutos serão instalados de forma a formar uma rede contínua de caixa a caixa e de luminária a luminária, permitindo a remoção e transposição dos condutores sem prejuízo para o isolamento. Os eletrodutos utilizados para ligação das luminárias aos interruptores seguirão o mesmo padrão dos demais eletrodutos.

As caixas de passagem e eletrodutos deverão formar uma rede rígida. As conexões serão feitas com luvas roscadas, sem ângulos superiores a 90 graus em uma única curva. As fixações em caixas metálicas exigirão buchas e arruelas, sendo que as tubulações vazias deverão ter guias de arame para facilitar a enfiação. Tampões deverão ser colocados nos eletrodutos logo após a instalação para impedir a entrada de objetos estranhos.


\subsection{CAIXAS DE PASSAGEM}

As caixas de passagem deverão ser padronizadas nas medidas de $4\inches x 2 \inches$, $3 \inches x 3 \inches$ ou $4 \inches x 4 \inches$, feitas de PVC para embutir em alvenaria.

\subsection{CONDUTORES}

Todos os condutores devem ser cabos isolados, exceto se indicado de outra forma, e devem possuir características especiais de propagação e autoextinção do fogo. Os cabos utilizados para alimentar a iluminação interna/externa e as tomadas devem ser do tipo cabo com isolamento para 450/750 V, com isolamento simples e das marcas Ficap, Pirelli ou similares, conforme a norma NBR-7288, com a bitola especificada em planta. É proibido realizar emendas nos cabos e todas as caixas de passagem são destinadas a facilitar a passagem dos cabos.

Os cabos utilizados para alimentar os quadros de distribuição devem ser unipolares de cobre, com capacidade de 0,6/1kV e isolamento de EPR/XLPE 90°C. As seções dos condutores estão indicadas nos quadros de carga e diagramas. Todos os cabos devem ser do tipo cabo e devem possuir as características especificadas:

\begin{itemize}
	\item O condutor é composto por fio de cobre nu, de têmpera mole, com encordoamento classe 2;
	\item  A isolação é feita de composto termofixo de Polietileno reticulado EPR/XLPE, com espessura reforçada e é anti-chama, sem capa de chumbo;
	\item As temperaturas máximas suportadas pelo condutor são de 90°C em serviço contínuo, 130°C em sobrecarga e 250°C em curto circuito.
\end{itemize}

A instalação dos condutores só pode começar após a tubulação estar completamente instalada, fixada e limpa, após a primeira camada de tinta nas paredes e antes da camada final. Para facilitar a instalação dos cabos nas tubulações, apenas o uso de parafina ou talco é permitido.

Emendas são permitidas apenas dentro das caixas de passagem, e devem ser bem soldadas e isoladas com fita isolante antichama da 3M ou de marca similar, ou ainda serem feitas utilizando conectores de torção. Emendas dentro de eletrodutos não são permitidas em nenhuma circunstância.

Conectores terminais do tipo ilhós devem ser usados para conectar os cabos aos barramentos ou bornes das chaves e disjuntores, para bitolas acima de $6$ $mm^2$.

\subsection{CIRCUITOS}

Dentro de cada eletroduto, serão utilizados no máximo 3 (três) cabos para circuitos monofásicos + terra ou bifásicos + terra, e 5 cabos para circuitos trifásicos a 4 fios + terra, com até 3 (três) ou 4 (quatro) circuitos. A retirada da cobertura ou isolação dos eletrodutos sem consulta prévia ao projetista será vedada. Os circuitos alimentadores dos quadros de distribuição serão identificados em planta, ao longo dos eletrodutos. Equipamentos especiais, como chuveiros e torneiras elétricas, devem ser ligados diretamente ao quadro de distribuição específico, com um conduto único para cada circuito. As condensadoras de ar deverão ser ligadas diretamente ao quadro de distribuição, com no máximo dois circuitos por conduto. Os condutores não deverão sofrer esforços mecânicos incompatíveis.

\subsection{ATERRAMENTO}

Os circuitos de distribuição devem ser acompanhados por condutores de proteção (terra), de acordo com o projeto, e todos os quadros devem ter um barramento de terra. É estritamente proibido conectar os condutores neutro e de proteção (terra) nos quadros de Distribuição de cargas geral ou terminal. Além disso, todos os condutores de proteção (terra) devem estar isolados no interior dos eletrodutos.

O aterramento dos barramentos de terra dos quadros de distribuição serão feitos utilizando esquema TN-S, segundo a norma NBR5410 - utilizando o eletrodo de aterramento da entrada de serviço.

\subsection{OBSERVAÇÕES}

Caso o cliente deseje alterar qualquer item, como por exemplo uma luminária, é necessário verificar a potência do dispositivo a ser substituído. Se a nova potência for maior do que a anterior, é necessário recalcular o dimensionamento dos condutores e disjuntores.

\section{RESPONSABILIDADE TÉCNICA}

A responsabilidade técnica atribuída a este projeto está sujeita à manutenção de todas as características, definições e especificações dos dispositivos, equipamentos e materiais descritos neste projeto, que devem ser utilizados durante a sua execução. Além disso, qualquer alteração que se faça necessária deve ser avaliada e autorizada por escrito pelo responsável técnico do projeto.