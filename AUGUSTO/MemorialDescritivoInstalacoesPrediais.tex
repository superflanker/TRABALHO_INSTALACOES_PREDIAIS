%% abtex2-modelo-trabalho-academico.tex, v-1.9.2 laurocesar
%% Copyright 2012-2017 by abnTeX2 group at http://abntex2.googlecode.com/ 
%%
%% This work may be distributed and/or modified under the
%% conditions of the LaTeX Project Public License, either version 1.3
%% of this license or (at your option) any later version.
%% The latest version of this license is in
%%   http://www.latex-project.org/lppl.txt
%% and version 1.3 or later is part of all distributions of LaTeX
%% version 2005/12/01 or later.
%%
%% This work has the LPPL maintenance status `maintained'.
%% 
%% The Current Maintainer of this work is Emílio Eiji Kavamura,
%% eek.edu@outlook.com; emilio.kavamura@ufpr.br
%% Further information about abnTeX2 are available on 
%%
%% http://abntex2.googlecode.com/
%%
%% https://code.google.com/p/abntex2/issues/ 
%% Further information about UFPR abnTeX2 are available on 
%%
%% https://github.com/eekBR/ufpr-abntex/
%%
%% This work consists of the files 
% 
%          main.tex   programa principal
%      00-dados.tex   entrada de dados 
%    00-pacotes.tex   pacotes carregados no modelo
% 00-pretextual.tex   processamento dos elementos pre-textuais
%          UFPR.sty   ajusta do modelo canonico às normas  UFPR
%
%    referencias.bib
%
%
%
%------------------------------------------------------------------------
% ------------------------------------------------------------------------
% abnTeX2: Modelo de Trabalho Academico (tese de doutorado, dissertacao de
% mestrado e trabalhos monograficos em geral) em conformidade com 
% ABNT NBR 14724:2011: Informacao e documentacao - Trabalhos academicos -
% Apresentacao
% ------------------------------------------------------------------------
% ------------------------------------------------------------------------

\documentclass[
        % -- opções da classe memoir --
        12pt,                           % tamanho da fonte
        openright,                      % capítulos começam em pág ímpar (insere página vazia caso preciso)
        %twoside,                        % para impressão em verso e anverso. Oposto a oneside
        oneside,
        a4paper,                        % tamanho do papel. 
        % -- opções da classe abntex2 --
        chapter=TITLE,         % títulos de capítulos convertidos em letras maiúsculas
        section=TITLE,         % títulos de seções convertidos em letras maiúsculas
        subsection=Title,      % títulos de subseções convertidos em letras maiúsculas
        %subsubsection=TITLE,  % títulos de subsubseções convertidos em letras maiúsculas
        % -- opções do pacote babel --
        % english,                        % idioma adicional para hifenização
        % french,                         % idioma adicional para hifenização
        % spanish,                        % idioma adicional para hifenização
        portugues,                      % o último idioma é o principal do documento
        %%%%%%%%%%%%
        %eek: colocação da opção para o sumario ter formatação tradicional
        sumario=tradicional             % título no formato tradicional
        ]{abntex2}


\usepackage{UFPR}
\usepackage{longtable}
% Pacotes básicos 
% ----------------------------------------------------------
%\usepackage{lmodern}			% Usa a fonte Latin Modern			
\usepackage[T1]{fontenc}		% Selecao de codigos de fonte.
\usepackage[utf8]{inputenc}		% Codificacao do documento (conversão automática dos acentos)
\usepackage{lastpage}			% Usado pela Ficha catalográfica
\usepackage{indentfirst}		% Indenta o primeiro parágrafo de cada seção.
\usepackage{color}		    	% Controle das cores
\usepackage{graphicx}			% Inclusão de gráficos
\usepackage{microtype} 			% para melhorias de justificação
\usepackage{ifthen}		    	% para montar condicionais
\usepackage[brazil]{babel}		% para utilizar termos em portugues
\usepackage[final]{pdfpages}    % para incluir páginas de arquivos pdf
\usepackage{lipsum}				% para geração de dummy text
\usepackage{csquotes}

%\usepackage[style=long]{glossaries}
%\usepackage{abntex2glossaries}

% permite representar o cancelamento de termos em texto ou equacoes
\usepackage{cancel} 		
% cores extendidas
\usepackage{xcolor} 		
% gera diagramas a partir de listas
\usepackage{smartdiagram}   
% Para a figura ficar na posição correta
\usepackage{float} 		    
% supporte para fontes da Text Companion 
\usepackage{textcomp} 		
% uso de longtable
\usepackage{longtable}		
% simbolos matematicos
\usepackage{amsmath}	
% páginas em paisagem
\usepackage{lscape}
% mescla de colunas em tabelas
\usepackage{multicol}
% mescla de linhas em tabelas
\usepackage{multirow}
% criação do indice de quadros
\usepackage{newfloat} 
% configura legenda 
\usepackage{caption} 	
% inclusão de pdfs
\usepackage{pdfpages}	
% tabela multirow
\usepackage{multirow}
% pacode de code highlighting
% \usepackage{minted}
% tikz
% \usepackage{tikz}
% \usetikzlibrary{automata, positioning, arrows}
% Greek-specific commands
%[format=plain]
	%\renewcommand\caption[1]{%
    \captionsetup{font=small}	% tamanho da fonte 10pt
    %,format=hang
 	% \caption{#1}}
	%\captionsetup{width=0.8\textwidth}

% Pacotes de citações BibLaTeX
% ----------------------------------------------------------
\usepackage[style=abnt,backref=true,backend=biber,citecounter=true,backrefstyle=three]{biblatex}


\DefineBibliographyStrings{brazil}{%
 backrefpage = {Citado \arabic{citecounter} vez na página},% originally "cited on page"
 backrefpages = {Citado \arabic{citecounter} vezes nas páginas},% originally "cited on pages"
}

% ----------------------------------------------------------

% alterando o aspecto da cor azul
\definecolor{blue}{RGB}{55,10,249}

% ----------------------------------------------------------
\PrepareListOf{quadro}{%
\renewcommand{\cftfigpresnum}{Quadro~}}

\DeclareFloatingEnvironment[
fileext=loq,
listname={\textbf{LISTA DE QUADROS}},
name=Quadro,
%placement=p,
within= none, % numeracao continua
%within=section, % numeracao reinicia em cada seccao
%chapterlistsgaps=off
]{quadro}

\newlistentry{quadro}{loq}{0}


% Customize ‘List of Diagrams’
\PrepareListOf{quadro}{%
\renewcommand{\cftquadropresnum}{\normalsize{QUADRO}~}
\setlength{\cftquadronumwidth}{3.2cm}
%\renewcommand{\cftquadroname}{\quadroname\space} 
\renewcommand*{\cftquadroaftersnum}{\hfill--\hfill}
}

\makeatletter
%% we define a helper macro for adjusting lists of new floats to
%% accept a * behind them for not being shown in the TOC, like
%% the other list printing commands in memoir
\newcommand{\AdjustForMemoir}[1]{%
  \csletcs{kept@listof#1}{listof#1}%
  \csdef{listof#1}{%
    \@ifstar
     {\csappto{newfloat@listof#1@hook}{\append@star}%
      \csuse{kept@listof#1}}%
     {\csuse{kept@listof#1}}%
  }
}
\def\append@star#1{#1*}
\makeatother


\AdjustForMemoir{quadro} % prepare `\listofdirfigures` so it accepts a *

\makeatletter
\let\oldcontentsline\contentsline
\def\contentsline#1#2{%
    \expandafter\ifx\csname l@#1\endcsname\l@section
	\expandafter\@firstoftwo
	\else
	\expandafter\@secondoftwo
	\fi
	{%
		\oldcontentsline{#1}{\MakeTextUppercase{#2}}%
	}{%
	\normalsize %ajusta tamanho da fonte na lista
	\oldcontentsline{#1}{#2}%
}%
}
\makeatother

% Ajusta indentação de Referencias no ToC
% ----------------------------------------------------------
\defbibheading{bay}[\bibname]{%
  \chapter*{#1}%
  \markboth{#1}{#1}%
  \addcontentsline{toc}{chapter}
  {\protect\numberline{}\bibname}
}

\makeatletter
\pretocmd{\chapter}{\addtocontents{toc}{\protect\addvspace{-5\p@}}}{}{}
\pretocmd{\section}{\addtocontents{toc}{\protect\addvspace{2\p@}}}{}{}
\makeatother
%%%%%%%%%%%%%%%%%%%%%%%%%%%%%%%%%%%%%%%%%%%%%%%%%%%%%%%
% Arquivo para entrada de dados para a parte pré textual
%%%%%%%%%%%%%%%%%%%%%%%%%%%%%%%%%%%%%%%%%%%%%%%%%%%%%%%
% 
% Basta digitar as informações indicidas, no formato 
% apresentado.
%
%%%%%%%
% Os dados solicitados são, na ordem:
%
% tipo do trabalho
% componentes do trabalho 
% título do trabalho
% nome do autor
% local 
% data (ano com 4 dígitos)
% orientador(a)
% coorientador(a)(as)(es)
% arquivo com dados bibliográficos
% instituição
% setor
% programa de pós gradução
% curso
% preambulo
% data defesa
% CDU
% errata
% assinaturas - termo de aprovação
% resumos & palavras chave
% agradecimentos
% dedicatoria
% epígrafe


% Informações de dados para CAPA e FOLHA DE ROSTO
%----------------------------------------------------------------------------- 
\tipotrabalho{Relatório Técnico}
%\tipotrabalho{Trabalho Acadêmico}
%    {Relatório Técnico}
%    {Dissertação}
%    {Tese}
%    {Monografia}

% Marcar Sim para as partes que irão compor o documento pdf
%----------------------------------------------------------------------------- 
 \providecommand{\terCapa}{Sim}
 \providecommand{\terFolhaRosto}{Sim}
 \providecommand{\terTermoAprovacao}{Nao}
 \providecommand{\terDedicatoria}{Nao}
 \providecommand{\terFichaCatalografica}{Nao}
 \providecommand{\terEpigrafe}{Nao}
 \providecommand{\terAgradecimentos}{Nao}
 \providecommand{\terErrata}{Nao}
 \providecommand{\terListaFiguras}{Sim}
 \providecommand{\terListaTabelas}{Sim}
 \providecommand{\terSiglasAbrev}{Nao}
 \providecommand{\terResumos}{Nao}
 \providecommand{\terSumario}{Sim}
 \providecommand{\terAnexo}{Nao}
 \providecommand{\terApendice}{Nao}
 \providecommand{\terIndiceR}{Nao}
%----------------------------------------------------------------------------- 

\titulo{Memorial Descritivo\\Memorial De Cálculo\\Residência TE344}
\autor{Augusto Mathias Adams - GRR20172143\\
	   Claudinei Eustaquio Rodrigues  - GRR20191718\\
   	   Moisés Alves Guergolet - GRR20202614}
\local{Curitiba}
\data{2023} %Apenas ano 4 dígitos

% Orientador ou Orientadora
\orientador{
	Prof\textordfeminine~Cleverson Luiz da Silva Pinto, Dr.}
%Prof Emílio Eiji Kavamura, MSc}
\orientadora{}
% Pode haver apenas uma orientadora ou um orientador
% Se houver os dois prevalece o feminino.

% Em termos de coorientação, podem haver até quatro neste modelo
% Sendo 2 mulhere e 2 homens.
% Coorientador ou Coorientadora
%\coorientador{}%Prof Morgan Freeman, DSc}
%\coorientadora{Prof\textordfeminine~Audrey Hepburn, DEng}

% Segundo Coorientador ou Segunda Coorientadora
\scoorientador{}
%Prof Jack Nicholson, DEng}
\scoorientadora{}
%Prof\textordfeminine~Ingrid Bergman, DEng}
% ----------------------------------------------------------
% \addbibresource{referencias.bib}

% ----------------------------------------------------------
\instituicao{%
Universidade Federal do Paraná}

\def \ImprimirSetor{}%
%Setor de Tecnologia}

\def \ImprimirProgramaPos{}%Programa de Pós Graduação em Engenharia de Construção Civil}

\def \ImprimirCurso{}%
%Curso de Engenharia Civil}

\preambulo{
Trabalho apresentado como requisito parcial de avaliação da Disciplina INSTALAÇÕES ELÉTRICAS PREDIAIS E INDUSTRIAIS I (TE-344) do Curso de Engenharia Elétrica, ênfase em Sistemas Embarcados - Noturno, do Departamento de Engenharia Elétrica da Universidade Federal do Paraná}

%----------------------------------------------------------------------------- 

\newcommand{\imprimirCurso}{}
%Programa de P\'os Gradua\c{c}\~ao em Engenharia da Constru\c{c}\~ao Civil}

\newcommand{\imprimirDataDefesa}{
09 de Dezembro de 2018}

\newcommand{\imprimircdu}{
02:141:005.7}

% ----------------------------------------------------------
\newcommand{\imprimirerrata}{
Elemento opcional da \cites[4.2.1.2]{NBR14724:2011}. Exemplo:

\vspace{\onelineskip}

FERRIGNO, C. R. A. \textbf{Tratamento de neoplasias ósseas apendiculares com
reimplantação de enxerto ósseo autólogo autoclavado associado ao plasma
rico em plaquetas}: estudo crítico na cirurgia de preservação de membro em
cães. 2011. 128 f. Tese (Livre-Docência) - Faculdade de Medicina Veterinária e
Zootecnia, Universidade de São Paulo, São Paulo, 2011.

\begin{table}[htb]
\center
\footnotesize
\begin{tabular}{|p{1.4cm}|p{1cm}|p{3cm}|p{3cm}|}
  \hline
   \textbf{Folha} & \textbf{Linha}  & \textbf{Onde se lê}  & \textbf{Leia-se}  \\
    \hline
    1 & 10 & auto-conclavo & autoconclavo\\
   \hline
\end{tabular}
\end{table}}

% Comandos de dados - Data da apresentação
\providecommand{\imprimirdataapresentacaoRotulo}{}
\providecommand{\imprimirdataapresentacao}{}
\newcommand{\dataapresentacao}[2][\dataapresentacaoname]{\renewcommand{\dataapresentacao}{#2}}

% Comandos de dados - Nome do Curso
\providecommand{\imprimirnomedocursoRotulo}{}
\providecommand{\imprimirnomedocurso}{}
\newcommand{\nomedocurso}[2][\nomedocursoname]
  {\renewcommand{\imprimirnomedocursoRotulo}{#1}
\renewcommand{\imprimirnomedocurso}{#2}}


% ----------------------------------------------------------
\newcommand{\AssinaAprovacao}{

\assinatura{%\textbf
   {Professora} \\ UFPR}
   \assinatura{%\textbf
   {Professora} \\ ENSEADE}
   \assinatura{%\textbf
   {Professora} \\ TIT}
   %\assinatura{%\textbf{Professor} \\ Convidado 4}
      
   \begin{center}
    \vspace*{0.5cm}
    %{\large\imprimirlocal}
    %\par
    %{\large\imprimirdata}
    \imprimirlocal, \imprimirDataDefesa.
    \vspace*{1cm}
  \end{center}
  }
  
% ----------------------------------------------------------
%\newcommand{\Errata}{%\color{blue}
%Elemento opcional da \textcite[4.2.1.2]{NBR14724:2011}. Exemplo:
%}

% ----------------------------------------------------------
\newcommand{\EpigrafeTexto}{%\color{blue}
\textit{``Não vos amoldeis às estruturas deste mundo, \\
		mas transformai-vos pela renovação da mente, \\
		a fim de distinguir qual é a vontade de Deus: \\
		o que é bom, o que Lhe é agradável, o que é perfeito.\\
		(Bíblia Sagrada, Romanos 12, 2)}
}

% ----------------------------------------------------------
\newcommand{\ResumoTexto}{
A aula prática referente ao laboratório 04 da disciplina de Microeletrônica I trata de 2 projetos de circuitos lógicos sequenciais: Temporizador com contagem de 0 a 9 e contador BCD progressivo/regressivo com habilitação. Todos os circuitos lógicos foram implementados utilizando o kit Nexys 2 da Digilent, utilizando a Ferramenta Ise, versão 14.7, da Xilink. Este relatório inclui a descrição e o desenvolvimento dos dois circuitos lógicos, juntamente com seus esquemáticos RTL obtidos após a síntese. Também constam  as simulações, explicação dos resultados, os códigos da implementação, do \textit{"testbench"} e o resumo dos recursos lógicos utilizados. 

}

\newcommand{\PalavraschaveTexto}{%\color{blue}
Circuitos lógicos; FPGA; Lógica Sequencial; Contadores; Temporizadores.}

% ----------------------------------------------------------
\newcommand{\AbstractTexto}{%\color{blue}
The practical class referring to laboratory 04 of the discipline of Microelectronics I deals with 2 projects of sequential logic circuits: Timer with count from 0 to 9 and progressive/regressive BCD counter with enabling. All logic circuits were implemented using Digilent's Nexys 2 kit, using Xilink's Ise Tool, version 14.7. This report includes the description and development of the two logic circuits, along with their RTL schematics obtained after synthesis. There are also simulations, explanation of results, implementation codes, \textit{"testbench"} and a summary of the logic resources used.
}
% ---
\newcommand{\KeywordsTexto}{%\color{blue}
logic circuits; FPGA; Sequential Logic; Counters; Timers.}

% ----------------------------------------------------------
\newcommand{\Resume}
{%\color{blue}
%Il s'agit d'un résumé en français.
} 
% ---
\newcommand{\Motscles}
{%\color{blue}
 %latex. abntex. publication de textes.
}

% ----------------------------------------------------------
\newcommand{\Resumen}
{%\color{blue}
%Este es el resumen en español.
}
% ---
\newcommand{\Palabrasclave}
{%\color{blue}
%latex. abntex. publicación de textos.
}

% ----------------------------------------------------------
\newcommand{\AgradecimentosTexto}{%\color{blue}
Os agradecimentos principais são direcionados à Gerald Weber, Miguel Frasson, Leslie H. Watter, Bruno Parente Lima, Flávio de  Vasconcellos Corrêa, Otavio Real Salvador, Renato Machnievscz\footnote{Os nomes dos integrantes do primeiro
projeto abn\TeX\ foram extraídos de \url{http://codigolivre.org.br/projects/abntex/}} e todos aqueles que contribuíram para que a produção de trabalhos acadêmicos conforme as normas ABNT com \LaTeX\ fosse possível.

Agradecimentos especiais são direcionados ao Centro de Pesquisa em Arquitetura da Informação\footnote{\url{http://www.cpai.unb.br/}} da Universidade de Brasília (CPAI), ao grupo de usuários
\emph{latex-br}\footnote{\url{http://groups.google.com/group/latex-br}} e aos novos voluntários do grupo \emph{\abnTeX}\footnote{\url{http://groups.google.com/group/abntex2} e
\url{http://abntex2.googlecode.com/}}~que contribuíram e que ainda
contribuirão para a evolução do \abnTeX.

Os agradecimentos principais são direcionados à Gerald Weber, Miguel Frasson, Leslie H. Watter, Bruno Parente Lima, Flávio de Vasconcellos Corrêa, Otavio Real Salvador, Renato Machnievscz\footnote{Os nomes dos integrantes do primeiro
projeto abn\TeX\ foram extraídos de \url{http://codigolivre.org.br/projects/abntex/}} e todos aqueles que contribuíram para que a produção de trabalhos acadêmicos conforme as normas ABNT com \LaTeX\ fosse possível.
}

% ----------------------------------------------------------
\newcommand{\DedicatoriaTexto}{%\color{blue}
\textit{ Este trabalho é dedicado às crianças adultas que,\\
   quando pequenas, sonharam em se tornar cientistas.}
	}



% compila o indice
% ----------------------------------------------------------

\makeindex
% ----------------------------------------------------------
% Início do documento
% ----------------------------------------------------------
\begin{document}
% ----------------------------------------------------------
% Adequando o uppercase titulo dos elementos nas suas respectivas legendas
% Definicoes que n\~ao funcionaram quando colocados no arquivo de estilos ou de pacotes

\renewcommand{\bibname}{{REFER\^ENCIAS}}
\renewcommand{\tablename}{TABELA }
\renewcommand{\figurename}{FIGURA }
\renewcommand{\figureautorefname}{FIGURA}
\renewcommand{\tableautorefname}{TABELA}
\newcommand{\inches}{\ensuremath{{}^{\prime\prime}}}

% ELEMENTOS PRÉ-TEXTUAIS
% ----------------------------------------------------------
% Capa
% ----------------------------------------------------------
 \ifthenelse{\equal{\terCapa}{Sim}}{
\imprimircapa}{}

% Folha de rosto
% ----------------------------------------------------------
\imprimirfolhaderosto*

% Inserir a ficha bibliografica
% ----------------------------------------------------------
 \ifthenelse{\equal{\terFichaCatalografica}{Sim}}
 {\insereFichaCatalografica{}\cleardoublepage}
 {}

% Inserir errata
% ----------------------------------------------------------
 \ifthenelse{\equal{\terErrata}{Sim}}
 {\begin{errata}%\color{blue}
   \imprimirerrata
  \end{errata}}
 {}

% Inserir folha de aprovação
% ----------------------------------------------------------
\ifthenelse{\equal{\terTermoAprovacao}{Sim}}{
\insereAprovacao}{}

% Dedicatória
% ----------------------------------------------------------
\ifthenelse{\equal{\terDedicatoria}{Sim}}{
\begin{dedicatoria}
   \vspace*{\fill}
   \centering
   \noindent
   \DedicatoriaTexto
   \vspace*{\fill}
\end{dedicatoria}
}{}

% Agradecimentos
% ----------------------------------------------------------

 \ifthenelse{\equal{\terAgradecimentos}{Sim}}
 {\begin{agradecimentos}
    \AgradecimentosTexto
  \end{agradecimentos}
  }{}
% Epígrafe
% ----------------------------------------------------------

\ifthenelse{\equal{\terEpigrafe}{Sim}}{
\begin{epigrafe}
    \vspace*{\fill}
	\begin{flushright}
        \EpigrafeTexto
	\end{flushright}
\end{epigrafe}
}{}

% RESUMOS
% ----------------------------------------------------------
% resumo em português
%\setlength{\absparsep}{18pt} % ajusta o espaçamento dos parágrafos do resumo
 \ifthenelse{\equal{\terResumos}{Sim}}{
\begin{resumo}
    \ResumoTexto
    
    %\vspace{\onelineskip}
    \noindent 
    \textbf{Palavras-chaves}: \PalavraschaveTexto
\end{resumo}

%% resumo em inglês
\begin{resumo}[ABSTRACT]
 \begin{otherlanguage*}{english}
   \AbstractTexto
   
   %\vspace{\onelineskip}
   \noindent 
   \textbf{Key-words}: \KeywordsTexto
 \end{otherlanguage*}
\end{resumo}


% resumo em francês 
\ifthenelse{\equal{\Resume}{}}
{}
{
 \begin{resumo}[RESUME]%Résumé
  \begin{otherlanguage*}{french}
     \Resume
     
     %\vspace{\onelineskip}
     \noindent      
     \textbf{Mots clés}: \Motscles
  \end{otherlanguage*}
 \end{resumo}
} 

% resumo em espanhol
\ifthenelse{\equal{\Resume}{}}{}
{ \begin{resumo}[RESUMEN]
  \begin{otherlanguage*}{spanish}
    \Resumen 
   
   %\vspace{\onelineskip}
   \noindent    
    \textbf{Palabras clave}: \Palabrasclave
  \end{otherlanguage*}
 \end{resumo}
}
}{}

% inserir lista de ilustrações
% ----------------------------------------------------------
\ifthenelse{\equal{\terListaFiguras}{Sim}}{
%\pdfbookmark[0]{\listfigurename}{lof}
\listoffigures*
\cleardoublepage
}{}


% inserir lista de tabelas
% ----------------------------------------------------------
\ifthenelse{\equal{\terListaTabelas}{Sim}}{
%\pdfbookmark[0]{\listtablename}{lot}
\listoftables*
\cleardoublepage
}{}


% inserir lista de abreviaturas e siglas 
% inserir lista de símbolos
% ----------------------------------------------------------

 \ifthenelse{\equal{\terSiglasAbrev}{Sim}}{
    \imprimirlistadesiglas
    \cleardoublepage
    \imprimirlistadesimbolos
    \cleardoublepage
 }{}

% inserir o sumario
% ----------------------------------------------------------
\ifthenelse{\equal{\terSumario}{Sim}}{
%\pdfbookmark[0]{\contentsname}{toc}
\tableofcontents*
%\cleardoublepage
}{}
 

 
 


% ----------------------------------------------------------
% ELEMENTOS TEXTUAIS
% ----------------------------------------------------------
\textual % \pagestyle{textualUFPR}

\pagestyle{simple}
% sugerido por Youssef Cherem 20170316
% https://mail.google.com/mail/u/0/?tab=wm#inbox/15ad3fe6f4e5ff1f

% Introdução (exemplo de capítulo sem numeração, mas presente no Sumário)
% ----------------------------------------------------------
\chapter[MEMORIAL DESCRITIVO]{MEMORIAL DESCRITIVO}

\section{APRESENTAÇÃO}

Este memorial descritivo é um documento técnico que descreve detalhadamente os aspectos da instalação elétrica da residência TE344, localizada na Rua dos Alfeneiros, número 04, bairro Boa Vista na cidade de Curitiba - Paraná - Brasil. O objetivo principal é fornecer informações precisas e completas sobre a instalação elétrica, incluindo os materiais utilizados, as normas técnicas aplicáveis e o dimensionamento dos componentes elétricos. A partir deste documento é possível ter uma visão geral do funcionamento e características da instalação elétrica, além de garantir sua segurança e eficiência. 

Essa documentação é essencial para garantir a segurança das pessoas que frequentam o local, bem como para assegurar o bom funcionamento dos equipamentos elétricos. 

\section{NORMAS APLICÁVEIS}

\subsection{NORMA NBR-5410}

A NBR-5410 é uma norma técnica brasileira que estabelece as condições necessárias para a instalação elétrica de baixa tensão em edificações. Ela define os princípios e requisitos mínimos de segurança para o projeto, construção e manutenção de instalações elétricas, visando garantir a proteção das pessoas e a integridade das edificações.

Entre os aspectos abordados pela norma estão a seleção dos materiais e equipamentos, a instalação de condutores e dispositivos de proteção, a execução de aterramentos, a proteção contra sobrecargas e curto-circuitos, entre outros.

A NBR-5410 é de suma importância para a segurança elétrica das edificações no Brasil e é uma referência obrigatória para os profissionais do setor, como engenheiros elétricos, eletricistas e projetistas. É importante ressaltar que o não cumprimento das normas pode acarretar riscos de acidentes elétricos, além de problemas na regularização de projetos e obras.

Portanto, a NBR-5410 é uma norma técnica fundamental para garantir a segurança nas instalações elétricas de baixa tensão em edificações, e sua observância é fundamental para evitar riscos à vida e ao patrimônio.

\subsection{NORMA NBR-5413}

A NBR-5413 é uma norma técnica brasileira que estabelece os critérios e parâmetros para o projeto de iluminação artificial em ambientes internos e externos. A norma é de extrema importância para garantir a segurança, conforto e bem-estar das pessoas que frequentam os espaços iluminados.

A NBR-5413 estabelece os requisitos mínimos de iluminância (quantidade de luz) que devem ser atendidos em diferentes tipos de ambientes, como escritórios, hospitais, escolas, lojas, vias públicas, entre outros. A norma também especifica as características das lâmpadas, luminárias e equipamentos auxiliares que devem ser utilizados, bem como os procedimentos de medição e avaliação da iluminância.

A norma aborda diversos aspectos importantes para o projeto de iluminação, como a distribuição da luz no ambiente, o controle de ofuscamento (brilho excessivo), a uniformidade da iluminação e o índice de reprodução de cor. Esses fatores são fundamentais para garantir uma iluminação adequada e confortável para as atividades realizadas no ambiente, além de contribuir para a saúde visual dos usuários.

Além disso, a NBR-5413 também estabelece diretrizes para o uso racional da energia elétrica, como a escolha de lâmpadas eficientes e o uso de sistemas de controle de iluminação, como sensores de presença e timers, para evitar o desperdício de energia.

Em resumo, a NBR-5413 é uma norma técnica importante para garantir a qualidade da iluminação artificial em ambientes internos e externos, levando em consideração não apenas a quantidade de luz, mas também a distribuição, uniformidade e eficiência energética. É fundamental que os projetos de iluminação sigam as diretrizes estabelecidas pela norma para garantir a segurança, conforto e bem-estar dos usuários dos espaços iluminados.

\subsection{NORMA NBR-5419}

A NBR-5419 é uma norma técnica brasileira que estabelece as diretrizes para o projeto, instalação e manutenção de sistemas de proteção contra descargas atmosféricas (SPDA). Seu objetivo é proteger as edificações, pessoas e equipamentos contra os efeitos das descargas elétricas atmosféricas, popularmente conhecidas como raios.

A norma aborda diversos aspectos relacionados ao projeto e instalação do SPDA, como a definição da área de proteção, a seleção dos materiais e equipamentos, a disposição dos condutores, as conexões e aterramentos, entre outros. Além disso, a NBR5419 estabelece as inspeções e manutenções necessárias para garantir o funcionamento correto do sistema.

É importante destacar que a instalação de um SPDA é essencial em locais com alta incidência de raios, e sua observância é obrigatória para edificações públicas e privadas com altura superior a 15 metros, segundo a legislação brasileira. A norma NBR-5419 é uma referência obrigatória para os profissionais responsáveis pelo projeto e instalação de sistemas de proteção contra descargas atmosféricas, como engenheiros, eletricistas e projetistas.

Portanto, a NBR-5419 é uma norma técnica fundamental para garantir a segurança das edificações, pessoas e equipamentos contra os efeitos das descargas elétricas atmosféricas. A observância da norma é fundamental para evitar riscos à vida e ao patrimônio, bem como para cumprir a legislação vigente.

\subsection{NORMA NBR-5444}

A NBR-5444 é uma norma técnica que define os símbolos gráficos utilizados em projetos de instalações elétricas prediais. Esses símbolos são padronizados e devem ser utilizados de acordo com as especificações da norma, para garantir uniformidade e clareza na representação das instalações elétricas.

A norma estabelece uma lista de símbolos gráficos que devem ser utilizados para representar os diversos elementos presentes em uma instalação elétrica predial, como tomadas, interruptores, disjuntores, luminárias, entre outros. Cada símbolo tem uma representação gráfica específica, com dimensões e formas padronizadas, permitindo que os projetistas e profissionais envolvidos na instalação elétrica tenham uma visão clara e precisa do projeto.

Além dos símbolos gráficos, a NBR-5444 também estabelece outras regras para a representação gráfica em desenho técnico de instalações elétricas prediais, como as linhas e os tipos de traços que devem ser utilizados, as dimensões das letras e dos números, as escalas, entre outros aspectos.

A padronização dos símbolos gráficos em projetos de instalações elétricas prediais é fundamental para garantir a segurança e a eficiência do sistema elétrico. Com a utilização de símbolos padronizados, é possível minimizar erros de interpretação e falhas na execução da instalação elétrica, evitando acidentes e problemas na utilização do sistema.

Em resumo, a NBR-5444 é uma norma técnica importante para garantir a padronização e a clareza na representação gráfica de projetos de instalações elétricas prediais, permitindo uma comunicação mais eficiente entre os profissionais envolvidos na construção e manutenção das edificações.

\subsection{NORMA NBR-14136}

A ABNT NBR-14136 é uma norma técnica brasileira que estabelece os requisitos de segurança para plugues e tomadas de uso doméstico e análogo, com tensões nominais de até 250 V e correntes nominais de até 20 A. Essa norma tem como objetivo garantir a segurança do usuário em relação a possíveis riscos elétricos, tais como choques elétricos e incêndios.

A norma NBR-14136 estabelece as dimensões, características construtivas, requisitos de segurança, marcações e ensaios que as tomadas e plugues devem atender. Ela determina, por exemplo, que as tomadas devem ser fabricadas com materiais resistentes a altas temperaturas e que os plugues devem ter proteção para evitar o contato acidental com as partes energizadas.

Essa norma é de grande importância para a indústria de equipamentos elétricos, fabricantes de plugues e tomadas, empresas de construção civil e para os consumidores finais, pois ela estabelece um padrão de segurança que deve ser observado na fabricação, instalação e uso desses equipamentos elétricos.

\subsection{NORMA NBR-7488}

A norma NBR-7288 estabelece os requisitos para cabos de potência isolados, com tensão nominal de 1 kV a 6 kV. A norma tem como objetivo garantir a segurança e a qualidade dos cabos de potência, estabelecendo critérios para a seleção dos materiais utilizados na fabricação, o dimensionamento dos condutores, o isolamento elétrico e a montagem dos cabos.

A norma define os critérios para a escolha do tipo de cabo de acordo com a sua aplicação, como por exemplo, para transmissão, distribuição ou uso em instalações fixas. Também são definidos os requisitos para a isolação dos cabos, que deve ser capaz de suportar as condições de operação, como temperatura, pressão e umidade.

A NBR-7288 também define as características dos cabos de potência, como as dimensões dos condutores, a resistência elétrica, a espessura do isolamento, a resistência mecânica e a flexibilidade dos cabos. Além disso, a norma estabelece os ensaios que devem ser realizados para avaliar a qualidade dos cabos, como ensaios elétricos, mecânicos e de envelhecimento.

É importante destacar que o cumprimento da NBR-7288 é obrigatório para a fabricação e instalação de cabos de potência isolados no Brasil. A norma é fundamental para garantir a segurança das instalações elétricas, reduzir o risco de falhas e assegurar a qualidade dos cabos utilizados em sistemas de transmissão e distribuição de energia elétrica.

\subsection{NORMA NBR-5111}

A NBR-5111 é uma norma técnica brasileira que estabelece as especificações para cabos de cobre nus, isolados e cobertos utilizados em instalações elétricas. A norma define os requisitos para as características físicas, mecânicas e elétricas dos cabos, bem como os ensaios que devem ser realizados para avaliar sua qualidade.

A NBR-5111 se aplica a cabos de cobre utilizados em baixa, média e alta tensão, em instalações fixas ou móveis. A norma estabelece as características dos cabos, como a composição química do cobre, as dimensões dos condutores, a resistência elétrica, a espessura do isolamento, a resistência mecânica e a flexibilidade dos cabos.

A norma também estabelece os ensaios que devem ser realizados para avaliar a qualidade dos cabos, como ensaios elétricos, mecânicos e de envelhecimento, que devem ser realizados em laboratórios acreditados pelo INMETRO.

O cumprimento da NBR-5111 é obrigatório para a fabricação e instalação de cabos de cobre no Brasil. A norma é fundamental para garantir a segurança das instalações elétricas, reduzir o risco de falhas e assegurar a qualidade dos cabos utilizados em sistemas de transmissão e distribuição de energia elétrica.

\subsection{NORMA TÉCNICA COPEL - NTC 900100}

A norma NTC 900100 é uma especificação técnica da Companhia Paranaense de Energia (Copel) que define as condições exigidas para a entrada de serviço de energia elétrica em empreendimentos. Ela estabelece os requisitos técnicos para o projeto, construção e montagem das instalações elétricas, incluindo a entrada de energia, o quadro geral de baixa tensão, a proteção contra surtos e curtos-circuitos, dentre outros aspectos relevantes.

A NTC 900100 é aplicável a todos os clientes da Copel que necessitam de uma nova entrada de serviço, além de servir como referência para projetistas, instaladores, construtores e outros profissionais envolvidos na construção de empreendimentos elétricos. É importante ressaltar que essa norma é específica da Copel e não necessariamente se aplica a outras concessionárias de energia elétrica.

\subsection{NORMA TÉCNICA COPEL - NTC 901100}

A norma técnica NTC 901100 estabelece as condições técnicas mínimas para o fornecimento de energia elétrica em tensão secundária em redes de distribuição de energia elétrica. Ela é aplicável a todos os clientes da Companhia Paranaense de Energia (Copel) que utilizam essa tensão de fornecimento.

Entre os aspectos regulados pela NTC 901100 estão: as características da tensão fornecida, os limites de variação da tensão, a proteção contra sobrecarga e curto-circuito, as condições para o fornecimento de energia em emergências, a instalação e a manutenção dos equipamentos elétricos dos clientes, dentre outros.

A NTC 901100 tem como objetivo garantir a qualidade e a continuidade do fornecimento de energia elétrica em tensão secundária, buscando sempre a satisfação do cliente e a segurança das instalações elétricas. É importante ressaltar que essa norma é específica da Copel e não necessariamente se aplica a outras concessionárias de energia elétrica.


Além dessas normas, existem diversas outras normas e regulamentações específicas para cada tipo de instalação elétrica, como as normas da ANEEL (Agência Nacional de Energia Elétrica) e as normas regulamentadoras do Ministério do Trabalho e Emprego (MTE).

Portanto, é fundamental que os profissionais responsáveis pelo projeto de instalações elétricas conheçam e observem as normas técnicas aplicáveis, a fim de garantir a segurança, a qualidade e a conformidade do sistema com as exigências legais.

\subsection{NORMA REGULAMENTADORA 10 - NR-10}

A Norma Regulamentadora NR-10 é uma norma técnica que estabelece os requisitos e as medidas de controle necessárias para garantir a segurança e a saúde dos trabalhadores que atuam em instalações elétricas. A NR-10 é de cumprimento obrigatório para todas as empresas que possuam empregados que trabalhem diretamente com eletricidade, seja na geração, transmissão, distribuição ou consumo.

A NR-10 tem como objetivo principal garantir a segurança dos trabalhadores que atuam com eletricidade, prevenindo acidentes e doenças ocupacionais decorrentes da exposição a riscos elétricos. A norma estabelece uma série de requisitos e medidas de controle, que incluem desde a elaboração de procedimentos de segurança até a utilização de equipamentos de proteção individual e coletiva.

Algumas das principais exigências da NR-10 incluem a realização de treinamentos para os trabalhadores que atuam em instalações elétricas, a elaboração de um prontuário de instalações elétricas, a realização de inspeções periódicas nas instalações elétricas e a adoção de medidas de segurança para a prevenção de acidentes, como o uso de equipamentos de proteção individual e coletiva.

Além disso, a NR-10 estabelece os limites de tensão e corrente elétrica para trabalhos em instalações elétricas, bem como as zonas de risco elétrico e as medidas de segurança necessárias para cada zona.

Em resumo, a NR-10 é uma norma técnica fundamental para garantir a segurança dos trabalhadores que atuam com eletricidade, estabelecendo requisitos e medidas de controle necessárias para prevenir acidentes e doenças ocupacionais decorrentes da exposição a riscos elétricos. O cumprimento da NR-10 é de suma importância para as empresas que atuam com eletricidade, garantindo a segurança e a saúde dos trabalhadores e a conformidade com as normas regulamentadoras do Ministério do Trabalho.

\section{CARACTERÍSTICAS DA EDIFICAÇÃO}
\begin{itemize}
	\item \textbf{\textit{Finalidade:}} Residência;
	\item \textbf{\textit{Paredes:}} Alvenaria;
	\item \textbf{\textit{Tipo de instalação:}} Baixa tensão;
	\item \textbf{\textit{Área total:}} $282$ $m^2$;
	\item \textbf{\textit{Número de pavimentos:}} $2$;
	\item \textbf{\textit{Número de unidades consumidoras:}} 1;
	\item \textbf{\textit{Tensão nominal:}} $127/220$ $V$;
\end{itemize}

\section{INSTALAÇÕES ELÉTRICAS}

\subsection{ENTRADA DE SERVIÇO}

O fornecimento de energia elétrica pela concessionária COPEL será realizada por uma entrada Categoria 41 (NTC 901100), do tipo trifásico, com 4 condutores, sendo 3 fases e 1 neutro de $25$ $mm^2$ (fase e neutro) EPR/XLPE 90°C e fio de proteção (PE) de $16$ $mm^2$ (aterramento) EPR/XLPE 90°C que serão conectados ao quadro de medição localizado próximo ao muro da residência, de modo a permitir fácil acesso ao leitor contratado pela concessionária e evitando problemas futuros para o cliente.

A especificação da entrada de serviço pode ser consultada TABELA 2 da NTC901100, item 9,2, referência ao ítem 4.3, na página 42, sendo a referida tabela mostrada na \autoref{fig:tabela2_ntc901100}:

\pagebreak

\figura
{ENTRADAS DE SERVIÇO - NTC901100} % Legenda
{.8} % % da largura da área de texto
{imagens/tabela2_ntc901100} % localização da figura
{Página 34 da NTC901100} % fonte da figura
{tabela2_ntc901100} % etiqueta
{}
{}

A entrada de serviço fica assim definida:

\begin{itemize}
	\item \textit{Tipo de entrada:} Categoria 41 COPEL, segundo a tabela 2 e a demanda calculada ($34.097,5$ $VA$);
	\item \textit{Ramal de Entrada:} tipo trifásica, categoria 41, com disjuntor trifásico de $100$ $A$ e Medidor Trifásico a 4 fios tipo T;
	\item \textit{Fios de fase/neutro:} cabos de cobre com seção nominal de $25$ $mm^2$ com isolamento EPR/XLPE 90°C;
	\item \textit{Fio de Proteção (PE):} cabo de cobre com seção nominal de $16$ $mm^2$ com isolamento EPR/XLPE 90°C;
	\item \textit{Modo de Instalação:} Maneira ``D'' (Subterrânea);
	\item \textit{Eletrodutos: } eletroduto corrugado com diâmetro de 40mm para o circuito 3F+N e um eletroduto corrugado com 20mm de diâmetro para o condutor de proteção.
\end{itemize}

Notas (NTC 901100): 

\begin{itemize}

	\item Maneira de instalar

	\subitem  \textit{\textbf{Maneira de instalar D:}} cabos unipolares ou cabo multipolar em eletroduto enterrado no solo (com proteção mecânica e/ou química adicional – popularmente cabo 1 kV).
	
	\item As dimensões estabelecidas na tabela para condutores e eletrodutos são mínimas.

	\item Para o ramal de entrada, a seção nominal do condutor neutro deve ser igual ao do(s) condutor (es) da(s) fase(s).

	\item Cada eletroduto deverá possuir um circuito completo [fase(s) e 1 neutro].

	\item Medidor:

	\subitem Medidor Trifásico 4 fios 127/220 V = \textit{Medidor T}
	
\end{itemize}


\subsubsection{ATERRAMENTO}

O aterramento junto ao poste da concessionária será feito através de uma haste de aterramento de $\frac{1}{2} \inches$ ($12,8$ $mm$) e $2,40$ $m$ de comprimento.

A instalação da haste de aterramento deverá ser feita segundo a norma NTC 901100 da COPEL, conforme mostra a \autoref{fig:aterramento_copel}:
\pagebreak 
\figura
{ALTERNATIVA DE ELETRODO - FIGURA 15 DA NTC901100} % Legenda
{.8} % % da largura da área de texto
{imagens/aterramento_copel} % localização da figura
{Página 53 da NTC901100} % fonte da figura
{aterramento_copel} % etiqueta
{}
{}


Notas (NTC 901100):

\begin{itemize}
	\item A instalação do ramal multiplexado (entrada da casa) será subterrânea, ou a maneira ``D'' segundo as notas da tabela 2 da NTC901100:
	
	\item Como alternativa a utilização aos conectores ``GAR'' ou Tipo Parafuso, poderá ser utilizada conexão com solda exotérmica ou conector cunha.
	
	\item A utilização de caixa para a haste de aterramento não é obrigatória.
	
	\item Não será permitida a utilização de conector tipo ``Terminal de Bateria'', conforme prescreve a NTC 927105.
	
\end{itemize}

O aterramento dos quadros de distribuição e seus respectivos barramentos de terra serão feitos utilizxanbdo esquema TN-S, conforme mostra a figura \autoref{fig:esquema_tns}:


\figura
{ESQUEMA TN-S} % Legenda
{.8} % % da largura da área de texto
{imagens/esquema_tns} % localização da figura
{Mundo da Elétrica} % fonte da figura
{esquema_tns} % etiqueta
{}
{}

Neste esquema, os condutores de neutro e proteção (N e PE) são separados, sendo unidos somente na haste de aterramento. O condutor de proteção que sai da haste de aterramento e vai até os quadros de distribuição tem seção nominal de $16$ $mm^2$.

\subsection{QUADROS DE DISTRIBUIÇÃO}

Os Quadro Geral e Auxiliar de Distribuição (QDC1 e QDC2) seguirão os padrões DIN/IEC e NEMA/UL, com disjuntores gerais para cada quadro, de acordo com o projeto. Além disso, os disjuntores para alimentação dos circuitos específicos e os interruptores diferenciais residuais (DR) serão instalados nesses quadros, conforme o diagrama unifilar.

Os disjuntores utilizados nos quadros de distribuição seguem o padrão DIN/IEC e são fornecidos por marcas como STECK, ABB, WEG ou similares, dispostos de acordo com o Diagrama Unifilar e respeitando o balanceamento de fases. A capacidade de condução de corrente dos barramentos também é indicada nos Quadros de Carga em planta. O Quadro de Distribuição será identificado de forma definitiva e duradoura em plaqueta acrílica individual e resinada, com a relação dos circuitos e seus equipamentos correspondentes, e todos os circuitos serão identificados nos quadros com etiquetas e anilhas plásticas. A entrada de energia nos quadros será comandada e protegida por disjuntores, com módulos de reserva para futuras ampliações. Todos os materiais utilizados serão de qualidade e procedência confiável.

De acordo com o item 6.5.4.10 da NBR-5410 \textit{``Os quadros de distribuição destinados a instalações residenciais e análogas devem ser entregues com a seguinte a advertência"} - mostrada na \autoref{fig:advertencia_nbr_5410_item_65410}:

\figura
{Advertência dos Quadros de Distribuição - NBR-5410} % Legenda
{.8} % % da largura da área de texto
{imagens/advertencia_nbr} % localização da figura
{Atom Elétrica} % fonte da figura
{advertencia_nbr_5410_item_65410} % etiqueta
{}
{}

Ainda, o item 6.5.4.11 da NBR-5410 diz que \textit{``A advertência de que trata 6.5.4.10 pode vir de fábrica ou ser provida no local, antes de a instalação ser entregue ao usuário, e não deve ser facilmente removível.''}

No caso de algum disjuntor não poder ser desligado sem aviso prévio aos usuários de equipamentos específicos, é necessário que o disjuntor seja equipado com um acessório apropriado ou algum tipo de sinalização que permita seu funcionamento normal. O uso de fitas adesivas não deve ser utilizado sob nenhuma circunstância. É importante ressaltar que apenas eletricistas qualificados devem ter acesso aos painéis.

\subsection{ILUMINAÇÃO}

Os circuitos de iluminação seguirão as especificações do projeto elétrico e serão derivados dos quadros de distribuição com fiação mínima de $1,5$ $mm^2$. As luminárias internas para área de convivência serão do tipo tubular de LED de $20W$ em chapa de aço galvanizada e pintada na cor branca, com refletor parabólico em alumínio anodizado de alta pureza e refletância. Já para as áreas de guarita, copa, banheiros e recepção serão utilizados plafons de plástico de sobrepor com lâmpadas LED de $9W$ e $12W$.

As caixas embutidas para interruptores seguirão dimensões padronizadas ($4\inches x 2 \inches$, $3 \inches x 3 \inches$ ou $4 \inches x 4 \inches$) e as luminárias serão instaladas em caixas embutidas tipo arandelas nas paredes a $2,20m$ do piso acabado, em caixas embutidas no forro para iluminação interna, e em caixas de ligação à prova de tempo para iluminação externa. As caixas de embutir em ambiente externo deverão ter apenas o olhal superior aberto para conexão com o eletroduto, a fim de evitar entrada de água e corpos estranhos na caixa. E nas caixas internas, apenas os olhais das caixas onde forem introduzidos eletrodutos serão abertos, devendo estar alinhadas e aprumadas.

\subsection{TOMADAS DE USO GERAL}


Tomadas de uso geral são aquelas que podem ser utilizadas para conectar equipamentos elétricos de diferentes tipos, como eletrodomésticos, ferramentas elétricas, aparelhos de som, computadores, entre outros. São pontos de energia elétrica previstos para atender às necessidades de uso comum em ambientes residenciais, comerciais ou industriais. A norma NBR-5410 estabelece as diretrizes para previsão de carga e número mínimo de tomadas de uso geral em diferentes tipos de ambientes. O objetivo é garantir a segurança e a eficiência da instalação elétrica, evitando sobrecarga e riscos de acidentes elétricos.

As tomadas de uso geral deste projeto serão alimentadas a partir dos quadros de distribuição correspondentes, de acordo com as normas e especificações aplicáveis. É exigido que todas as tomadas sejam aterradas, com pino de ligação a terra no padrão Brasileiro de conectores, seguindo as normas de segurança vigentes. As tomadas de uso geral serão projetadas em cada ambiente, próximas à porta de entrada e sob o interruptor da iluminação, em conformidade com as normas e diretrizes do projeto elétrico. As caixas para tomadas deverão ter dimensões padronizadas ($4 \inches x 2 \inches$ ou $4 \inches x 4 \inches$), de forma a permitir a instalação dos módulos previstos e garantir a funcionalidade e segurança do sistema. Todas as tomadas de uso geral deverão ser dotadas de conector de aterramento (PE), conforme ABNT \textit{NBR-14136}, com indicação diferenciada de tensão de trabalho. As tomadas de energia elétrica poderão ser de instalação embutida caixa $4 \inches x 2 \inches$, quando destinadas a uma única tomada, e em caixa $4 \inches x4 \inches$ quando se destinarem a mais de uma tomada. Ademais, é exigido que todas as tomadas sejam providas de fio-terra, conforme normas e especificações técnicas aplicáveis.

As tomadas de energia elétrica a serem utilizadas serão do tipo 2P + T, com capacidade para 10A/250V e serão embutidas na alvenaria de acordo com a altura indicada no projeto. A instalação das tomadas deverá seguir a polarização especificada, garantindo a correta conexão dos polos fase, neutro e terra, confirme a \autoref{fig:fig_tomada}:

\figura
{Polaridade das Tomadas - NBR-14136} % Legenda
{.8} % % da largura da área de texto
{imagens/polaridade_tomadas} % localização da figura
{Sala da Elétrica} % fonte da figura
{fig_tomada} % etiqueta
{}
{}

\subsection{TOMADAS DE USO ESPECÍFICO}
De acordo com a NBR5410, tomada de uso específico é aquela destinada a um único equipamento elétrico, com características especiais de plugue e tensão. São exemplos de equipamentos que necessitam de tomadas de uso específico: ar-condicionado, forno elétrico, máquina de lavar, secadora de roupas, entre outros. A norma estabelece requisitos específicos para as tomadas de uso específico, como a necessidade de serem identificadas com a informação do equipamento que deve ser conectado, além de exigir a instalação de dispositivos de proteção contra choques elétricos, como o DR. É importante ressaltar que as tomadas de uso específico devem atender aos requisitos de segurança estabelecidos na NBR-14136, que regulamenta as tomadas de uso geral e de uso específico.

As tomadas de uso específico para chuveiros e equipamentos com potência superior a $3000W$ serão constituídas por fios de 2 fases, neutro e terra saindo do Quadro de Distribuição corespondente, sendo a seção do aterramento o mesmo dos condutores carregados desse circuito, segundo as definições da NBR-5410. 

As tomadas de uso específico destinadas a equipamentos da cozinha e lavanderia serão alimentadas por fios de fase, neutro e terra saindo dos quadros de distribuição correspondente, sendo a seção do aterramento o mesmo dos condutores carregados desse circuito, segundo as definições da NBR-5410. As tomadas de energia elétrica a serem utilizadas serão do tipo 2P + T, com capacidade para 20A/250V e serão embutidas na alvenaria de acordo com a altura indicada no projeto.

Todas as tomadas de tensão nominal de $127V$, exceto as tomadas para chuveiros e equipamentos com potência superior a $3000W$, cuja tensão nominal é de $220V$.

\subsection{ELETRODUTOS}

Todos os circuitos serão instalados em eletrodutos de PVC corrugados, de cor amarela, com propriedades antichamas e antitóxicos, embutidos em paredes ou em lajes. Serão adotados eletrodutos de $1$ $\frac{1}{4} \inches$ quando não houver indicação de diâmetro externo. Os eletrodutos serão instalados de forma a formar uma rede contínua de caixa a caixa e de luminária a luminária, permitindo a remoção e transposição dos condutores sem prejuízo para o isolamento. Os eletrodutos utilizados para ligação das luminárias aos interruptores seguirão o mesmo padrão dos demais eletrodutos.

As caixas de passagem e eletrodutos deverão formar uma rede rígida. As conexões serão feitas com luvas roscadas, sem ângulos superiores a 90 graus em uma única curva. As fixações em caixas metálicas exigirão buchas e arruelas, sendo que as tubulações vazias deverão ter guias de arame para facilitar a enfiação. Tampões deverão ser colocados nos eletrodutos logo após a instalação para impedir a entrada de objetos estranhos.


\subsection{CAIXAS DE PASSAGEM}

As caixas de passagem deverão ser padronizadas nas medidas de $4\inches x 2 \inches$, $3 \inches x 3 \inches$ ou $4 \inches x 4 \inches$, feitas de PVC para embutir em alvenaria.

\subsection{CONDUTORES}

Todos os condutores devem ser cabos isolados, exceto se indicado de outra forma, e devem possuir características especiais de propagação e autoextinção do fogo. Os cabos utilizados para alimentar a iluminação interna/externa e as tomadas devem ser do tipo cabo com isolamento para 450/750 V, com isolamento simples e das marcas Ficap, Pirelli ou similares, conforme a norma NBR-7288, com a bitola especificada em planta. É proibido realizar emendas nos cabos e todas as caixas de passagem são destinadas a facilitar a passagem dos cabos.

Os cabos utilizados para alimentar os quadros de distribuição devem ser unipolares de cobre, com capacidade de 0,6/1kV e isolamento de EPR/XLPE 90°C. As seções dos condutores estão indicadas nos quadros de carga e diagramas. Todos os cabos devem ser do tipo cabo e devem possuir as características especificadas:

\begin{itemize}
	\item O condutor é composto por fio de cobre nu, de têmpera mole, com encordoamento classe 2;
	\item  A isolação é feita de composto termofixo de Polietileno reticulado EPR/XLPE, com espessura reforçada e é anti-chama, sem capa de chumbo;
	\item As temperaturas máximas suportadas pelo condutor são de 90°C em serviço contínuo, 130°C em sobrecarga e 250°C em curto circuito.
\end{itemize}

A instalação dos condutores só pode começar após a tubulação estar completamente instalada, fixada e limpa, após a primeira camada de tinta nas paredes e antes da camada final. Para facilitar a instalação dos cabos nas tubulações, apenas o uso de parafina ou talco é permitido.

Emendas são permitidas apenas dentro das caixas de passagem, e devem ser bem soldadas e isoladas com fita isolante antichama da 3M ou de marca similar, ou ainda serem feitas utilizando conectores de torção. Emendas dentro de eletrodutos não são permitidas em nenhuma circunstância.

Conectores terminais do tipo ilhós devem ser usados para conectar os cabos aos barramentos ou bornes das chaves e disjuntores, para bitolas acima de $6$ $mm^2$.

\subsection{CIRCUITOS}

Dentro de cada eletroduto, serão utilizados no máximo 3 (três) cabos para circuitos monofásicos + terra ou bifásicos + terra, e 5 cabos para circuitos trifásicos a 4 fios + terra, com até 3 (três) ou 4 (quatro) circuitos. A retirada da cobertura ou isolação dos eletrodutos sem consulta prévia ao projetista será vedada. Os circuitos alimentadores dos quadros de distribuição serão identificados em planta, ao longo dos eletrodutos. Equipamentos especiais, como chuveiros e torneiras elétricas, devem ser ligados diretamente ao quadro de distribuição específico, com um conduto único para cada circuito. As condensadoras de ar deverão ser ligadas diretamente ao quadro de distribuição, com no máximo dois circuitos por conduto. Os condutores não deverão sofrer esforços mecânicos incompatíveis.

\subsection{ATERRAMENTO}

Os circuitos de distribuição devem ser acompanhados por condutores de proteção (terra), de acordo com o projeto, e todos os quadros devem ter um barramento de terra. É estritamente proibido conectar os condutores neutro e de proteção (terra) nos quadros de Distribuição de cargas geral ou terminal. Além disso, todos os condutores de proteção (terra) devem estar isolados no interior dos eletrodutos.

O aterramento dos barramentos de terra dos quadros de distribuição serão feitos utilizando esquema TN-S, segundo a norma NBR5410 - utilizando o eletrodo de aterramento da entrada de serviço.

\subsection{OBSERVAÇÕES}

Caso o cliente deseje alterar qualquer item, como por exemplo uma luminária, é necessário verificar a potência do dispositivo a ser substituído. Se a nova potência for maior do que a anterior, é necessário recalcular o dimensionamento dos condutores e disjuntores.

\section{RESPONSABILIDADE TÉCNICA}

A responsabilidade técnica atribuída a este projeto está sujeita à manutenção de todas as características, definições e especificações dos dispositivos, equipamentos e materiais descritos neste projeto, que devem ser utilizados durante a sua execução. Além disso, qualquer alteração que se faça necessária deve ser avaliada e autorizada por escrito pelo responsável técnico do projeto.
\chapter[MEMORIAL DE CÁLCULO]{MEMORIAL DE CÁLCULO}

\section{CÁLCULOS DAS PREVISÕES DE CARGAS}

\subsection{CRITÉRIOS DA NBR-5410}
\label{criterios_carga_nbr}
\subsubsection{CÁLCULO DE CARGA DAS TOMADAS DE USO GERAL}

Para os banheiros, a NBR-5410 prevê pelo menos um ponto de tomada próximo ao lavatório, com cuidados para evitar riscos de choques elétricos.

Para varandas com área inferior a $2$ $m^2$ ou profundidade inferior a $0,80$ $m$, é necessário pelo menos um ponto de tomada, podendo ser instalado próximo ao acesso.

Em salas e dormitórios, deve ser previsto pelo menos um ponto de tomada para cada $5$ $m$, ou fração de perímetro, sendo recomendável equipar pontos com mais de uma tomada.

Para cozinhas, copas, copas-cozinhas, áreas de serviço, lavanderias e locais similares, é necessário pelo menos um ponto de tomada para cada $3,5$ $m$, ou fração de perímetro, sendo que acima da bancada da pia devem ser previstas no mínimo duas tomadas, no mesmo ponto ou em pontos distintos.

Para demais cômodos, o número mínimo de pontos de tomada é definido de acordo com a área, variando de um ponto para áreas de até $2,25$ $m^2$ até um ponto para cada $5$ $m$, ou fração de perímetro, para áreas acima de $6$ $m^2$.

A recomendação é incluir no projeto elétrico tomadas adicionais em locais estratégicos, devido à crescente utilização de aparelhos eletroeletrônicos no dia a dia, para evitar a necessidade de adaptadores, extensões e outros multiplicadores de tomadas.

Caso os locais, como banheiros, cozinhas, copas, copas-cozinhas, áreas de serviço, lavanderias ou locais análogos, tenham 3 ou mais tomadas, 3 delas devem ser instaladas com capacidade de pelo menos 600 VA, e as demais com capacidade de pelo menos 100 VA.

Já nos demais cômodos ou dependências, a norma exige que cada tomada tenha capacidade de pelo menos 100 VA.

\subsubsection{CÁLCULO DE CARGA DAS TOMADAS DE USO ESPECÍFICO}

A NBR-5410 não estabelece uma previsão de carga padrão para tomadas de uso específico, pois essas tomadas são destinadas a equipamentos elétricos específicos, cada um com sua demanda de carga elétrica particular. Dessa forma, a previsão de carga para tomadas de uso específico deve ser feita de acordo com as informações do fabricante do equipamento elétrico a ser conectado a ela. As informações sobre a carga elétrica requerida pelo equipamento podem ser encontradas no manual do usuário ou na placa de identificação do equipamento. A instalação elétrica deve ser dimensionada de forma a atender a demanda de carga de todos os equipamentos elétricos conectados, incluindo as tomadas de uso específico, garantindo assim a segurança e o bom funcionamento do sistema elétrico.

Desta forma, a previsão de carga das tomadas de uso específico é simplesmente a soma da potência instalada de cada equipamento previsto no projeto.

\subsubsection{CÁLCULO DE CARGA DE ILUMINAÇÃO}

A NBR-5410 estabelece critérios para a previsão de carga de iluminação em instalações elétricas residenciais. As diretrizes para a previsão de carga de iluminação são as seguintes:

\begin{itemize}
	\item Em cada cômodo ou dependência, deve ser previsto pelo menos um ponto de luz com potência mínima de $100$ $VA$;
	\item Em cômodos com área igual ou inferior a $6$ $m^2$, deve ser prevista pelo menos uma carga de $100$ $VA$;
	\item Em cômodos com área superior a $6$ $m^2$, a previsão de carga deve considerar $100$ $VA$ para os primeiros $6$ $m^2$, acrescidos de $60$ $VA$ para cada aumento de $4$ $m^2$ inteiros.
	
\end{itemize}

É importante destacar que esses valores são mínimos e que a previsão de carga de iluminação deve considerar a utilização prevista para cada ambiente, a fim de garantir que a capacidade do circuito seja suficiente para atender à demanda de carga elétrica requerida pelos pontos de iluminação, sem sobrecarregar o sistema elétrico e garantindo a segurança do usuário.

\subsection{CARGA INSTALADA}

A TABELA~\ref{tab:previsao_de_carga} mostra o levantamento dos pontos de iluminação, tomadas de uso geral e tomadas de uso específico, bem como suas cargas instaladas, de acordo com os critérios descritos na Seção~\ref{criterios_carga_nbr}:

\begin{table}[H]
	\centering
	\caption{CÁLCULO DE PONTOS E DE PREVISÃO DE CARGA}
	\label{tab:previsao_de_carga}
	\resizebox{\columnwidth}{!}{%
	\begin{tabular}{|l|l|ll|llll|lll|ll|}
		\hline
		\multirow{2}{*}{PAVIMENTO}          & \multirow{2}{*}{DEPENDÊNCIA} & \multicolumn{2}{l|}{DIMENSÕES}                                       & \multicolumn{4}{l|}{ILUMINAÇÃO}                                                                                                                        & \multicolumn{3}{l|}{TUG}                                                                                    & \multicolumn{2}{l|}{TUE}                             \\ \cline{3-13} 
		&                              & \multicolumn{1}{l|}{AREA ($m^2$)}              & PERÍMETRO (m)          & \multicolumn{1}{l|}{N. PONTOS}          & \multicolumn{1}{l|}{POT.UNIT}             & \multicolumn{1}{l|}{POT. UNIT. CALC}      & POT. TOTAL           & \multicolumn{1}{l|}{N. PONTOS}          & \multicolumn{1}{l|}{POT. UNIT}            & POT. TOTAL            & \multicolumn{1}{l|}{APARELHO}             & POTENCIA \\ \hline
		\multirow{17}{*}{TERREO}            & HALL ENT + EXTERNO           & \multicolumn{1}{l|}{1,001}                  & 4,1399                 & \multicolumn{1}{l|}{1}                  & \multicolumn{1}{l|}{100}                  & \multicolumn{1}{l|}{40}                   & 400                  & \multicolumn{1}{l|}{1}                  & \multicolumn{1}{l|}{100}                  & 100                   & \multicolumn{1}{l|}{}                     &          \\ \cline{2-13} 
		& SALA DE ESTAR                & \multicolumn{1}{l|}{25,39}                  & 20,71                  & \multicolumn{1}{l|}{4}                  & \multicolumn{1}{l|}{100}                  & \multicolumn{1}{l|}{340}                  & 400                  & \multicolumn{1}{l|}{5}                  & \multicolumn{1}{l|}{100}                  & 500                   & \multicolumn{1}{l|}{AR CONDICIONADO}      & 3600     \\ \cline{2-13} 
		& QUARTO 2                     & \multicolumn{1}{l|}{19,08}                  & 20                     & \multicolumn{1}{l|}{3}                  & \multicolumn{1}{l|}{100}                  & \multicolumn{1}{l|}{280}                  & 300                  & \multicolumn{1}{l|}{5}                  & \multicolumn{1}{l|}{100}                  & 500                   & \multicolumn{1}{l|}{AR CONDICIONADO}      & 3600     \\ \cline{2-13} 
		& \multirow{2}{*}{GARAGEM}     & \multicolumn{1}{l|}{\multirow{2}{*}{36,93}} & \multirow{2}{*}{29,6}  & \multicolumn{1}{l|}{\multirow{2}{*}{6}} & \multicolumn{1}{l|}{\multirow{2}{*}{100}} & \multicolumn{1}{l|}{\multirow{2}{*}{520}} & \multirow{2}{*}{600} & \multicolumn{1}{l|}{\multirow{2}{*}{6}} & \multicolumn{1}{l|}{\multirow{2}{*}{100}} & \multirow{2}{*}{600}  & \multicolumn{1}{l|}{MOTOR PORTÃO}         & 650      \\ \cline{12-13} 
		&                              & \multicolumn{1}{l|}{}                       &                        & \multicolumn{1}{l|}{}                   & \multicolumn{1}{l|}{}                     & \multicolumn{1}{l|}{}                     &                      & \multicolumn{1}{l|}{}                   & \multicolumn{1}{l|}{}                     &                       & \multicolumn{1}{l|}{TOMADA POTENCIA 220V} & 1000     \\ \cline{2-13} 
		& SALA DE JANTAR               & \multicolumn{1}{l|}{26,34}                  & 21,43                  & \multicolumn{1}{l|}{4}                  & \multicolumn{1}{l|}{100}                  & \multicolumn{1}{l|}{400}                  & 400                  & \multicolumn{1}{l|}{5}                  & \multicolumn{1}{l|}{100}                  & 500                   & \multicolumn{1}{l|}{}                     &          \\ \cline{2-13} 
		& \multirow{5}{*}{COZINHA}     & \multicolumn{1}{l|}{\multirow{5}{*}{9,49}}  & \multirow{5}{*}{12,65} & \multicolumn{1}{l|}{\multirow{5}{*}{1}} & \multicolumn{1}{l|}{\multirow{5}{*}{100}} & \multicolumn{1}{l|}{\multirow{5}{*}{100}} & \multirow{5}{*}{100} & \multicolumn{1}{l|}{\multirow{2}{*}{1}} & \multicolumn{1}{l|}{\multirow{2}{*}{}}    & \multirow{2}{*}{600}  & \multicolumn{1}{l|}{GRILL}                & 1200     \\ \cline{12-13} 
		&                              & \multicolumn{1}{l|}{}                       &                        & \multicolumn{1}{l|}{}                   & \multicolumn{1}{l|}{}                     & \multicolumn{1}{l|}{}                     &                      & \multicolumn{1}{l|}{}                   & \multicolumn{1}{l|}{}                     &                       & \multicolumn{1}{l|}{LAVA LOUÇA}           & 1500     \\ \cline{9-13} 
		&                              & \multicolumn{1}{l|}{}                       &                        & \multicolumn{1}{l|}{}                   & \multicolumn{1}{l|}{}                     & \multicolumn{1}{l|}{}                     &                      & \multicolumn{1}{l|}{\multirow{3}{*}{3}} & \multicolumn{1}{l|}{\multirow{3}{*}{}}    & \multirow{3}{*}{1350} & \multicolumn{1}{l|}{TORNEIRA ELETRICA}    & 4500     \\ \cline{12-13} 
		&                              & \multicolumn{1}{l|}{}                       &                        & \multicolumn{1}{l|}{}                   & \multicolumn{1}{l|}{}                     & \multicolumn{1}{l|}{}                     &                      & \multicolumn{1}{l|}{}                   & \multicolumn{1}{l|}{}                     &                       & \multicolumn{1}{l|}{FOGÃO ELÉTRICO}       & 6500     \\ \cline{12-13} 
		&                              & \multicolumn{1}{l|}{}                       &                        & \multicolumn{1}{l|}{}                   & \multicolumn{1}{l|}{}                     & \multicolumn{1}{l|}{}                     &                      & \multicolumn{1}{l|}{}                   & \multicolumn{1}{l|}{}                     &                       & \multicolumn{1}{l|}{FORNO ELÉTRICO}       & 6500     \\ \cline{2-13} 
		& BWC SOCIAL                   & \multicolumn{1}{l|}{2,23}                   & 6                      & \multicolumn{1}{l|}{1}                  & \multicolumn{1}{l|}{100}                  & \multicolumn{1}{l|}{100}                  & 100                  & \multicolumn{1}{l|}{1}                  & \multicolumn{1}{l|}{100}                  & 600                   & \multicolumn{1}{l|}{}                     &          \\ \cline{2-13} 
		& BWC Q2                       & \multicolumn{1}{l|}{4,12}                   & 8,8                    & \multicolumn{1}{l|}{1}                  & \multicolumn{1}{l|}{100}                  & \multicolumn{1}{l|}{100}                  & 100                  & \multicolumn{1}{l|}{1}                  & \multicolumn{1}{l|}{100}                  & 600                   & \multicolumn{1}{l|}{CHUVEIRO}             & 6500     \\ \cline{2-13} 
		& BWC Q1                       & \multicolumn{1}{l|}{4,02}                   & 8,1                    & \multicolumn{1}{l|}{1}                  & \multicolumn{1}{l|}{100}                  & \multicolumn{1}{l|}{100}                  & 100                  & \multicolumn{1}{l|}{1}                  & \multicolumn{1}{l|}{100}                  & 600                   & \multicolumn{1}{l|}{CHUVEIRO}             & 6500     \\ \cline{2-13} 
		& QUARTO 1                     & \multicolumn{1}{l|}{18,91}                  & 19,79                  & \multicolumn{1}{l|}{3}                  & \multicolumn{1}{l|}{100}                  & \multicolumn{1}{l|}{280}                  & 300                  & \multicolumn{1}{l|}{4}                  & \multicolumn{1}{l|}{100}                  & 400                   & \multicolumn{1}{l|}{AR CONDICIONADO}      & 3600     \\ \cline{2-13} 
		& JARDIM EXTERNO               & \multicolumn{1}{l|}{8,09}                   & 11,5                   & \multicolumn{1}{l|}{1}                  & \multicolumn{1}{l|}{100}                  & \multicolumn{1}{l|}{100}                  & 100                  & \multicolumn{1}{l|}{3}                  & \multicolumn{1}{l|}{100}                  & 300                   & \multicolumn{1}{l|}{}                     &          \\ \cline{2-13} 
		& JARDIM INTERNO               & \multicolumn{1}{l|}{12,64}                  & 17,49                  & \multicolumn{1}{l|}{2}                  & \multicolumn{1}{l|}{100}                  & \multicolumn{1}{l|}{160}                  & 200                  & \multicolumn{1}{l|}{4}                  & \multicolumn{1}{l|}{100}                  & 400                   & \multicolumn{1}{l|}{}                     &          \\ \hline
		\multirow{13}{*}{SEGUNDO PAVIMENTO} & SUITE                        & \multicolumn{1}{l|}{25,59}                  & 23,41                  & \multicolumn{1}{l|}{4}                  & \multicolumn{1}{l|}{100}                  & \multicolumn{1}{l|}{340}                  & 400                  & \multicolumn{1}{l|}{5}                  & \multicolumn{1}{l|}{100}                  & 500                   & \multicolumn{1}{l|}{AR CONDICIONADO}      & 3600     \\ \cline{2-13} 
		& BWC SUITE                    & \multicolumn{1}{l|}{3,12}                   & 7,4                    & \multicolumn{1}{l|}{1}                  & \multicolumn{1}{l|}{100}                  & \multicolumn{1}{l|}{100}                  & 100                  & \multicolumn{1}{l|}{1}                  & \multicolumn{1}{l|}{100}                  & 600                   & \multicolumn{1}{l|}{CHUVEIRO}             & 6500     \\ \cline{2-13} 
		& \multirow{2}{*}{LAVANDERIA}  & \multicolumn{1}{l|}{\multirow{2}{*}{9,8}}   & \multirow{2}{*}{13}    & \multicolumn{1}{l|}{\multirow{2}{*}{1}} & \multicolumn{1}{l|}{\multirow{2}{*}{100}} & \multicolumn{1}{l|}{\multirow{2}{*}{100}} & \multirow{2}{*}{100} & \multicolumn{1}{l|}{\multirow{2}{*}{4}} & \multicolumn{1}{l|}{}                     & 700                   & \multicolumn{1}{l|}{MÁQUINA LAVA E SECA}  & 3500     \\ \cline{10-13} 
		&                              & \multicolumn{1}{l|}{}                       &                        & \multicolumn{1}{l|}{}                   & \multicolumn{1}{l|}{}                     & \multicolumn{1}{l|}{}                     &                      & \multicolumn{1}{l|}{}                   & \multicolumn{1}{l|}{}                     & 1200                  & \multicolumn{1}{l|}{}                     &          \\ \cline{2-13} 
		& BWC SOCIAL                   & \multicolumn{1}{l|}{2,04}                   & 6,06                   & \multicolumn{1}{l|}{1}                  & \multicolumn{1}{l|}{100}                  & \multicolumn{1}{l|}{100}                  & 100                  & \multicolumn{1}{l|}{1}                  & \multicolumn{1}{l|}{100}                  & 600                   & \multicolumn{1}{l|}{}                     &          \\ \cline{2-13} 
		& CLOSET Q3                    & \multicolumn{1}{l|}{4,7}                    & 8,63                   & \multicolumn{1}{l|}{1}                  & \multicolumn{1}{l|}{100}                  & \multicolumn{1}{l|}{100}                  & 100                  & \multicolumn{1}{l|}{2}                  & \multicolumn{1}{l|}{100}                  & 200                   & \multicolumn{1}{l|}{}                     &          \\ \cline{2-13} 
		& QUARTO 3                     & \multicolumn{1}{l|}{19,48}                  & 20,8                   & \multicolumn{1}{l|}{3}                  & \multicolumn{1}{l|}{100}                  & \multicolumn{1}{l|}{280}                  & 300                  & \multicolumn{1}{l|}{5}                  & \multicolumn{1}{l|}{100}                  & 500                   & \multicolumn{1}{l|}{AR CONDICIONADO}      & 3600     \\ \cline{2-13} 
		& BWC Q3                       & \multicolumn{1}{l|}{3,97}                   & 8,7                    & \multicolumn{1}{l|}{1}                  & \multicolumn{1}{l|}{100}                  & \multicolumn{1}{l|}{100}                  & 100                  & \multicolumn{1}{l|}{1}                  & \multicolumn{1}{l|}{100}                  & 600                   & \multicolumn{1}{l|}{CHUVEIRO}             & 6500     \\ \cline{2-13} 
		& TERRAÇO                      & \multicolumn{1}{l|}{16,18}                  & 16,28                  & \multicolumn{1}{l|}{2}                  & \multicolumn{1}{l|}{100}                  & \multicolumn{1}{l|}{220}                  & 200                  & \multicolumn{1}{l|}{2}                  & \multicolumn{1}{l|}{100}                  & 200                   & \multicolumn{1}{l|}{}                     &          \\ \cline{2-13} 
		& SALA INTIMA                  & \multicolumn{1}{l|}{19,58}                  & 18,02                  & \multicolumn{1}{l|}{3}                  & \multicolumn{1}{l|}{100}                  & \multicolumn{1}{l|}{280}                  & 300                  & \multicolumn{1}{l|}{4}                  & \multicolumn{1}{l|}{100}                  & 400                   & \multicolumn{1}{l|}{AR CONDICIONADO}      & 3600     \\ \cline{2-13} 
		& CORREDOR SUITE               & \multicolumn{1}{l|}{1,71}                   & 5,5                    & \multicolumn{1}{l|}{1}                  & \multicolumn{1}{l|}{100}                  & \multicolumn{1}{l|}{40}                   & 100                  & \multicolumn{1}{l|}{1}                  & \multicolumn{1}{l|}{100}                  & 100                   & \multicolumn{1}{l|}{}                     &          \\ \cline{2-13} 
		& HALL                         & \multicolumn{1}{l|}{1,85}                   & 5,65                   & \multicolumn{1}{l|}{1}                  & \multicolumn{1}{l|}{100}                  & \multicolumn{1}{l|}{40}                   & 100                  & \multicolumn{1}{l|}{1}                  & \multicolumn{1}{l|}{100}                  & 100                   & \multicolumn{1}{l|}{}                     &          \\ \cline{2-13} 
		& ESCADA                       & \multicolumn{1}{l|}{4,9}                    & 8,9                    & \multicolumn{1}{l|}{2}                  & \multicolumn{1}{l|}{100}                  & \multicolumn{1}{l|}{100}                  & 200                  & \multicolumn{3}{l|}{NORMA NÃO PREVÊ}                                           & \multicolumn{1}{l|}{}                     &          \\ \hline
		& TOTAL                        & \multicolumn{1}{l|}{}                       &                        & \multicolumn{1}{l|}{48}                 & \multicolumn{1}{l|}{}                     & \multicolumn{1}{l|}{}                     & 4800                 & \multicolumn{1}{l|}{66}                 & \multicolumn{1}{l|}{}                     & 12650                 & \multicolumn{1}{l|}{}                     & 72950    \\ \hline
	\end{tabular}}
\end{table}

A previsão de carga é de $92.400$ $VA$, segundo a soma total das cargas instaladas.

\section{DETERMINAÇÃO DA PROVÁVEL DEMANDA}

\subsection{MÉTODOS DE CÁLCULO}

\subsubsection{MÉTODO COBEI}
O Método COBEI é uma forma de calcular a demanda de potência elétrica para um determinado estabelecimento ou edifício. A sigla COBEI significa "Comitê Brasileiro de Eletricidade, Eletrônica, Iluminação e Telecomunicações", e esse método é baseado em normas técnicas brasileiras.

A demanda de potência elétrica é a quantidade de energia que é requerida pelo estabelecimento em um determinado momento. Essa demanda é influenciada pelo número de aparelhos eletrodomésticos e eletrônicos em uso, pelo tipo de atividade realizada no local, entre outros fatores.

O Método COBEI leva em consideração a potência nominal dos equipamentos elétricos que serão utilizados no local, além de fatores como o fator de simultaneidade (ou seja, a probabilidade de que todos os equipamentos elétricos estejam em uso ao mesmo tempo), o fator de demanda (que é a fração da carga máxima esperada) e a previsão de crescimento da demanda futura.

Com base nesses dados, o Método COBEI permite calcular a demanda de potência elétrica para o estabelecimento, o que é importante para a escolha adequada do tamanho do transformador e do quadro de distribuição de energia elétrica.

\subsubsection{MÉTODO CEMIG}

O método CEMIG é um método utilizado para o cálculo da demanda de potência elétrica em edificações. Ele foi desenvolvido pela Companhia Energética de Minas Gerais (CEMIG) e leva em consideração diversos fatores, como o número e a potência dos equipamentos instalados, o tipo de atividade desenvolvida na edificação, as características da iluminação e o tipo de climatização utilizado, entre outros. A partir dessas informações, é possível estimar a demanda de potência elétrica máxima que será requerida pela edificação em determinado momento, o que é importante para a adequação do sistema elétrico e a definição da tarifa de energia elétrica a ser aplicada.

\subsection{PROVÁVEL DEMANDA}

Para cálculo da demanda, utiliza-se dois métodos, um para iluminação e tomadas de uso geral e um diferente para cálculo da demanda das tomadas de uso específico. O método COBEI é bastante prático para cálculos de demanda de tomadas de uso geral e iluminação, pois considera somente a carga instalada para a determinação da demanda. Porém, para as tomadas de uso especifico, o método COBEI considera demanda de 100\%. Portanto, o método CEMIG é utilizado apenas para as tomadas de uso específico, pois considera .

Segundo o método COBEI, o fator de demanda de tomadas de uso geral e iluminação é obtido da tabela mostrada na \autoref{fig:demanda_cobei}:

\figura
{FATORES DE DEMANDA SEGUNDO A POTÊNCIA INSTALADA - CT-64/COBEI} % Legenda
{.8} % % da largura da área de texto
{imagens/demanda_cobei} % localização da figura
{CT-64/COBEI} % fonte da figura
{demanda_cobei} % etiqueta
{}
{}

Segundo o método CEMIG, o fator de demanda para tomadas de uso específico é obtida da tabela mostrada na \autoref{fig:demanda_cemig}:

\figura
{FATORES DE DEMANDA PARA TOMADAS DE USO ESPECIFICO - MÉTODO CEMIG} % Legenda
{.8} % % da largura da área de texto
{imagens/demanda_cemig} % localização da figura
{ND 5.2 CEMIG} % fonte da figura
{demanda_cemig} % etiqueta
{}
{}

A potência instalada para iluminação e tomadas de uso geral é de $17.450$ $VA$ ($P_{COB}$), portanto, segundo a \autoref{fig:demanda_cobei}, o fator de demanda ($F_{COB}$) é $0,24$;

A potência instalada para as tomads de uso geral é de $72.950$ $VA$ ($P_{CEMIG}$), sendo o número de tomadas igual a 18. Portanto, a \autoref{fig:demanda_cemig} indica um fator de demanda ($F_{CEMIG}$) de $0,41$.

A demanda total da edificação é a soma das demandas individuais:

\begin{equation}
	D = F_{COB} \times P_{COB} + F_{CEMIG} \times P_{CEMIG}
\end{equation}

Portanto, a demanda provável será igual a:

\begin{equation}
	\begin{split}
		D = 17.450 \times 0,24 + 72.950 \times 0,41\\
		D = 34.097,5 VA
	\end{split}
\end{equation}

Logo, a demanda provável da edificação é de $34.097,5$ $VA$

A demanda provável é importante para a escolha da entrada de serviço. Segundo a tabela 2 da norma NTC 901100, o padrão da entrada de serviço segue as especificações listadas abaixo:

\begin{itemize}
	\item \textbf{\textit{Categoria}}: 41
	\item \textbf{\textit{Demanda Máxima:}} $38$ $kVA$
	\item \textbf{\textit{Tipo de fornecimento:}} Trifásico a 4 fios.
	\item \textbf{\textit{Bitola do fio (Ramal de entrada):}} para as fases, $25$ $mm^2$ EPR/XLPE 90°C,  condutor de aterramento nu ou encapado de $16$ $mm^2$;
	\item \textbf{\textit{Poste: }}DAN 200
\end{itemize}

A entrada de serviço fica assim definida:

\begin{itemize}
	\item \textit{Tipo de entrada:} Categoria 41 COPEL, segundo a tabela 2 e a demanda requerida ($34.097,5$ $VA$);
	\item \textit{Ramal de Entrada:} tipo trifásica, categoria 41, com disjuntor trifásico de $100$ $A$ e Medidor Trifásico a 4 fios tipo T;
	\item \textit{Fios de fase/neutro:} cabos de cobre com seção nominal de $25$ $mm^2$ com isolamento EPR/XLPE 90°C;
	\item \textit{Fio de Proteção (PE):} cabo de cobre com seção nominal de $16$ $mm^2$ com isolamento EPR/XLPE 90°C;
	\item \textit{Modo de Instalação:} Maneira ``D'' (Subterrânea);
	\item \textit{Eletrodutos: } eletroduto corrugado com diâmetro de 40mm para o circuito 3F+N e um eletroduto corrugado com 20mm de diâmetro para o condutor de proteção.
\end{itemize}

\section{DIVISÃO DE CIRCUITOS}

A NBR-5410 define critérios para a divisão de circuitos em instalações elétricas de baixa tensão. Entre os critérios definidos, podemos citar:

\begin{itemize}
	\item A divisão dos circuitos deve ser feita de forma a evitar a sobrecarga dos condutores e dos dispositivos de proteção.
	
	\item A corrente de projeto dos circuitos deve ser escolhida com base na carga elétrica a ser alimentada.
	
	\item Os circuitos devem ser agrupados de acordo com a sua finalidade, podendo ser divididos em circuitos de iluminação, circuitos de tomadas, circuitos de equipamentos fixos, entre outros.
	
	\item Os circuitos de iluminação devem ser separados dos circuitos de tomadas, exceto em casos específicos, como em áreas de trabalho.
	
	\item Em áreas externas ou úmidas, os circuitos devem ser protegidos por dispositivos diferenciais-residuais (DR).
	
	\item Para circuitos de equipamentos fixos, devem ser consideradas as especificações do fabricante em relação à potência e corrente de partida.
	
	\item Em áreas com risco de explosão ou incêndio, devem ser utilizados dispositivos de proteção específicos.
	
\end{itemize}


A aplicação destes critérios gerou os circuitos de distribuição, dispostos na TABELA~\ref{tab:distribuicao_circuitos}, de acordo com os equipamentos elétricos instalados e determinação dos quadros de distribuição:

\begin{longtable}{|l|l|l|l|}
	\caption{DISTRIBUIÇÃO DE CIRCUITOS DE ACORDO COM O EQUIPAMENTO E QUADRO DE DISTRIBUIÇÃO}
	\label{tab:distribuicao_circuitos}\\
	\hline
	CIRCUITO & TIPO & QUADRO & TENSÃO \\ \hline
	\endfirsthead
	%
	\endhead
	%
	C12, C13 & AR CONDICIONADO & QDC1 & 220 \\ \hline
	C20, C21 & TORNEIRA ELETRICA & QDC1 & 220 \\ \hline
	C26, C27 & CHUVEIRO & QDC1 & 220 \\ \hline
	C28, C29 & CHUVEIRO & QDC1 & 220 \\ \hline
	C14, C15 & AR CONDICIONADO & QDC1 & 220 \\ \hline
	C16, C17 & AR CONDICIONADO & QDC1 & 220 \\ \hline
	C12, C13 & AR CONDICIONADO & QDC2 & 220 \\ \hline
	C18, C19 & CHUVEIRO & QDC2 & 220 \\ \hline
	C8, C9 & AR CONDICIONADO & QDC2 & 220 \\ \hline
	C16, C17 & CHUVEIRO & QDC2 & 220 \\ \hline
	C10, C11 & AR CONDICIONADO & QDC2 & 220 \\ \hline
	C18, C19 & \begin{tabular}[c]{@{}l@{}}MOTOR PORTÃO\\ TOMADA POTENCIA 220V\end{tabular} & QDC1 & 220 \\ \hline
	C22, C23 & FOGÃO ELÉTRICO & QDC1 & 220 \\ \hline
	C24, C25 & FORNO ELÉTRICO & QDC1 & 220 \\ \hline
	C14, C15 & MÁQUINA LAVA E SECA & QDC2 & 220 \\ \hline
	C4 & \begin{tabular}[c]{@{}l@{}}TUG \\ HALL ENTRADA\\ SALA DE ESTAR\\ BWC SOCIAL\end{tabular} & QDC1 & 127 \\ \hline
	C5 & \begin{tabular}[c]{@{}l@{}}TUG \\ QUARTO 2\\ BWC Q2\end{tabular} & QDC1 & 127 \\ \hline
	C6 & \begin{tabular}[c]{@{}l@{}}TUG\\ GARAGEM\\ JARDIM EXTERNO\end{tabular} & QDC1 & 127 \\ \hline
	C7 & \begin{tabular}[c]{@{}l@{}}TUG\\ SALA DE JANTAR\\ COZINHA\end{tabular} & QDC1 & 127 \\ \hline
	C8 & \begin{tabular}[c]{@{}l@{}}TUG\\ COZINHA GRILL\end{tabular} & QDC1 & 127 \\ \hline
	C9 & \begin{tabular}[c]{@{}l@{}}TUG\\ COZINHA LAVA LOUÇA\end{tabular} & QDC1 & 127 \\ \hline
	C10 & \begin{tabular}[c]{@{}l@{}}TUG\\ COZINHA\end{tabular} & QDC1 & 127 \\ \hline
	C11 & \begin{tabular}[c]{@{}l@{}}TUG\\ BWC Q1\\ QUARTO 1\\ JARDIM INTERNO\end{tabular} & QDC1 & 127 \\ \hline
	C3 & \begin{tabular}[c]{@{}l@{}}TUG\\ CLOSET Q3\\ QUARTO 3\\ BWC Q3\end{tabular} & QDC2 & 127 \\ \hline
	C4 & \begin{tabular}[c]{@{}l@{}}TUG\\ TERRAÇO\\ SALA INTIMA\\ BWC SOCIAL\end{tabular} & QDC2 & 127 \\ \hline
	C5 & \begin{tabular}[c]{@{}l@{}}TUG\\ LAVANDERIA\\ CORREDOR SUITE\\ HALL\end{tabular} & QDC2 & 127 \\ \hline
	C6 & \begin{tabular}[c]{@{}l@{}}TUG\\ LAVANDERIA\end{tabular} & QDC2 & 127 \\ \hline
	C7 & \begin{tabular}[c]{@{}l@{}}TUG\\ SUITE\\ BWC SUITE\end{tabular} & QDC2 & 127 \\ \hline
	C1 & \begin{tabular}[c]{@{}l@{}}ILUMINAÇÃO\\ HALL ENTRADA\\ SALA DE ESTAR\\ QUARTO 2\end{tabular} & QDC1 & 127 \\ \hline
	C2 & \begin{tabular}[c]{@{}l@{}}ILUMINAÇÃO\\ GARAGEM\\ SALA DE JANTAR\end{tabular} & QDC1 & 127 \\ \hline
	C3 & \begin{tabular}[c]{@{}l@{}}ILUMINAÇÃO\\ COZINHA\\ BWC SOCIAL\\ BWC Q2\\ BWC Q1\\ QUARTO 1\\ JARDIM EXTERNO\\ JARDIM INTERNO\end{tabular} & QDC1 & 127 \\ \hline
	C1 & \begin{tabular}[c]{@{}l@{}}ILUMINAÇÃO\\ CLOSET Q3\\ QUARTO 3\\ BWC Q3\\ TERRAÇO\\ SALA INTIMA\\ ESCADA\end{tabular} & QDC2 & 127 \\ \hline
	C2 & \begin{tabular}[c]{@{}l@{}}ILUMINAÇÃO\\ SUITE\\ LAVANDERIA\\ BWC SUITE\\ BWC SOCIAL\\ CORREDOR SUITE\\ HALL\end{tabular} & QDC2 & 127 \\ \hline
\end{longtable}


\section{BALANCEAMENTO DE CARGA}

O balanceamento de carga é feita utilizando o seguinte critério:

\begin{itemize}
	
	\item Distribuir os circuitos nos quadros de distribuição, de forma equilibrada, para evitar sobrecarga em uma ou mais fases da rede trifásica.
	
\end{itemize}

A aplicação deste critério gerou a distribuição de carga disposta na TABELA~\ref{tab:balanceamento_de_carga}:

\begin{longtable}{|l|l|l|l|l|l|}
	\caption{BALANCEAMENTO DE CARGA DE ACORDO COM O EQUIPAMENTO E POTÊNCIA INSTALADA}
	\label{tab:balanceamento_de_carga}\\
	\hline
	CIRCUITO & QUADRO & TENSÃO & FASE F1 & FASE F2 & FASE F3 \\ \hline
	\endfirsthead
	%
	\endhead
	%
	C12, C13 & QDC1   & 220    & 1800    & 1800    &         \\ \hline
	C20, C21 & QDC1   & 220    & 2250    & 2250    &         \\ \hline
	C26, C27 & QDC1   & 220    & 3250    &         & 3250    \\ \hline
	C28, C29 & QDC1   & 220    & 3250    & 3250    &         \\ \hline
	C14, C15 & QDC1   & 220    & 1800    &         & 1800    \\ \hline
	C16, C17 & QDC1   & 220    &         & 1800    & 1800    \\ \hline
	C12, C13 & QDC2   & 220    &         & 1800    & 1800    \\ \hline
	C18, C19 & QDC2   & 220    &         & 3250    & 3250    \\ \hline
	C8, C9   & QDC2   & 220    & 1800    & 1800    &         \\ \hline
	C16, C17 & QDC2   & 220    & 3250    &         & 3250    \\ \hline
	C10, C11 & QDC2   & 220    & 1800    &         & 1800    \\ \hline
	C18, C19 & QDC1   & 220    &         & 825     & 825     \\ \hline
	C22, C23 & QDC1   & 220    & 3250    &         & 3250    \\ \hline
	C24, C25 & QDC1   & 220    &         & 3250    & 3250    \\ \hline
	C14, C15 & QDC2   & 220    & 1750    & 1750    &         \\ \hline
	C4       & QDC1   & 127    &         &         & 1200    \\ \hline
	C5       & QDC1   & 127    &         &         & 1100    \\ \hline
	C6       & QDC1   & 127    & 900     &         &         \\ \hline
	C7       & QDC1   & 127    &         & 1100    &         \\ \hline
	C8       & QDC1   & 127    &         &         & 1200    \\ \hline
	C9       & QDC1   & 127    &         & 1500    &         \\ \hline
	C10      & QDC1   & 127    &         & 1350    &         \\ \hline
	C11      & QDC1   & 127    &         & 1400    &         \\ \hline
	C3       & QDC2   & 127    &         & 1300    &         \\ \hline
	C4       & QDC2   & 127    &         & 1200    &         \\ \hline
	C5       & QDC2   & 127    &         & 900     &         \\ \hline
	C6       & QDC2   & 127    &         &         & 1200    \\ \hline
	C7       & QDC2   & 127    & 1100    &         &         \\ \hline
	C1       & QDC1   & 127    & 1100    &         &         \\ \hline
	C2       & QDC1   & 127    & 1000    &         &         \\ \hline
	C3       & QDC1   & 127    &         &         & 1000    \\ \hline
	C1       & QDC2   & 127    & 1200    &         &         \\ \hline
	C2       & QDC2   & 127    & 900     &         &         \\ \hline
	&        &        &         &         &         \\ \hline
	TOTAL    &        &        & 30400   & 30525   & 29975   \\ \hline
\end{longtable}

As fases ficam assim com cargas de:

\begin{itemize}
	\item Fase F1: $30.400$ $VA$
	\item Fase F2: $30.525$ $VA$
	\item Fase F3: $29.975$ $VA$
\end{itemize}

Não existe diferença significativa entre as cargas das fases, sendo portanto considerada equilibrada como carga trifásica.

\section{FATOR DE CORREÇÃO POR AGRUPAMENTO DE CIRCUITOS}

A escolha de agrupar circuitos elétricos em uma instalação pode ter vários motivos, incluindo a necessidade de facilitar a identificação e manutenção dos circuitos, a redução do risco de sobrecarga elétrica e a adequação às normas de segurança elétrica aplicáveis. Também pode ser uma forma de otimizar o uso da energia elétrica e garantir um melhor funcionamento dos equipamentos elétricos conectados.

Determinar se haverá agrupamento de circuitos é importante para determinar fatores de cálculo, como o fator de agrupamento, um dos fatores de cálculo da corrente de projeto, que de fato determina o dimensionamento do fio dos circuitos em questão. Os fatores de agrupamento, por quantidade de condutores carregados e quantidade de circuitos são determinados pela tabela 42 da NBR-5410, que é mostrada na \autoref{fig:fator_de_agrupamento_nbr}:

\figura
{FATORES DE CORREÇÃO POR AGRUPAMENTO DE CIRCUITOS - NBR-5410} % Legenda
{.8} % % da largura da área de texto
{imagens/fator_de_agrupamento_nbr} % localização da figura
{NBR-5410:2004} % fonte da figura
{fator_de_agrupamento_nbr} % etiqueta
{}
{}
\pagebreak
O agrupamento de circuitos da edificação, junto com o fator de agrupamento ($FCA$) fica definido como mostrado na TABELA~\ref{tab:agrupamento_circuitos}

\begin{longtable}{|lllll|}
	\caption{AGRUPAMENTO MÁXIMO DE CIRCUITOS}
	\label{tab:agrupamento_circuitos}\\
	\hline
	\endfirsthead
	%
	\endhead
	%
	\multicolumn{3}{|l|}{CIRCUITOS}                                                                & \multicolumn{1}{l|}{TOTAL DE CIRCUITOS} & FCA \\ \hline
	\multicolumn{1}{|l|}{C1}       & \multicolumn{1}{l|}{C4}       & \multicolumn{1}{l|}{C12, C13} & \multicolumn{1}{l|}{3}                  & 0,7 \\ \hline
	\multicolumn{1}{|l|}{C4}       & \multicolumn{1}{l|}{C5}       & \multicolumn{1}{l|}{C14, C15} & \multicolumn{1}{l|}{3}                  & 0,7 \\ \hline
	\multicolumn{1}{|l|}{C3}       & \multicolumn{1}{l|}{C26, C27} & \multicolumn{1}{l|}{}         & \multicolumn{1}{l|}{2}                  & 0,8 \\ \hline
	\multicolumn{1}{|l|}{C2}       & \multicolumn{1}{l|}{C6}       & \multicolumn{1}{l|}{C18, C19} & \multicolumn{1}{l|}{3}                  & 0,7 \\ \hline
	\multicolumn{1}{|l|}{C8}       & \multicolumn{1}{l|}{C9}       & \multicolumn{1}{l|}{C10}      & \multicolumn{1}{l|}{3}                  & 0,7 \\ \hline
	\multicolumn{1}{|l|}{C2}       & \multicolumn{1}{l|}{C7}       & \multicolumn{1}{l|}{C20, C21} & \multicolumn{1}{l|}{3}                  & 0,7 \\ \hline
	\multicolumn{1}{|l|}{C3}      & \multicolumn{1}{l|}{C11}      & \multicolumn{1}{l|}{C16, C17} & \multicolumn{1}{l|}{3} & 0,7 \\ \hline
	\multicolumn{1}{|l|}{C22,C23} & \multicolumn{1}{l|}{C24, C25} & \multicolumn{1}{l|}{}         & \multicolumn{1}{l|}{2} & 0,8 \\ \hline
	\multicolumn{1}{|l|}{C28, C29} & \multicolumn{1}{l|}{}         & \multicolumn{1}{l|}{}         & \multicolumn{1}{l|}{1}                  & 1   \\ \hline
	\multicolumn{1}{|l|}{C1}       & \multicolumn{1}{l|}{C3}       & \multicolumn{1}{l|}{C8, C9}   & \multicolumn{1}{l|}{3}                  & 0,7 \\ \hline
	\multicolumn{1}{|l|}{C4}       & \multicolumn{1}{l|}{C16, C17} & \multicolumn{1}{l|}{}         & \multicolumn{1}{l|}{2}                  & 0,8 \\ \hline
	\multicolumn{1}{|l|}{C6}       & \multicolumn{1}{l|}{C18, C19} & \multicolumn{1}{l|}{}         & \multicolumn{1}{l|}{2}                  & 0,8 \\ \hline
	\multicolumn{1}{|l|}{C1}       & \multicolumn{1}{l|}{C5}       & \multicolumn{1}{l|}{C14, C15} & \multicolumn{1}{l|}{3}                  & 0,7 \\ \hline
	\multicolumn{1}{|l|}{C2}       & \multicolumn{1}{l|}{C7}       & \multicolumn{1}{l|}{C12, C13} & \multicolumn{1}{l|}{3}                  & 0,7 \\ \hline
	\multicolumn{1}{|l|}{C4}       & \multicolumn{1}{l|}{C10, C11} & \multicolumn{1}{l|}{}         & \multicolumn{1}{l|}{2}                  & 0,8 \\ \hline
\end{longtable}

\section{FATOR DE CORREÇÃO DE TEMPERATURA}

A temperatura média anual de Curitiba é de cerca de 16°C. No verão, a temperatura média fica em torno de 21°C e no inverno a temperatura média fica em torno de 12°C. No entanto, é importante lembrar que esses valores são apenas médias e as temperaturas podem variar bastante ao longo do ano e até mesmo dentro do mesmo dia.

O fator de correção de temperatura, para a temperatura ambiente e de solo, é dado pela tabela 40 da NMBR-5410, mostrada na \autoref{fig:fator_temperatura_nbr}:

\figura
{FATORES DE CORREÇÃO DE TEMPERATURA - NBR-5410} % Legenda
{.8} % % da largura da área de texto
{imagens/fator_temperatura_nbr} % localização da figura
{NBR-5410:2004} % fonte da figura
{fator_temperatura_nbr} % etiqueta
{}
{}

A temperatura considerada no projeto é 30°C, para linhas não subterrâneas e embutidas em alvenaria e lajes, e de 20°C para linhas subterrâneas, para ser conservador. Portanto, O fator de correção de temperatura ($FCT$) é igual a 1.

\section{DIMENSIONAMENTO DE CONDUTORES}

\subsection{CRITÉRIOS DE CÁLCULO}

O dimensionamento dos condutores é feito seguindo o roteiro abaixo:

\begin{itemize}
	\item Determina-se as seções dos condutores conforme a Capacidade de Corrente;
	\item Determina-se as seções dos condutores pelo Limite de Queda de Tensão;
	\item Determina-se as seções dos condutores pela seção mínima;
\end{itemize}

\subsubsection{CAPACIDADE DE CORRENTE}

A corrente corrigida é calculada da seguinte maneira:

\begin{equation}
	I_p = \frac{I_n}{FCT \times FCA} 
	\label{eqn:corrente_corrigida}
\end{equation}

Onde:

\begin{itemize}
	\item $I_p$: corrente corrigida;
	\item $I_n$: corrente de projeto;
	\begin{itemize}
		\item Para circuitos terminais será a corrente de projeto da maior derivação do disjuntor;
		\item Para circuitos de distribuição será a maior corrente de fase;
	\end{itemize}
	\item $FCT$: fator de correção de temperatura;
	\item $FCA$: fator de correção por agrupamento de circuitos;
\end{itemize}

A corrente de projeto é calculada da seguinte forma:

\begin{equation}
	I_n = \frac{P_n}{\eta_f \times V \times cos(\phi) \times \eta}
	\label{eqn:corrente_projeto}
\end{equation}

Onde:

\begin{itemize}
	\item $I_n$: corrente de projeto;
	\item $P_n$:  Potência nominal, em $W$;
	\item $\eta_f$: número de fases do circuito;
	\begin{itemize}
		\item Para circuitos com esquema F+N, F+F o valor é 1;
		\item Para circuitos com esquema 3F+N o valor é 3;
		\item Para circuitos com esquema 3F o valor é $\sqrt{3}$;
	\end{itemize}
	\item $V$: Tensão nominal;
	\begin{itemize}
		\item Para circuitos com esquema F+N e 3F+N é utilizado o valor de tensão de fase;
		\item Para circuitos com esquema F+F e 3F é utilizado o valor de tensão de linha;
	\end{itemize}
	\item $cos(\phi)$: fator de potência do equipamento;
	\item $\eta$: rendimento do equipamento.
\end{itemize}

Após calculada a corrente projeto de cada equipamento instalado em um determinado circuito, a corrente de projeto do circuito será a soma de todas as correntes de projeto dos equipamentos ligados ao circuito em questão.

\subsubsection{QUEDA DE TENSÃO}

O método utilizado para cálculo da queda de tensão é o método da tensão unitária.

O método da tensão unitária é um dos métodos utilizados para calcular a queda de tensão em uma instalação elétrica. Esse método é baseado no cálculo da queda de tensão por unidade de comprimento do condutor, considerando a corrente elétrica que passa pelo mesmo.

O cálculo da queda de tensão pelo método da tensão unitária pode ser realizado a partir da seguinte equação:

\begin{equation}
	\Delta V = K \times I \times L
\end{equation}

Onde:

\begin{itemize}
	\item $\Delta V$ é a queda de tensão no condutor, em volts (V);
	\item $K$ é a constante de proporcionalidade que relaciona a queda de tensão com a corrente e o comprimento do condutor, em ohms por quilômetro ($\frac{\Omega}{km}$);
	\item $I$ é a corrente elétrica que passa pelo condutor, em amperes ($A$);
	\item $L$ é o comprimento do condutor, em quilômetros ($km$).	
\end{itemize}


Para calcular o valor da constante $K$, é necessário conhecer a resistividade do material do condutor e sua seção transversal. A NBR-5410 define valores padrão para a resistividade dos condutores de cobre e alumínio, que podem ser utilizados para esse cálculo.

É importante ressaltar que a queda de tensão deve ser limitada a valores aceitáveis para garantir o correto funcionamento dos equipamentos elétricos. A NBR-5410 estabelece limites de queda de tensão para diferentes tipos de circuitos, que devem ser respeitados durante o projeto e a instalação da rede elétrica.

a constante $K$ é calculada pela seguinte equação:

\begin{equation}
	K = \frac{e_{\%} \times V}{I_n \times L}
\end{equation}

Onde:

\begin{itemize}
	\item $e_{\%}$: queda de tensão em porcentagem;
	\item $V$: Tensão nominal;
	\item $I_n$: Corrente de projeto;
	\item $L$: comprimento em quilômetros.
\end{itemize}

\subsubsection{SEÇÃO MÍNIMA}

A NBR-5410 define valores mínimos de seção transversal para os condutores utilizados em instalações elétricas de baixa tensão. Esses valores são definidos com base na corrente elétrica e na finalidade do circuito.

De forma geral, para circuitos de iluminação, a seção mínima dos condutores deve ser de $1,5$ $mm^2$ para correntes até $10$ $A$, e $2,5$ $mm^2$ para correntes entre $10$ $A$ e $16$ $A$. Para circuitos de tomadas, a seção mínima deve ser de $2,5$ $mm^2$ para correntes até $16$ $A$, e $4$ $mm^2$ para correntes entre $16$ $A$ e $20$ $A$.

Para circuitos de equipamentos específicos, como chuveiros elétricos, ar-condicionado, motores elétricos, entre outros, a seção mínima dos condutores deve ser definida com base nas especificações do fabricante do equipamento e nos critérios definidos pela NBR 5410.

Vale lembrar que a seção transversal dos condutores deve ser capaz de suportar a corrente elétrica prevista no circuito, evitando sobreaquecimentos e riscos de incêndio.
\pagebreak
\subsection{CÁLCULO DOS CONDUTORES}

Aplicando os critérios de cálculo aos valores da TABELA~\ref{tab:previsao_de_carga} resulta nos valores da TABELA~\ref{tab:calculo_de_condutores}:

\begin{table}[H]
	\caption{DIMENSIONAMENTO DE CONDUTORES - TODOS OS CRITÉRIOS}
	\label{tab:calculo_de_condutores}
	\resizebox{\columnwidth}{!}{%
	\begin{tabular}{|c|c|c|c|c|c|c|c|c|c|c|c|}
		\hline
		CIRCUITO &
		CORRENTE DE PROJETO IP &
		FCT &
		FCA &
		CORRENTE DE CORRIGIDA IP' &
		\begin{tabular}[c]{@{}c@{}}COMPRIMENTO \\  L (m)\end{tabular} &
		QUEDA DE TENSÃO &
		I CONDUTOR &
		DIA FASE CORRENTE &
		DIA FASE QUEDA &
		DIA NEUTRO &
		DIA PROTEÇÃO \\ \hline
		C12, C13 & 16,364 & 1 & 0,70 & 23,38 & 12,23 & 43,97 & 24,00 & 2,5 &  & 2,5 & 2,5 \\ \hline
		C20, C21 & 20,455 & 1 & 0,70 & 29,22 & 14,86 & 28,95 & 32,00 & 4   &  & 4   & 4   \\ \hline
		C26, C27 & 29,545 & 1 & 0,80 & 36,93 & 5,88  & 50,65 & 41,00 & 6   &  & 6   & 6   \\ \hline
		C28, C29 & 29,545 & 1 & 1,00 & 29,55 & 9,36  & 31,82 & 32,00 & 4   &  & 4   & 4   \\ \hline
		C14, C15 & 16,364 & 1 & 0,70 & 23,38 & 10,85 & 49,56 & 24,00 & 2,5 &  & 2,5 & 2,5 \\ \hline
		C16, C17 & 16,364 & 1 & 0,70 & 23,38 & 13,79 & 39,00 & 24,00 & 2,5 &  & 2,5 & 2,5 \\ \hline
		C12, C13 & 16,364 & 1 & 0,70 & 23,38 & 13,35 & 40,28 & 24,00 & 2,5 &  & 2,5 & 2,5 \\ \hline
		C18, C19 & 29,545 & 1 & 0,80 & 36,93 & 11,36 & 26,22 & 41,00 & 6   &  & 6   & 6   \\ \hline
		C8, C9   & 16,364 & 1 & 0,70 & 23,38 & 6,86  & 78,39 & 24,00 & 2,5 &  & 2,5 & 2,5 \\ \hline
		C16, C17 & 29,545 & 1 & 0,80 & 36,93 & 9,22  & 32,30 & 41,00 & 6   &  & 6   & 6   \\ \hline
		C10, C11 & 16,364 & 1 & 0,80 & 20,45 & 9,04  & 59,49 & 24,00 & 2,5 &  & 2,5 & 2,5 \\ \hline
		C18, C19 & 7,500  & 1 & 0,70 & 10,71 & 19,92 & 58,90 & 24,00 & 2,5 &  & 2,5 & 2,5 \\ \hline
		C22, C23 & 29,545 & 1 & 0,80 & 36,93 & 14,70 & 20,26 & 41,00 & 6   &  & 6   & 6   \\ \hline
		C24, C25 & 29,545 & 1 & 0,80 & 36,93 & 14,84 & 20,07 & 41,00 & 6   &  & 6   & 6   \\ \hline
		C14, C15 & 15,909 & 1 & 0,70 & 22,73 & 16,83 & 32,87 & 24,00 & 2,5 &  & 2,5 & 2,5 \\ \hline
		C4       & 9,449  & 1 & 0,70 & 13,50 & 15,14 & 35,51 & 24,00 & 2,5 &  & 2,5 & 2,5 \\ \hline
		C5       & 8,661  & 1 & 0,70 & 12,37 & 14,05 & 41,74 & 24,00 & 2,5 &  & 2,5 & 2,5 \\ \hline
		C6       & 7,087  & 1 & 0,70 & 10,12 & 20,72 & 34,60 & 24,00 & 2,5 &  & 2,5 & 2,5 \\ \hline
		C7       & 8,661  & 1 & 0,70 & 12,37 & 12,06 & 48,63 & 24,00 & 2,5 &  & 2,5 & 2,5 \\ \hline
		C8       & 9,449  & 1 & 0,70 & 13,50 & 14,12 & 38,08 & 24,00 & 2,5 &  & 2,5 & 2,5 \\ \hline
		C9       & 11,811 & 1 & 0,70 & 16,87 & 15,70 & 27,40 & 24,00 & 2,5 &  & 2,5 & 2,5 \\ \hline
		C10      & 10,630 & 1 & 0,70 & 15,19 & 16,48 & 29,00 & 24,00 & 2,5 &  & 2,5 & 2,5 \\ \hline
		C11      & 11,024 & 1 & 0,70 & 15,75 & 17,06 & 27,01 & 24,00 & 2,5 &  & 2,5 & 2,5 \\ \hline
		C3       & 10,236 & 1 & 0,70 & 14,62 & 14,02 & 35,40 & 24,00 & 2,5 &  & 2,5 & 2,5 \\ \hline
		C4       & 9,449  & 1 & 0,80 & 11,81 & 10,86 & 49,51 & 24,00 & 2,5 &  & 2,5 & 2,5 \\ \hline
		C5       & 7,087  & 1 & 0,70 & 10,12 & 14,64 & 48,96 & 24,00 & 2,5 &  & 2,5 & 2,5 \\ \hline
		C6       & 9,449  & 1 & 0,80 & 11,81 & 17,03 & 31,57 & 24,00 & 2,5 &  & 2,5 & 2,5 \\ \hline
		C7       & 8,661  & 1 & 0,70 & 12,37 & 16,88 & 34,75 & 24,00 & 2,5 &  & 2,5 & 2,5 \\ \hline
		C1       & 8,661  & 1 & 0,70 & 12,37 & 16,64 & 35,25 & 17,50 & 1,5 &  & 1,5 & 1,5 \\ \hline
		C2       & 7,874  & 1 & 0,70 & 11,25 & 15,50 & 41,62 & 17,50 & 1,5 &  & 1,5 & 1,5 \\ \hline
		C3       & 7,874  & 1 & 0,70 & 11,25 & 17,87 & 36,10 & 17,50 & 1,5 &  & 1,5 & 1,5 \\ \hline
		C1       & 9,449  & 1 & 0,70 & 13,50 & 7,98  & 67,37 & 17,50 & 1,5 &  & 1,5 & 1,5 \\ \hline
		C2       & 7,087  & 1 & 0,70 & 10,12 & 12,63 & 56,76 & 17,50 & 1,5 &  & 1,5 & 1,5 \\ \hline
	\end{tabular}}
\end{table}

A tensão unitária não é suficiente para considerar o critério da queda de tensão, portanto a decisão fica entre o critério da capacidade de corrente e a seção mínima. Para os circuitos de iluminação e tomadas de uso geral, o critério da seção mínima mostrou-se mais do que suficiente.

As tomadas de uso específico foram calculadas pelo critério da capacidade de corrente. Para os circuitos com seção de $4$ $mm^2$, decidiu-se optar por cabos de $6$ $mm^2$ por padronização da instalação.

Os cabos de proteção são, segundo a norma NBR-5410, de seção igual a fase e neutro.

\section{DIMENSIONAMENTO DE ELETRODUTOS}

O dimensionamento de eletrodutos é uma etapa importante no projeto de uma instalação elétrica, pois os eletrodutos são responsáveis por proteger e conduzir os condutores elétricos de forma segura e eficiente. O dimensionamento adequado dos eletrodutos é importante para evitar o superaquecimento dos condutores e garantir a dissipação de calor, bem como para permitir a manutenção e a passagem de novos condutores no futuro.

O dimensionamento de eletrodutos deve levar em consideração a quantidade de condutores que serão instalados, sua seção transversal, o tipo de instalação (aparente ou embutida), o tipo de eletroduto (flexível ou rígido), a temperatura ambiente, entre outros fatores.

A NBR 5410 estabelece critérios para o dimensionamento de eletrodutos, levando em conta a quantidade e o diâmetro dos condutores, bem como a disposição dos eletrodutos no ambiente. Para isso, a norma apresenta tabelas e fórmulas que permitem calcular o diâmetro mínimo do eletroduto, a partir do número de condutores, sua seção transversal e o tipo de instalação.

Além dos critérios estabelecidos pela norma, o dimensionamento de eletrodutos deve levar em conta a facilidade de manutenção e inspeção dos condutores, bem como a necessidade de espaço para passagem de novos cabos no futuro. Também é importante considerar a possibilidade de expansão da instalação elétrica, prevendo o uso de eletrodutos com capacidade para suportar novos circuitos e equipamentos.

A taxa máxima de ocupação em relação à área da seção transversal dos
eletrodutos não deve ser superior a:


\begin{itemize}
	
	\item 53\% no caso de um condutor ou cabo;
	\item 31\% no caso de dois condutores ou cabos;
	\item 40\% no caso de três ou mais condutores ou cabos
	
\end{itemize}

A área útil de um eletroduto é definido como:

\begin{equation}
	A_e = \pi \times \frac{D_i^2}{4}
\end{equation}

E considerando que a soma das áreas externas dos condutores a serem instalados é dado por:

\begin{equation}
	\sum A_{cond}
\end{equation}

Então, o diâmetro interno do eletroduto é dado por:

\begin{equation}
	D_d = \sqrt{\frac{4 \times \sum A_{cond}}{f \times \pi}}
\end{equation}

Onde $f$ é a taxa de ocupação do eletroduto.

Segue-se um roteiro básico:

\begin{itemize}
	\item Determina-se a seção total ocupada pelos	condutores usando a tabela de fabricantes de condutores.
	\item Determina-se o diâmetro externo nominal do eletroduto
	(mm) consultando as	tabelas de eletrodutos
	\item Caso os condutores instalados em um mesmo	eletroduto sejam do mesmo tipo e tenham seções nominais iguais, pode-se eliminar os passos anteriores e encontrar o	diâmetro externo nominal	do eletroduto em função	da quantidade e seção dos condutores diretamente	por tabelas específicas
\end{itemize}

No caso, o terceiro passo do roteiro não é utilizado pois os circuitos agrupados são de seções nominais diferentes. Portanto, somente os 2 primeiros passos são utilizados.

Aplicados os 2 passos citados, chega-se à conclusão de que os eletrodutos indicados são de $\frac{3}{4}\inches$ e $1\inches$ de diâmetro interno. Por padronização, escolhe-se o eletroduto corrugado de $32$ $mm$ ou $1$ $\frac{1}{4}\inches$ de diâmetro externo.

\section{DIMENSIONAMENTO DOS DISPOSITIVOS DE PROTEÇÃO}

O dimensionamento dos dispositivos de proteção elétrica é fundamental para garantir a segurança da instalação elétrica e dos usuários, evitando riscos de curto-circuito, sobrecarga e outros tipos de falhas elétricas.

Os dispositivos de proteção elétrica mais comuns em uma instalação elétrica de baixa tensão incluem os disjuntores, os fusíveis, os interruptores diferenciais residuais (IDRs) e os dispositivos de proteção contra surtos (DPS).

O dimensionamento desses dispositivos deve ser realizado levando em consideração a corrente elétrica prevista no circuito, bem como as características dos equipamentos elétricos a serem protegidos. Os disjuntores, por exemplo, devem ser dimensionados para suportar a corrente elétrica máxima prevista no circuito, evitando o risco de sobrecarga. Os fusíveis também devem ser dimensionados para suportar a corrente máxima, além de serem escolhidos de acordo com a curva de atuação necessária para a proteção do circuito.

Já os interruptores diferenciais residuais (IDRs) são responsáveis por proteger as pessoas contra choques elétricos, detectando correntes de fuga no circuito e interrompendo o fornecimento de energia elétrica. O dimensionamento dos IDRs deve levar em conta a corrente elétrica máxima do circuito e o tempo de atuação necessário para garantir a proteção adequada.

Por fim, os dispositivos de proteção contra surtos (DPS) são responsáveis por proteger os equipamentos eletrônicos contra danos causados por surtos de tensão. O dimensionamento dos DPS deve levar em conta a tensão nominal da instalação elétrica, bem como a capacidade de absorção dos surtos de tensão.

\subsection{DISJUNTORES DE PROTEÇÃO (DP)}

Os disjuntores devem proteger contra sobrecarga e curto-circuito. O circuito mais longo da residência tem cerca de 18 metros de comprimento, portanto, desconsidera-se o calculo de curto-circuito para efeitos de dimensionamento, considerando-se somente a corrente máxima de sobrecarga.

Utilizando a corrente máxima de sobrecarga como critério de dimensionamento, chega-se aos resultados mostrados na TABELA~\ref{tab:dimensionamento_disjuntores} para cada circuito:


\begin{longtable}{|c|c|c|c|c|}
	\caption{DIMENSIONAMENTO DOS DISJUNTORES POR QUADRO E CIRCUITO}
	\label{tab:dimensionamento_disjuntores}\\
	\hline
	CIRCUITO                       & QUADRO & I CONDUTOR & DISJUNTOR CORRENTE & N POLOS \\ \hline
	\endfirsthead
	%
	\endhead
	%
	C12, C13                       & QDC1   & 24,00      & 20                 & 2       \\ \hline
	C20, C21                       & QDC1   & 32,00      & 30                 & 2       \\ \hline
	\multicolumn{1}{|l|}{C26, C27} & QDC1   & 41,00      & 40                 & 2       \\ \hline
	\multicolumn{1}{|l|}{C28, C29} & QDC1   & 32,00      & 30                 & 2       \\ \hline
	\multicolumn{1}{|l|}{C14, C15} & QDC1   & 24,00      & 20                 & 2       \\ \hline
	\multicolumn{1}{|l|}{C16, C17} & QDC1   & 24,00      & 20                 & 2       \\ \hline
	C12, C13                       & QDC2   & 24,00      & 20                 & 2       \\ \hline
	C18, C19                       & QDC2   & 41,00      & 40                 & 2       \\ \hline
	C8, C9                         & QDC2   & 24,00      & 20                 & 2       \\ \hline
	C16, C17                       & QDC2   & 41,00      & 40                 & 2       \\ \hline
	C10, C11                       & QDC2   & 24,00      & 20                 & 2       \\ \hline
	\multicolumn{1}{|l|}{C18, C19} & QDC1   & 24,00      & 20                 & 2       \\ \hline
	C22, C23                       & QDC1   & 41,00      & 40                 & 2       \\ \hline
	C24, C25                       & QDC1   & 41,00      & 40                 & 2       \\ \hline
	C14, C15                       & QDC2   & 24,00      & 20                 & 2       \\ \hline
	C4                             & QDC1   & 24,00      & 20                 & 1       \\ \hline
	C5                             & QDC1   & 24,00      & 20                 & 1       \\ \hline
	C6                             & QDC1   & 24,00      & 20                 & 1       \\ \hline
	C7                             & QDC1   & 24,00      & 20                 & 1       \\ \hline
	C8                             & QDC1   & 24,00      & 20                 & 1       \\ \hline
	C9                             & QDC1   & 24,00      & 20                 & 1       \\ \hline
	C10                            & QDC1   & 24,00      & 20                 & 1       \\ \hline
	C11                            & QDC1   & 24,00      & 20                 & 1       \\ \hline
	C3                             & QDC2   & 24,00      & 20                 & 1       \\ \hline
	C4                             & QDC2   & 24,00      & 20                 & 1       \\ \hline
	C5                             & QDC2   & 24,00      & 20                 & 1       \\ \hline
	C6                             & QDC2   & 24,00      & 20                 & 1       \\ \hline
	C7                             & QDC2   & 24,00      & 20                 & 1       \\ \hline
	C1                             & QDC1   & 17,50      & 15                 & 1       \\ \hline
	C2                             & QDC1   & 17,50      & 15                 & 1       \\ \hline
	C3                             & QDC1   & 17,50      & 15                 & 1       \\ \hline
	C1                             & QDC2   & 17,50      & 15                 & 1       \\ \hline
	C2                             & QDC2   & 17,50      & 15                 & 1       \\ \hline
\end{longtable}

Como disjuntor geral dos quadros de distribuição, escolhe-se:

\begin{itemize}
	\item QDC1: $80$ $A$
	\item QDC2: $70$ $A$
\end{itemize}

\subsection{DISPOSITIVOS DE PROTEÇÃO CONTRA SURTOS (DPS)}

O cálculo de DPS (Dispositivos de Proteção contra Surtos) é feito levando em consideração a tensão nominal da instalação elétrica e a capacidade de absorção de surtos de tensão dos equipamentos eletrônicos que serão protegidos. O objetivo é garantir que os equipamentos estejam adequadamente protegidos contra surtos de tensão que possam causar danos ou falhas.

O cálculo do DPS envolve a análise da curva de atuação do dispositivo, que determina a capacidade de absorção de surtos de tensão em função do nível de tensão aplicado. Essa curva de atuação é definida pelo fabricante do DPS e deve ser levada em conta no momento do cálculo.

O cálculo é feito a partir da seguinte fórmula:
\begin{equation}
	P = \frac{U_c \times I_{sc} \times K \times S}{U_p}
\end{equation}

Onde:

\begin{itemize}
	\item $P$: é a capacidade de corrente de impulso do DPS, em kA;
	\item $U_c$: é a tensão nominal da instalação elétrica, em volts;
	\item $I_{sc}$: é a corrente de curto-circuito máxima prevista na instalação elétrica, em kA;
	\item $K$: é o fator de correção, que leva em conta as características da instalação elétrica (fator de potência, comprimento dos cabos, etc.);
	\item $S$: é o fator de proteção, que leva em conta o tipo de equipamento a ser protegido (por exemplo, computadores, equipamentos de telecomunicações, etc.);
	\item $U_p$: é a tensão residual máxima permitida após o surto de tensão, em volts.
	
\end{itemize}

Após o cálculo, é importante selecionar um DPS com capacidade de corrente de impulso igual ou superior ao valor obtido na fórmula, levando em conta também as normas e regulamentos locais.

Para um fator de proteção $S=1$, $K=0,8$, $U_p=600V$, $I_{sc}=10000A$ (Assumida) e $U_c=127V$, obtém-se:

\begin{equation}
	\begin{split}
		P = \frac{127 \times 0,8 \times 1 \times 10000}{600} \\
		P = 1693,33 A
	\end{split}
\end{equation}

Portanto, DPS com correntes de pulso maiores que $1,7$ $kA$ servem ao propósito.

É bastante comum no mercado DPS com correntes de pulso de $5kA$ e tensão nominal de $275$ $V$, sendo este o valor escolhido como calculado para os DPS da instalação elétrica.

Para o QDC1, utiliza-se 4 DPS de $5kA$ e tensão nominal de $275$ $V$, sendo o mesmo válido para o QDC2.

\subsection{DISPOSITIVOS DIFERENCIAIS RESIDUAIS (DR)}

O cálculo do DR (Dispositivo Diferencial Residual) é feito para garantir a proteção das pessoas contra choques elétricos, desligando o circuito elétrico em caso de fuga de corrente elétrica.

O IDR é calculado a partir da corrente nominal do circuito elétrico que ele vai proteger, que é geralmente igual ou inferior à corrente nominal do dispositivo de proteção contra sobrecorrente (disjuntor ou fusível) correspondente.

A norma NBR-5410 exige que circuitos em áreas úmidas e tomadas de uso específico sejam protegidos por DRs. Para a proteção da vida humana, é necessário também que a corrente de sensibilidade seja igual ou inferior a $30$ $mA$.


Dos circuitos de tomadas de uso específico, tem-se a previsão de circuitos com DR mostrada na TABELA~\ref{tab:dr}:

% Please add the following required packages to your document preamble:
% \usepackage{longtable}
% Note: It may be necessary to compile the document several times to get a multi-page table to line up properly
\begin{longtable}{|c|c|c|c|c|}
	\caption{DIMENSIONAMENTO DOS DISPOSITIVOS DIFERENCIAIS RESIDUAIS}
	\label{tab:dr}\\
	\hline
	CIRCUITO                       & QUADRO & I CONDUTOR & DR (I=30mA) & N POLOS \\ \hline
	\endfirsthead
	%
	\endhead
	%
	C12, C13                       & QDC1   & 24,00      & 20          & 2       \\ \hline
	C20, C21                       & QDC1   & 32,00      & 30          & 2       \\ \hline
	\multicolumn{1}{|l|}{C26, C27} & QDC1   & 41,00      & 40          & 2       \\ \hline
	\multicolumn{1}{|l|}{C28, C29} & QDC1   & 32,00      & 30          & 2       \\ \hline
	\multicolumn{1}{|l|}{C14, C15} & QDC1   & 24,00      & 20          & 2       \\ \hline
	\multicolumn{1}{|l|}{C16, C17} & QDC1   & 24,00      & 20          & 2       \\ \hline
	C12, C13                       & QDC2   & 24,00      & 20          & 2       \\ \hline
	C18, C19                       & QDC2   & 41,00      & 40          & 2       \\ \hline
	C8, C9                         & QDC2   & 24,00      & 20          & 2       \\ \hline
	C16, C17                       & QDC2   & 41,00      & 40          & 2       \\ \hline
	C10, C11                       & QDC2   & 24,00      & 20          & 2       \\ \hline
	\multicolumn{1}{|l|}{C18, C19} & QDC1   & 24,00      & 20          & 2       \\ \hline
	C22, C23                       & QDC1   & 41,00      & 40          & 2       \\ \hline
	C24, C25                       & QDC1   & 41,00      & 40          & 2       \\ \hline
	C14, C15                       & QDC2   & 24,00      & 20          & 2       \\ \hline
\end{longtable}

Em virtude de haver muitos circuitos com DR, optou-se por 2 IDRs, um para cada quadro, em todos os circuitos listados na TABELA~\ref{tab:dr}:

\begin{itemize}
	\item \textit{QDC1:} IDR $I_n = 80 A$ ($\Delta I_n = 30 mA$)
	\item \textit{QDC2:} IDR $I_n = 70 A$ ($\Delta I_n = 30 mA$)
\end{itemize}


% PARTE DA PREPARAÇÃO DA PESQUISA
% ----------------------------------------------------------
%\part{Preparação da pesquisa}
%\input{02-cap02}
%

% PARTE DOS REFERENCIAIS TEÓRICOS
% ----------------------------------------------------------
%\part{Referenciais teóricos}
%\input{03-cap03}

% PARTE DOS RESULTADOS
% ----------------------------------------------------------
%\part{Resultados}
%\input{05-cap05}


% Finaliza a parte no bookmark do PDF
% para que se inicie o bookmark na raiz
% e adiciona espaço de parte no Sumário
% ----------------------------------------------------------
%\phantompart

% ---
% Conclusão (outro exemplo de capítulo sem numeração e presente no sumário)
% ---
%\chapter*[Conclusão]{Conclusão}
%\addcontentsline{toc}{chapter}{Conclusão}
% ---
%\input{99-cap99}

% ELEMENTOS PÓS-TEXTUAIS
% ----------------------------------------------------------
\postextual


% Referências bibliográficas
% ----------------------------------------------------------
%\bibliography{referencias}


% \printbibliography[heading=bay]
% ----------------------------------------------------------


% Glossário
% ----------------------------------------------------------
% Consulte o manual da classe abntex2 para orientações sobre o glossário.
%
%\glossary

% Apêndices
% ----------------------------------------------------------
\ifthenelse{\equal{\terApendice}{Sim}}
{\begin{apendicesenv}

% Imprime uma página indicando o início dos apêndices
\partapendices

   % Existem várias formas de se colocar anexos.
   % O exemplo abaixo coloca 2 apêndices denominados de 
   % DESENVOLVIMENTO DETALHADO DA PINTURA e 
   % ESCOLHA DO MATERIAL DE IMPRESSÃO:
   % ---
   % --- insere um capítulo que é tratado como um apêndice
   %\chapter{DESENVOLVIMENTO DETALHADO DA PINTURA}
   % 
   %\lipsum[29] % gera um parágrafo
   %
   % --- insere um capítulo que é tratado como um apêndice
   %\chapter{ESCOLHA DO MATERIAL DE IMPRESSÃO}
   % 
   %\lipsum[30] % gera um parágrafo


% --- Insere o texto do arquivo ap01.tex
% 
% --- O conteúdo do arquivo pode ser vários anexos ou um único apêndices.
%     A vantagem de se utilizar este procedimento é de suprimi-lo
%     das compilações enquanto se processa o resto do documento.

\input{ap01}

\end{apendicesenv}
}{}


% Anexos
% ----------------------------------------------------------
\ifthenelse{\equal{\terAnexo}{Sim}}{
\begin{anexosenv}

% --- Imprime uma página indicando o início dos anexos
 \partanexos

   % Existem várias formas de se colocar anexos.
   % O exemplo abaixo coloca 2 anexos denominados de 
   % TABELA DE VALORES e GRÁFICOS DE BALANCEMANTO:
   % ---
   % --- insere um capítulo que é tratado como um anexo
   %\chapter{TABELAS DE VALORES}
   % 
   %\lipsum[31] % gera um parágrafo
   %
   % --- insere um capítulo que é tratado como um anexo
   %\chapter{GRÁFICOS DE BALANCEAMENTO}
   % 
   %\lipsum[32] % gera um parágrafo


% --- Insere o texto do arquivo ax01.tex
% 
% --- O conteúdo do arquivo pode ser vários anexos ou um único anexo.
%     A vantagem de se utilizar este procedimento é de suprimi-lo
%     das compilações enquanto se processa o resto do documento.

 \input{ax01} 


\end{anexosenv}
}{}

% INDICE REMISSIVO
%---------------------------------------------------------------------
\ifthenelse{\equal{\terIndiceR}{Sim}}{
\phantompart
\printindex
}{}

\end{document}
