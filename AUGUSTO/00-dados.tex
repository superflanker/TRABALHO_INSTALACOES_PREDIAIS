%%%%%%%%%%%%%%%%%%%%%%%%%%%%%%%%%%%%%%%%%%%%%%%%%%%%%%%
% Arquivo para entrada de dados para a parte pré textual
%%%%%%%%%%%%%%%%%%%%%%%%%%%%%%%%%%%%%%%%%%%%%%%%%%%%%%%
% 
% Basta digitar as informações indicidas, no formato 
% apresentado.
%
%%%%%%%
% Os dados solicitados são, na ordem:
%
% tipo do trabalho
% componentes do trabalho 
% título do trabalho
% nome do autor
% local 
% data (ano com 4 dígitos)
% orientador(a)
% coorientador(a)(as)(es)
% arquivo com dados bibliográficos
% instituição
% setor
% programa de pós gradução
% curso
% preambulo
% data defesa
% CDU
% errata
% assinaturas - termo de aprovação
% resumos & palavras chave
% agradecimentos
% dedicatoria
% epígrafe


% Informações de dados para CAPA e FOLHA DE ROSTO
%----------------------------------------------------------------------------- 
\tipotrabalho{Relatório Técnico}
%\tipotrabalho{Trabalho Acadêmico}
%    {Relatório Técnico}
%    {Dissertação}
%    {Tese}
%    {Monografia}

% Marcar Sim para as partes que irão compor o documento pdf
%----------------------------------------------------------------------------- 
 \providecommand{\terCapa}{Sim}
 \providecommand{\terFolhaRosto}{Sim}
 \providecommand{\terTermoAprovacao}{Nao}
 \providecommand{\terDedicatoria}{Nao}
 \providecommand{\terFichaCatalografica}{Nao}
 \providecommand{\terEpigrafe}{Nao}
 \providecommand{\terAgradecimentos}{Nao}
 \providecommand{\terErrata}{Nao}
 \providecommand{\terListaFiguras}{Sim}
 \providecommand{\terListaTabelas}{Sim}
 \providecommand{\terSiglasAbrev}{Nao}
 \providecommand{\terResumos}{Nao}
 \providecommand{\terSumario}{Sim}
 \providecommand{\terAnexo}{Nao}
 \providecommand{\terApendice}{Nao}
 \providecommand{\terIndiceR}{Nao}
%----------------------------------------------------------------------------- 

\titulo{Memorial Descritivo\\Memorial De Cálculo\\Residência TE344}
\autor{Augusto Mathias Adams - GRR20172143\\
	   Claudinei Eustaquio Rodrigues  - GRR20191718\\
   	   Moisés Alves Guergolet - GRR20202614}
\local{Curitiba}
\data{2023} %Apenas ano 4 dígitos

% Orientador ou Orientadora
\orientador{
	Prof\textordfeminine~Cleverson Luiz da Silva Pinto, Dr.}
%Prof Emílio Eiji Kavamura, MSc}
\orientadora{}
% Pode haver apenas uma orientadora ou um orientador
% Se houver os dois prevalece o feminino.

% Em termos de coorientação, podem haver até quatro neste modelo
% Sendo 2 mulhere e 2 homens.
% Coorientador ou Coorientadora
%\coorientador{}%Prof Morgan Freeman, DSc}
%\coorientadora{Prof\textordfeminine~Audrey Hepburn, DEng}

% Segundo Coorientador ou Segunda Coorientadora
\scoorientador{}
%Prof Jack Nicholson, DEng}
\scoorientadora{}
%Prof\textordfeminine~Ingrid Bergman, DEng}
% ----------------------------------------------------------
% \addbibresource{referencias.bib}

% ----------------------------------------------------------
\instituicao{%
Universidade Federal do Paraná}

\def \ImprimirSetor{}%
%Setor de Tecnologia}

\def \ImprimirProgramaPos{}%Programa de Pós Graduação em Engenharia de Construção Civil}

\def \ImprimirCurso{}%
%Curso de Engenharia Civil}

\preambulo{
Trabalho apresentado como requisito parcial de avaliação da Disciplina INSTALAÇÕES ELÉTRICAS PREDIAIS E INDUSTRIAIS I (TE-344) do Curso de Engenharia Elétrica, ênfase em Sistemas Embarcados - Noturno, do Departamento de Engenharia Elétrica da Universidade Federal do Paraná}

%----------------------------------------------------------------------------- 

\newcommand{\imprimirCurso}{}
%Programa de P\'os Gradua\c{c}\~ao em Engenharia da Constru\c{c}\~ao Civil}

\newcommand{\imprimirDataDefesa}{
09 de Dezembro de 2018}

\newcommand{\imprimircdu}{
02:141:005.7}

% ----------------------------------------------------------
\newcommand{\imprimirerrata}{
Elemento opcional da \cites[4.2.1.2]{NBR14724:2011}. Exemplo:

\vspace{\onelineskip}

FERRIGNO, C. R. A. \textbf{Tratamento de neoplasias ósseas apendiculares com
reimplantação de enxerto ósseo autólogo autoclavado associado ao plasma
rico em plaquetas}: estudo crítico na cirurgia de preservação de membro em
cães. 2011. 128 f. Tese (Livre-Docência) - Faculdade de Medicina Veterinária e
Zootecnia, Universidade de São Paulo, São Paulo, 2011.

\begin{table}[htb]
\center
\footnotesize
\begin{tabular}{|p{1.4cm}|p{1cm}|p{3cm}|p{3cm}|}
  \hline
   \textbf{Folha} & \textbf{Linha}  & \textbf{Onde se lê}  & \textbf{Leia-se}  \\
    \hline
    1 & 10 & auto-conclavo & autoconclavo\\
   \hline
\end{tabular}
\end{table}}

% Comandos de dados - Data da apresentação
\providecommand{\imprimirdataapresentacaoRotulo}{}
\providecommand{\imprimirdataapresentacao}{}
\newcommand{\dataapresentacao}[2][\dataapresentacaoname]{\renewcommand{\dataapresentacao}{#2}}

% Comandos de dados - Nome do Curso
\providecommand{\imprimirnomedocursoRotulo}{}
\providecommand{\imprimirnomedocurso}{}
\newcommand{\nomedocurso}[2][\nomedocursoname]
  {\renewcommand{\imprimirnomedocursoRotulo}{#1}
\renewcommand{\imprimirnomedocurso}{#2}}


% ----------------------------------------------------------
\newcommand{\AssinaAprovacao}{

\assinatura{%\textbf
   {Professora} \\ UFPR}
   \assinatura{%\textbf
   {Professora} \\ ENSEADE}
   \assinatura{%\textbf
   {Professora} \\ TIT}
   %\assinatura{%\textbf{Professor} \\ Convidado 4}
      
   \begin{center}
    \vspace*{0.5cm}
    %{\large\imprimirlocal}
    %\par
    %{\large\imprimirdata}
    \imprimirlocal, \imprimirDataDefesa.
    \vspace*{1cm}
  \end{center}
  }
  
% ----------------------------------------------------------
%\newcommand{\Errata}{%\color{blue}
%Elemento opcional da \textcite[4.2.1.2]{NBR14724:2011}. Exemplo:
%}

% ----------------------------------------------------------
\newcommand{\EpigrafeTexto}{%\color{blue}
\textit{``Não vos amoldeis às estruturas deste mundo, \\
		mas transformai-vos pela renovação da mente, \\
		a fim de distinguir qual é a vontade de Deus: \\
		o que é bom, o que Lhe é agradável, o que é perfeito.\\
		(Bíblia Sagrada, Romanos 12, 2)}
}

% ----------------------------------------------------------
\newcommand{\ResumoTexto}{
A aula prática referente ao laboratório 04 da disciplina de Microeletrônica I trata de 2 projetos de circuitos lógicos sequenciais: Temporizador com contagem de 0 a 9 e contador BCD progressivo/regressivo com habilitação. Todos os circuitos lógicos foram implementados utilizando o kit Nexys 2 da Digilent, utilizando a Ferramenta Ise, versão 14.7, da Xilink. Este relatório inclui a descrição e o desenvolvimento dos dois circuitos lógicos, juntamente com seus esquemáticos RTL obtidos após a síntese. Também constam  as simulações, explicação dos resultados, os códigos da implementação, do \textit{"testbench"} e o resumo dos recursos lógicos utilizados. 

}

\newcommand{\PalavraschaveTexto}{%\color{blue}
Circuitos lógicos; FPGA; Lógica Sequencial; Contadores; Temporizadores.}

% ----------------------------------------------------------
\newcommand{\AbstractTexto}{%\color{blue}
The practical class referring to laboratory 04 of the discipline of Microelectronics I deals with 2 projects of sequential logic circuits: Timer with count from 0 to 9 and progressive/regressive BCD counter with enabling. All logic circuits were implemented using Digilent's Nexys 2 kit, using Xilink's Ise Tool, version 14.7. This report includes the description and development of the two logic circuits, along with their RTL schematics obtained after synthesis. There are also simulations, explanation of results, implementation codes, \textit{"testbench"} and a summary of the logic resources used.
}
% ---
\newcommand{\KeywordsTexto}{%\color{blue}
logic circuits; FPGA; Sequential Logic; Counters; Timers.}

% ----------------------------------------------------------
\newcommand{\Resume}
{%\color{blue}
%Il s'agit d'un résumé en français.
} 
% ---
\newcommand{\Motscles}
{%\color{blue}
 %latex. abntex. publication de textes.
}

% ----------------------------------------------------------
\newcommand{\Resumen}
{%\color{blue}
%Este es el resumen en español.
}
% ---
\newcommand{\Palabrasclave}
{%\color{blue}
%latex. abntex. publicación de textos.
}

% ----------------------------------------------------------
\newcommand{\AgradecimentosTexto}{%\color{blue}
Os agradecimentos principais são direcionados à Gerald Weber, Miguel Frasson, Leslie H. Watter, Bruno Parente Lima, Flávio de  Vasconcellos Corrêa, Otavio Real Salvador, Renato Machnievscz\footnote{Os nomes dos integrantes do primeiro
projeto abn\TeX\ foram extraídos de \url{http://codigolivre.org.br/projects/abntex/}} e todos aqueles que contribuíram para que a produção de trabalhos acadêmicos conforme as normas ABNT com \LaTeX\ fosse possível.

Agradecimentos especiais são direcionados ao Centro de Pesquisa em Arquitetura da Informação\footnote{\url{http://www.cpai.unb.br/}} da Universidade de Brasília (CPAI), ao grupo de usuários
\emph{latex-br}\footnote{\url{http://groups.google.com/group/latex-br}} e aos novos voluntários do grupo \emph{\abnTeX}\footnote{\url{http://groups.google.com/group/abntex2} e
\url{http://abntex2.googlecode.com/}}~que contribuíram e que ainda
contribuirão para a evolução do \abnTeX.

Os agradecimentos principais são direcionados à Gerald Weber, Miguel Frasson, Leslie H. Watter, Bruno Parente Lima, Flávio de Vasconcellos Corrêa, Otavio Real Salvador, Renato Machnievscz\footnote{Os nomes dos integrantes do primeiro
projeto abn\TeX\ foram extraídos de \url{http://codigolivre.org.br/projects/abntex/}} e todos aqueles que contribuíram para que a produção de trabalhos acadêmicos conforme as normas ABNT com \LaTeX\ fosse possível.
}

% ----------------------------------------------------------
\newcommand{\DedicatoriaTexto}{%\color{blue}
\textit{ Este trabalho é dedicado às crianças adultas que,\\
   quando pequenas, sonharam em se tornar cientistas.}
	}

